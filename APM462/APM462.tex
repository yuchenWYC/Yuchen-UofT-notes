\documentclass[11pt]{article}
% Libraries.
\usepackage{amsmath}
\usepackage{amssymb}
\usepackage{pgfplots}
\usepackage{graphicx}
\usepackage{enumitem}
\usepackage{hyperref}
\usepackage{fancyhdr}
\usepackage{perpage}
\usepackage{float}
\usepackage{esint}
\usepackage[margin=1.5cm]{geometry}
\usepackage{../raina}

% Property settings.
\MakePerPage{footnote}
\pagestyle{headings}

% Attr.
\title{APM462 Fall 2019\\ Lecture Notes}
\author{\blue{\href{https://www.yuchenwyc.com}{Yuchen Wang}}}

\begin{document}
    \maketitle
    \tableofcontents
    \newpage

\section{Matrix Calculus}
\paragraph{Row v.s. Column Vector}
Our default rule is that every vector is a column vector unless explicitly stated otherwise. \\
This is also known as the \under{numerator layout}. \\
\under{Special case}: For $f: \real^n \rightarrow \real$, $Df$ is a $ 1 \times n$ matrix or row vector.

\subsection{Matrix Multiplication}
\paragraph{Definition 1.1.1}
Let A be $m \times n$, and B be $n \times p$, and let the product AB be
$$C = AB$$
then $C$ is a $m \times p$ matrix, with element $(i,j)$ given by
$$c_{ij} = \sum_{k=1}^n a_{ik}b_{kj}$$
for all $i = 1, 2, \hdots, m, j = 1,2,\hdots,p$.
\paragraph{Proposition 1.1.2}
Let $A$ be $m \times n$, and $x$ be $n \times 1$, then the typical element of the product
$$ z = Ax$$
is given by
$$z_i = \sum_{k=1}^n a_{ik}x_k$$
for all $i= 1,2,\hdots,m$.\\
Similarly, let $y$ be $m \times 1$, then the typical element of the product
$$z^T = y^TA$$
is given by
$$ z_i^T = \sum_{k=1}^n a_{ki}y_k$$
for all $i = 1, 2,\hdots, n$.  \\
Finally, the scalar resulting from the product
$$\alpha = y^T A x$$
is given by
$$\alpha = \sum_{j=1}^m\sum_{k=1}^n a_{jk}y_i x_k$$

\subsection{Partitioned Matrices}
\paragraph{Proposition 1.2.1}
Let $A$ be a square, nonsingular matrix of order $m$. Partition A as
$$ A = \begin{bmatrix}
	A_{11} & A_{12} \\
	A_{21} & A_{22}
\end{bmatrix}
$$
so that $A_{11}$ and $A_{22}$ are invertible. \\
Then
$$ A^{-1} = \begin{bmatrix}
	(A_{11} - A_{12}A_{22}^{-1}A_{21})^{-1} & -A_{11}^{-1}A_{12}(A_{22} - A_{21}A_{11}^{-1}A_{12})^{-1} \\
	-A_{22}^{-1}A_{21}(A_{11} - A_{12}A_{22}^{-1}A_{21})^{-1} & (A_{22} - A_{21}A_{11}^{-1}A_{12})^{-1}
\end{bmatrix}$$
\ti{proof: } \\
Direct multiplication of the proposed $A^{-1}$ and $A$ yields
$$A^{-1}A = I$$ \qed

\subsection{Matrix Differentiation}
\paragraph{Proposition 1.3.1}
$$\frac{\partial A}{\partial x} = \frac{\partial A^T}{\partial x}$$
\paragraph{Proposition 1.3.2}
Let $$y = Ax$$
where $y$ is $m\times 1$, $x$ is $n \times 1$, A is $m \times n$, and $A$ does not depend on $x$. Suppose that $x$ is a function of the vector $z$, while $A$ is independent of $z$. Then
$$\frac{\partial y }{\partial z} = A \frac{\partial x}{\partial z}$$

\paragraph{Proposition 1.3.3}
Let the \textcolor{red}{scalar} $\alpha$ be defined by
$$\alpha = y^T Ax$$
where $y$ is $m \times 1$, $x$ is $n \times 1$, $A$ is $m \times n$, and $A$ is independent of $x$ and $y$, then
$$\frac{\partial \alpha}{\partial x} = y^TA$$
and 
$$\frac{\partial \alpha}{\partial y} = x^TA^T$$

\paragraph{Proposition 1.3.4}
For the special case where the \textcolor{red}{scalar} $\alpha$ is given by the quadratic form
$$\alpha = x^TAx$$
where $x$ is $n \times 1$, $A$ is $n \times n$, and $A$ does not depend on $x$, then
$$\frac{\partial \alpha}{\partial x} = x^T (A + A^T)$$
\ti{proof: } \\
By definition
$$\alpha = \sum_{j=1}^n \sum_{i=1}^n a_{ij}x_ix_j$$
Differentiating with respect to the $k$th element of $x$ we have
$$\frac{\partial \alpha}{\partial x_k} = \sum_{j=1}^n a_{kj}x_J + \sum_{i=1}^n a_{ik}x_i$$
for all $k = 1,2, \hdots, n$, and consequently,
$$\frac{\partial \alpha}{\partial x} = x^TA^T + x^TA = x^T(A^T + A)$$ \qed

\paragraph{Proposition 1.3.4}
For the special case where \textcolor{red}{$A$ is a symmetric matrix} and
$$\alpha = x^TAx$$
where $x$ is $n \times 1$, $A$ is $n \times n$, and $A$ does not depend on $x$, then
$$\frac{\partial \alpha}{\partial x} = 2x^TA$$

\paragraph{Proposition 1.3.5}
Let the \textcolor{red}{scalar} $\alpha$ be defined by
$$\alpha = y^Tx$$
where $y$ is $n \times 1$, $x$ is $n \times 1$, and both $y$ and $x$ are functions of the vector $z$. Then
$$\frac{\partial \alpha}{\partial z} = x^T \frac{\partial y}{\partial z} + y^T \frac{\partial x}{\partial z}$$

\paragraph{Proposition 1.3.6}
Let the \textcolor{red}{scalar} $\alpha$ be defined by
$$\alpha = x^Tx$$
where $x$ is $n \times 1$, and $x$ is a functions of the vector $z$. Then
$$\frac{\partial \alpha}{\partial z} = 2x^T \frac{\partial y}{\partial z}$$

\paragraph{Proposition 1.3.7}
Let the \textcolor{red}{scalar} $\alpha$ be defined by
$$\alpha = y^TAx$$
where $y$ is $m \times 1$, $A$ is $m \times n$, $x$ is $n \times 1$, and both $y$ and $x$ are functions of the vector $z$, while $A$ does not depend on $z$. Then
$$\frac{\partial \alpha}{\partial z} = x^T A^T\frac{\partial y}{\partial z} + y^T A\frac{\partial x}{\partial z}$$

\paragraph{Proposition 1.3.8}
Let $A$ be an invertible, $m \times m$ matrix whose elements are functions of the scalar parameter $\alpha$. Then
$$\frac{\partial A^{-1}}{\partial \alpha} = -A^{-1}\frac{\partial A}{\partial \alpha} A^{-1}$$
\ti{proof: } \\
Start with the definition of the inverse
$$A^{-1}A = I$$
and differentiate, yielding
$$A^{-1}\frac{\partial A}{\partial \alpha} + \frac{\partial A^{-1}}{\partial \alpha}A = 0$$
rearranging the terms yields
$$\frac{\partial A^{-1}}{\partial \alpha}  = -A^{-1}\frac{\partial A}{\partial \alpha}A^{-1}$$\qed

\paragraph{Vector-by-vector Differentiation Identities 1.3.9}
 \begin{figure}[h]
	\centering
	\includegraphics[scale=0.6]{p1.png}
\end{figure}

\paragraph{Young's Theorem 1.3.10}
i.e. Symmetry of second derivatives
$$[\nabla_{xy}f(x,y)]^T = \nabla_{yx}f(x,y)$$
\ti{proof: } \\
This is straightforward by writing out the elements of the matrix. \qed


\section{Second-year Calculus Review}
functions $\real \rightarrow \real$
\subsection{Mean Value Theorem in 1 Dimension}
$g \in C^1$ on $\real$
$$\frac{g(x+h) - g(x)}{h} = g'(x + \theta h)$$
where $\theta \in (0,1)$ \\
Or equivalently,
$$g(x+h) = g(x) + hg'(x+\theta h)$$

\subsection{1st Order Taylor Approximation}
$g \in C^1$ on $\real$
$$g(x + h) = g(x) + hg'(x) + o(h)$$
where $o(h)$ is ``little $o$" of $h$, the error term.

\noindent Say a function $f(h) = o(h)$, this means $\underset{h \rightarrow 0}{\lim} \frac{f(h)}{h} = 0$\\
For example, for $f(h) = h^2$, we can say $f(h) = o(h)$, \\
since $\underset{h \rightarrow 0}{\lim} \frac{f(h)}{h} = \underset{h \rightarrow 0}{\lim} \frac{h^2}{h} = \underset{h \rightarrow 0}{\lim} h = 0$\\
\ti{proof: } (Use MVT): \\
WTS : $g(x+h) - g(x) - hg'(x) = o(h)$
\begin{align*}
	\underset{h \rightarrow 0}{\lim} \frac{[g(x+h) - g(x)] - hg'(x)}{h} &= \underset{h \rightarrow 0}{\lim} \frac{[hg'(x+\theta h)] - hg'(x)}{h} \\
	&=\underset{h \rightarrow 0}{\lim} \, g'(x + \theta h) - g'(x) \\
	&=\underset{h \rightarrow 0}{\lim} \, g'(x) - g'(x) \\
	&= 0
\end{align*}
\qed

\subsection{2nd Order Mean Value Theorem}
$g \in C^2$ on $\real$
$$g(x + h) = g(x) + hg'(x) + \frac{h^2}{2} g'(x+\theta h)$$
for some $\theta \in (0,1)$ \\\\
\ti{proof: } \\
WTS: $g(x+h) - g(x) - hg'(x) - \frac{h^2}{2}g''(x) = o(h^2)$
\begin{align*}
	\underset{h \rightarrow 0}{\lim} \frac{g(x+h) - g(x) - hg'(x) - \frac{h^2}{2}g''(x)}{h^2}
	&= \underset{h \rightarrow 0}{\lim} \frac{[\frac{h^2}{2}g'(x+\theta h)] - \frac{h^2}{2}g''(x)}{h^2}\\
	&=\underset{h \rightarrow 0}{\lim}\, \frac{1}{2}(g''(x+\theta h) - g''(x)) \\
	&=\underset{h \rightarrow 0}{\lim}\, \frac{1}{2}(g''(x) - g''(x)) \\
	&= 0
\end{align*}
\qed

multivariate functions: $\real^n \rightarrow \real$
\subsection{Recall: Definition of gradient}
Gradient of $f: \real^n \rightarrow \real$ at $x \in \real^n$ (denoted $\nabla f(x)$) if exists is a vector characterized by the property:
$$\underset{\vv \rightarrow \vo}{\lim} \frac{f(\vx+\vv)-f(\vx) - \nabla f(\vx) \cdot \vv}{||\vv||} = 0$$
In Cartesian coordinates, $\nabla f(\vx) = (\frac{\partial f}{\partial \vx_1}(\vx), \hdots, \frac{\partial f}{\partial \vx_n}(\vx))$

\subsection{Mean Value Theorem in $n$ dimension}
$f \in C^1$ on $\real^n$, then for any $\vx, \vv \in \real^n$,
$$f(\vx + \vv) = f(\vx) + \nabla f(\vx+ \theta \vv) \cdot \vv$$
for some $\theta \in (0,1)$\\\\
\ti{proof: } Reduce to 1-dimension case \\
$g(t) := f(\vx+t\vv), t \in \real$ \\
\begin{align*}
	g'(t) &= \frac{d}{dt}f(\vx+t\vv) \\
	&= \sum_{i=1}^n \frac{\partial f}{\partial \vx_i}(\vx + t\vv)\cdot \frac{d(\vx+t\vv)_i}{dt}
	\tag{by Chain Rule} \\
	&= \sum \frac{\partial f}{\partial \vx_i}(\vx + t\vv)\cdot \frac{d(\vx_i + t\vv_i)}{dt} \\
	&= \sum \frac{\partial f}{\partial \vx_i} (\vx + t\vv)\cdot \vv_i \\
	&= \nabla f(\vx + t\vv)\cdot \vv \tag{*}
\end{align*}
$g \in C^1$ on $\real$\\
Using MVT in $\real$:
\begin{align*}
f(\vx+\vv) &= g(1) \\
&= g(0+1) \\
&= g(0) + 1g'(0+\theta 1) \tag{$\theta \in (0,1)$}\\
&= g(0) + g'(\theta) \\
&= f(\vx) + \nabla f(\vx+\theta \vv) \cdot \vv \tag{by (*)}
\end{align*}
\qed

\subsection{1st Order Taylor Approximation in $\real^n$}
$f \in C^1$ on $\real^n$
$$f(\vx + \vv) = f(\vx) + \nabla f(\vx) \cdot \vv + o(||\vv||)$$
\ti{proof: } \\
\begin{align*}
	\underset{||\vv|| \rightarrow 0}{\lim} \frac{[f(\vx + \vv) - f(\vx)] - \nabla f(\vx) \cdot \vv }{||\vv||} 
	&= \underset{||\vv|| \rightarrow 0}{\lim} \frac{[\nabla f(\vx + \theta \vv) \cdot \vv] - \nabla f(\vx) \cdot \vv}{||\vv||} \\
	&= \underset{||\vv|| \rightarrow 0}{\lim} [\nabla f(\vx + \theta \vv) - \nabla f(\vx)] \cdot \frac{\vv}{||\vv||} \\
	&= 0 \tag{$\frac{\vv}{||\vv||}$ is a unit vector, remains 1}
\end{align*}
\qed
\subsection{2nd Order Mean Value Theorem in $\real^n$}
$f \in C^2$ on $\real ^n$
$$ f(\vx + \vv) = f(\vx) + \nabla f(\vx) \cdot \vv + \frac{1}{2}\vv^T\nabla^2 f(\vx + \theta \vv) \cdot \vv$$
\paragraph{Remarks}
In this course, \textcolor{red}{$\nabla^2$ means Hessian, not Laplacian.}
$$\nabla^2 f(\vx) =  \left( \frac{\partial^2 f}{\partial \vx_i \partial \vx_j}\right)_{1\leq i, j\leq n}(\vx) = \begin{pmatrix}
	\frac{\partial f}{\partial_1^2} & \frac{\partial f}{\partial_1 \partial_2} & \hdots \\
	\frac{\partial f}{\partial_2 \partial_1} & \hdots \\
	\vdots
\end{pmatrix}$$
The Hessian matrix is \textcolor{blue}{symmetric}. This is sometimes called \under{Clairaut's Theorem}.\\
\under{note}: $\vv^T \nabla^2 f(\vx)\vv = \sum_{1 \leq i, j\leq n} \frac{\partial^2 f}{\partial \vx_i \partial \vx_j}f(\vx)\vv_i \vv_j$

\subsection{2nd Order Taylor Approximation in $\real^n$}
$f \in C^2$ on $\real^n$ \\
$$f(\vx + \vv) = f(\vx) + \nabla f(\vx) \cdot \vv + \frac{1}{2}\vv^T\nabla^2 f(\vx) \vv + o(||\vv||^2)$$
\ti{proof: } \\
\begin{align*}
	\underset{||\vv|| \rightarrow 0}{\lim} \frac{[f(\vx + \vv) - f(\vx)] - \nabla f(\vx)\cdot \vv - \frac{1}{2} \vv^T \nabla^2 f(\vx)\vv}{||\vv||^2} 
	&= \underset{||\vv|| \rightarrow 0}{\lim} \frac{[\frac{1}{2}\vv^T\nabla^2 f(\vx + \theta \vv) \cdot \vv] - \frac{1}{2}\vv^T\nabla^2 f(\vx) \cdot \vv}{||\vv||^2} \tag{By 2nd MVT}\\
	&=  \underset{||\vv|| \rightarrow 0}{\lim} \frac{1}{2}(\frac{\vv}{||\vv||})^T[\nabla^2 f(\vx + \theta \vv) - \nabla^2 f(\vx)](\frac{\vv}{||\vv||}) \\
	&= 0
\end{align*}
\qed 
\subsection{Geometric Meaning of Gradient}
$f: \real^n \rightarrow \real$\\
Rate of change of $f$ at $\vx$ in direction $\vv \, (||\vv||=1) = \frac{d}{dt}|_{t=0}f(\vx + t\vv)$
\begin{align*}
	\frac{d}{dt}|_{t=0}f(\vx + t\vv) &= \nabla f(\vx + t\vv) \cdot \vv |_{t=0} \\
	&= \nabla f(\vx)\cdot \vv \\
	&= |\nabla f(\vx)| |\vv| \cos \theta\\
	&= |\nabla f(\vx)| \cos \theta \tag{$||\vv||  = 1$}
\end{align*}
maximized at $\theta = 0$ \\
So $\nabla f(\vx)$ points in the direction of steepest ascent.

\subsection{Implicit Function Theorem}
$f: \real^{n+1} \rightarrow \real \in C^1$ \\
Fix $(\va,b) \in \real^n \times \real$ s.t. $f(\va,b) = 0$. \\
If $\nabla f(\va,b) \neq 0$, then $\{(\vx, y) \in (\real^n \times \real) | f(\vx, y) = 0\}$ is locally (near $(\va, b)$) the graph of a function.

\subsection{Level Sets of $f$}
$c$-level set of $f$ := $\{ \vx \in \real^n | f(\vx) = c\}$
\paragraph{Fact} gradient $\nabla f(\vx_0) \perp$ level curve (through $\vx_0$)

\section{Convex Sets \& Functions}
\subsection{Definitions}
\paragraph{Definition of Convex Set}
$\Omega \subseteq \real^n$ is a \under{convex set} if
$\vx_1, \vx_2 \in \Omega \Rightarrow s\vx_1 + (1-s)\vx_2 \in \Omega$ where $s \in [0,1]$
\paragraph{Definition of Convex Function}
A function $f: \text{convex } \Omega \subseteq \real^n$ is \under{convex} if 
$$f(s\vx_1 + (1-s)\vx_2) \leq sf(\vx_1) + (1-s)f(\vx_2)$$
for all $\vx_1, \vx_2 \in \Omega$ and all $s \in [0, 1]$
\paragraph{Remarks}
Second line above (or equal to) the graph
\paragraph{Definition of Concave Function}
A function $f$ is \under{concave} if $-f$ is convex.

\subsection{Basic Properties of Convex Functions}
Let $\Omega \subseteq \real^n$ be a convex set.
\begin{enumerate}
	\item $f_1, f_2$ are convex functions on $\Omega \Rightarrow f_1 + f_2$ is a convex function on $\Omega$.
	\item $f$ is a convex function, $a \geq 0 \Rightarrow af$ is a convex function.
	\item $f$ is a convex function on $\Omega \Rightarrow$ The sublevel sets of $f$, $SL_c:=\{\vx \in \real^n | f(\vx) \leq c\}$ is convex.
\end{enumerate}
\under{\ti{proof of (3):}} \\
Let $x_1, x_2 \in SL_C$, so that $f(x_1) \leq c$ and $f(x_2) \leq c$. \\
WTS: $sx_1 + (1-s)x_2 \in SL_c$ for any $s \in [0,1]$
\begin{align*}
	f(sx_1 + (1-s)x_2) &\leq sf(x_1) + (1-s)f(x_2) \tag{$f$ is convex}\\
	&\leq sc + (1-s)c \\
	&= c \\
\Rightarrow sx_1 + (1-s)x_2 \in SL_c
\end{align*}
\qed

\paragraph{Example of a convex function}
Let $f: \real \rightarrow \real, \,f(x) = |x|$ \\
Let $x_1, x_2 \in \real, s \in [0,1]$ \\
Then
\begin{align*}
	f(sx_1 + (1-s)x_2) &= |sx_1 + (1-s)x_2| \\
	&\leq |sx_1| + |(1-s)x_2| \tag{by Triangle Inequality}\\
	&= s|x_1| + (1-s)|x_2| \\
	&= sf(x_1) + (1-s)f(x_2)
\end{align*}
Then $f$ is a convex function.
\paragraph{Theorem - Characterization of $C^1$ convex functions}
Let $f: \text{convex subset of $\real^n$ } \Omega \rightarrow \real$ be a $C^1$ function. \\
Then,\\
$f$ is convex $\iff$ $f(y) \geq f(x) + \nabla f(x)\cdot(y-x)$ 
for all $x,y \in \Omega$
\paragraph{Remarks}
Tangent line below the graph.\\\\
\ti{proof: } \\
($\Rightarrow$) \\
$f$ is convex, then by definition,
\begin{align*}
f(s\vx_1 + (1-s)\vx_2) &\leq sf(\vx_1) + (1-s)f(\vx_2)\\
f(s\vx_1 + (1-s)\vx_2) - f(\vx_2) &\leq s(f(\vx_1) - f(\vx_2)) \\
\frac{f(s\vx_1 + (1-s)\vx_2) - f(\vx_2)}{s} &\leq f(\vx_1) - f(\vx_2) \\
\underset{s \rightarrow 0}{\lim} \frac{f(\vx_2 + s(\vx_1 - \vx_2)) - f(\vx_2)}{s} &\leq f(\vx_1) - f(\vx_2) \\
\nabla f(\vx_2)\cdot(\vx_1 - \vx_2) &\leq f(\vx_1) - f(\vx_2) \tag{since $\frac{d}{ds}|_{s=0} f(\vx_2 + s(\vx_1 - \vx_2)) = \nabla f(\vx_2) \cdot (\vx_1 - \vx_2)$}\\
f(\vx_2) + \nabla f(\vx_2)\cdot(\vx_1 - \vx_2) &\leq f(\vx_1)\\
f(\vx) + \nabla f(\vx) \cdot (\vy-\vx) &\leq f(\vy)
\end{align*}
where $0\leq s \leq 1$ \\
($\Leftarrow$)\\
Fix $\vx_0, \vx_1 \in \Omega$ and $s \in (0,1)$ \\
Let $x = s\vx_0 + (1-s)\vx_1$ \\
$$\begin{cases}
	f(\vx_0) &\geq f(\vx) + \nabla f(\vx) \cdot (\vx_0 - \vx) \\
	&= f(\vx) + \nabla f(\vx) \cdot (1-s)(\vx_0 - \vx_1) \\
	f(\vx_1) &\geq f(\vx) + \nabla f(\vx) \cdot (\vx_1 - \vx) \\
	&= f(\vx) + \nabla f(\vx) \cdot s(\vx_1 - \vx_0) \\
\end{cases}$$
	$$\begin{cases}sf(x_0) &\geq sf(\vx) + \nabla f(\vx) \cdot (1-s) \cdot s(\vx_0 - \vx_1) \\
	(1-s)f(\vx_1) &\geq (1-s)f(\vx) + \nabla f(\vx) \cdot (1-s)\cdot s(\vx_1 - \vx_0)  \end{cases}$$
Then
$$sf(\vx_0) + (1-s)f(\vx_1) \geq f(x) + 0$$
Then $f$ is convex.
\qed

\subsection{Criterions for convexity}
\paragraph{$C^1$ criterion for convexity}
$$f: \Omega \rightarrow \real \text{ is convex } \iff f(y) \geq f(x) + \nabla f(x) \cdot (y-x)$$
for all $x, y \in \Omega$

\paragraph{Theorem: $C^2$ criterion for convexity}
Let $f \in C^2$ on $\Omega \subseteq \real^n$ (here we assume $\Omega \subseteq \real^n$ is a convex set containing an interior point) \\
Then $$f \text{ is convex on } \Omega \iff \nabla^2 f(x) \geq 0$$ for all $x \in \Omega$


\paragraph{Remark 1}
Let $A$ be an $n \times n$ matrix.\\
``$A \geq 0$" means $A$ is positive semi-definite:
$$v^TAv \geq 0$$ for all $v \in \real^n$

\paragraph{Remark 2}
In $\real$, 
$$f \text{ is convex } \iff f'(x) \geq 0$$ for all $x \in \Omega$ \\
(``concave up" in first year calculus) \\\\
\ti{\under{proof for Theorem:}} \\
Recall 2nd order MVT: \\
$$f(y) = f(x) + \nabla f(x)\cdot (y-x) + \frac{1}{2}(y-x)^T \nabla^2 f(x+s(y-x))\cdot (y-x)$$
for some $s \in [0,1]$ \\
($\Leftarrow$) \\
Since $\nabla^2 f(x) \geq 0$, then 
$$\frac{1}{2}(y-x)^T \nabla^2 f(x+s(y-x))\cdot (y-x) \geq 0$$
Then
$$f(y) \geq f(x) + \nabla f(x) \cdot (y-x)$$
for all $x, y \in \Omega$. \\
Then by $C^1$ criterion,
$f$ is convex. \\
($\Rightarrow$) \\
Assume $f$ is convex on $\Omega$. \\
Suppose for contradiction that $\nabla^2 f(x)$ is not positive semi-definite at some $x \in \Omega$. \\
Then $\exists v \neq 0$ s.t. $v^T \nabla^2f(x) v < 0$
$v$ could be arbitrarily small and $>0$\\
Let $y = x + v$, then
$$(y-x)^T\nabla^2 f(x + s(y-x))\cdot (y-x) < 0$$
for all $s \in [0,1]$ \\
Then by MVT,
$$f(y) < f(x) + \nabla f(x)\cdot(y-x)$$
for some $x, y \in \Omega$, and this contradicts the $C^1$ criterion. \qed

\subsection{Minimization and Maximization of Convex Functions}
\paragraph{Theorem}
$f:$ convex $\Omega \subseteq \real^n \rightarrow \real$ is a convex function. \\
Suppose $\Gamma := \{ x \in \Omega | f(x) = \underset{\Omega}{\min} \, f(x)\} \neq \emptyset$\\
(i.e. minimizer exists) \\
Then $\Gamma$ is a convex set, and any local minimum of $f$ is a global minimum of $f$. \\
\ti{proof: } \\\\
Let $m = \underset{\Omega}{\min} f(x)$.
$$\Gamma = \{ x \in \Omega | f(x) = m\} = \{ x \in \Omega | f(x) \leq m \}$$
(sublevel set) \\
Then by Basic Properties of Convex Sets, $\Gamma$ is convex. \\
Let $x$ be a local minimum of $f$.\\
Suppose for contradiction that $\exists y$ s.t. $f(y) < f(x)$ \\
(i.e. $x$ is not a global minimum)
\begin{align*}
	f(sy + (1-s)x) &\leq sf(y) + (1-s)f(x) \\
	&< sf(x) + (1-s)f(x) \tag{$f(y) < f(x)$}\\
	&= f(x)
\end{align*}
for all $s \in (0,1)$ \\
As $s$ approaches 0, $s$ approaches $x$. \\
Then we have
$\underset{s \rightarrow 0}{\lim} f(sy + (1-s)x) = f(x) < f(x)$.\\
which is a contradiction. \qed


\paragraph{Theorem}
If $f: \Omega \subseteq \real^n \rightarrow \real$ is a convex function, and $\Omega$ is convex and compact, then $$\underset{\Omega}{\max} f = \underset{\partial \Omega}{\max} f$$
\paragraph{Remarks}
Maximum value of $f$ is attained (also) on the boundary of $\Omega$ \\
\ti{proof: } \\
Since $\Omega$ is closed, $\partial \Omega \subseteq \Omega$, so $\underset{\Omega}{\max} f \geq \underset{\partial \Omega}{\max} f$.\\
Suppose $f(x_0) = \underset{\Omega}{\max} \, f$ for some $x_0 \notin \partial \Omega$.
Let $L$ be an arbitrary line through $x_0$. \\
By convexity and compactness of $\Omega$, L meets $\partial \Omega$ at two points $x_1, x_2$. 
\\
Let $x_0 + sx_1 + (1-s)x_2$ for $ s\in (0,1)$
\begin{align*}
	f(x_0) &= f(sx_1 + (1-s)x_2) \\
	&\leq sf(x_1) + (1-s)f(x_2) \tag{$f$ convex}\\
	&\leq \max\{f(x_1), f(x_2)\} \\
	&\leq \underset{\partial \Omega}{\max} \, f\\
	&\leq \underset{\Omega}{\max} \, f = f(x_0) 
\end{align*}
This implies that $$\underset{\Omega}{\max} \, f = \underset{\partial \Omega}{\max} \, f$$
as wanted.
\qed

\paragraph{Example}
$$|ab| \leq \frac{1}{p}|a|^p + \frac{1}{q}|b|^q$$
where $p,q > 1$ s.t. $\frac{1}{p} + \frac{1}{q} = 1$.\\
\under{Special cases:}
\begin{enumerate}
	\item $$p=q=2, |ab| \leq \frac{|a|^2 + |b|^2}{2}$$ 
	\item 
	$$p=3, q=\frac{3}{2}, |ab| \leq \frac{1}{3}|a|^3 + \frac{2}{3}|b|^{\frac{3}{2}}$$
\end{enumerate}
\ti{proof: } \\
Since function $f(x) = -\log (x)$ is convex, then
\begin{align*}
	(-\log)|ab| &= (-\log)|a| + (-\log)|b| \\
	&= \frac{1}{p}(-\log)|a|^p + \frac{1}{q}(-\log)|b|^q \\
	&\geq (-\log)(\frac{1}{p}|a|^p + \frac{1}{q}|b|^q)\\
	(-\log)|ab| &\geq (-\log)(\frac{1}{p}|a|^p + \frac{1}{q}|b|^q) \\
	\log|ab| &\leq \log(\frac{1}{p}|a|^p + \frac{1}{q}|b|^q) \\
	|ab| &\leq \frac{1}{p}|a|^p + \frac{1}{q}|b|^q \tag{exponential function is increasing}
\end{align*}
\qed 
\section{Basics of Unconstrained Optimization}
\subsection{Extreme Value Theorem}
Suppose $f: \real^n \rightarrow \real$ is continuous, and compact set $K \subseteq \real^n$
Then the problem
$$\underset{x \in K}{\min} \, f(x)$$ has a solution.

\paragraph{Recall}
\begin{enumerate}
	\item $$K \subseteq \real^n \text{ compact} \iff K \text{ closed and bounded}$$
	\item If $h_1, \hdots, h_k$ and $g_1, \hdots, g_m$ are continuous functions on $\real^n$, then the set of all points $x \in \real^n$ s.t.
	$$\begin{cases}
		h_i(x) = 0  &\text{for all $i$} \\
		g_j(x) \leq 0 &\text{for all $j$}
	\end{cases}$$
	is a closed set.
	\item If such a set is also bounded, then it is compact.
\end{enumerate}

\paragraph{Example}
$$\{(x,y) \in \real^2 | x^2 + y^2 -1 = 0 \}$$
by (2), this is a closed set \\
by (3), this is a compact set.

\paragraph{Remarks}
$f: \Omega \subseteq \real^n \rightarrow \real$ convex does not imply $f$ is continuous.

\subsection{Unconstrained Optimization}
$$\underset{x \in \Omega \subseteq \real^n}{\min f(x)}$$
typically
\begin{enumerate}
	\item $\Omega \subseteq \real^n$
	\item $\Omega = \real^n$
	\item $\Omega = $ open
	\item $\Omega = \overline{\text{open}}$ 
\end{enumerate}

\paragraph{Remark}
\begin{enumerate}
	\item $\max f(x) = - (\min - f(x))$
	\item $\min f(x) = - (\max - f(x))$
\end{enumerate}

\paragraph{Definition: local minimum}
We say that $f$ has a \under{local minimum} at a point $x_0 \in \Omega$ if
$$f(x_0) \leq f(x)$$ for all $x \in B_{\Omega}^\varepsilon (x_0)$, where $B_{\Omega}^\varepsilon (x_0) = \{ x \in \Omega: |x - x_0| < \varepsilon\}$ which is an open ball around $x_0$ inside $\Omega$ of radius $\varepsilon >0$. \\
We say that $f$ has a \under{strict local minimum} at a point $x_0 \in \Omega$ if
$$f(x_0) < f(x)$$ for all $x \in B_{\Omega}^\varepsilon (x_0) \setminus \{x_0\}$

\subsection{1st order necessary condition for local minimum}
\paragraph{Theorem} Let $f$ be a $C^1$ function on $\Omega \subseteq \real^n$. If $x_0 \in \Omega$ is a local minimum of $f$, then
$$ \nabla f(x_0) \cdot v \geq 0$$
for all feasible directions $v$ at $x_0$

\paragraph{Definition: feasible direction}
$v \in \real^n$ is a \under{feasible direction} at $x_0 \in \Omega$ if $$x_0 + sv \in \Omega$$ for all $0 \leq s \leq \bar{s}$ where $\bar{s} \in \real$
\paragraph{Remarks}
\textcolor{red}{Feasible directions go into the set.}
\paragraph{Corollary}
Special case: If $\Omega = \real^n$ is an open set, then any direction is a feasible direction.
Then $x_0$ is a local minimum of $f$ on $\Omega$ implies that $\nabla f(x_0) \cdot v \geq 0$ for all $v \in \real^n$.
$$\begin{cases}
	\nabla f(x_0) \cdot v \geq 0 \\
	\nabla f(x_0) \cdot (-v) \geq 0 \iff \nabla f(x_0) \cdot v \leq 0
\end{cases} \implies \nabla f(x_0) \cdot v = 0 \text{ for all $v \in \real^n$}$$
$$\implies \nabla f(x_0) = 0$$
\begin{proof}
[ ]
\end{proof}

\subsection{2nd order necessary condition for local minimum}
$f \in C^2, \Omega \subseteq \real^n$ \\
If $x_0 \in \Omega$ is a local minimum of $f$ on $\Omega$, then
\begin{enumerate}
	\item $\nabla f(x_0) \cdot v \geq 0$ for al feasible directions $v$ at $x_0$
	\item If $\nabla f(x_0) \cdot v = 0$, then $v^T \nabla^2 f(x_0) v \geq 0$
	(function curves up)
\end{enumerate}
\begin{proof}
[]
\end{proof}

\paragraph{Remark}
If $x_0$ is an interior point of $\Omega$, then
$$\nabla f(x_0) = 0, \quad \nabla^2 f(x_0)\geq 0$$
$$f'(x_0) = 0, \quad f''(x_0) \geq 0$$

\paragraph{Definition: principal minor}
Let $A$ be an $n \times n$ matrix. A $k \times k$ submatrix of $A$ formed by deleting $n-k$ rows of $A$, and the \textcolor{red}{same} $n - k$ columns of $A$, is called \under{principal submatrix} of $A$. The determinant of a principal submatrix of $A$ is called a \under{principal minor} of $A$.

\paragraph{Definition: leading principal minor}
Let $A$ be an $n \times n$ matrix. The $k$th order principal submatrix of $A$ obtained by deleting the \textcolor{red}{last} $n-k$ rows and columns of $A$ is called the $k$-th order \under{leading principal submatrix} of $A$, and its determinant is called a \under{leading principal minor} of $A$.

\paragraph{Definition: positive definiteness (Sylvester's Criterion)}
A $n \times n$ matrix $A$ is \\
\begin{enumerate}
	\item \under{positive definite} if $v^TAv > 0$ for all $v \neq 0 \iff $ all eigenvalues $>0$ $\iff$ \textcolor{red}{all leading principle minors $> 0$}
	\item \under{positive semi-definite} if $v^TAv \geq 0$ for all $v \iff $ all eigenvalues $\geq 0$ $\iff$ \textcolor{red}{all principle minors $\geq 0$}
\end{enumerate}

\paragraph{Lemma}
Suppose $\nabla^2 f(x_0)$ is positive definite, then
$$\exists a > 0 \,\, s.t. \,\, v^T\nabla^2f(x_0)v \geq a||v||^2 \quad \forall v$$


\subsection{2nd order sufficient condition (for interior points)}
$f \in C^2$ on $\Omega$ \\
If $\begin{cases}
	\nabla f(x_0) = 0 \\
	\nabla^2 f(x_0) > 0
\end{cases},$ then $x_0$ is a strict local minimum. \\
\begin{proof}
	[ ]
\end{proof}


\section{Optimization with Equality Constraints}

\subsection{Definitions of Related Spaces}
\paragraph{Definition 5.1.1: surface}
$$M =  \text{``surface"}  = \{ x \in \real^n | h_1(x) = 0, \hdots, h_k(x) = 0\}$$ 
where $h_i \in C^1$

\paragraph{Definition 5.1.2: differentiable curve on surface}
A \under{differentiable curve} on surface $M \subseteq \real^n$ is a $C^1$ function
$$ x: (-\epsilon, \epsilon) \rightarrow M: s \mapsto x(s)$$ 
\paragraph{Remarks}
\begin{enumerate}
	\item Let $x(s)$ be a differentiable curve on $M$ that passes through $x_0 \in M$, say $x(0) = x_0$. The vector $v = \frac{d}{ds}|_{s=0} \, x(0)$ touches M ``tangentially". We say $v$ is generated by $x(s)$.
	\item In previous calculus courses, differentiable curves are often referred to as parameterizations.
\end{enumerate}
\paragraph{Definition 5.1.3: tangent vector}
Any vector $v$ which is generated by some differentiable curve on $M$ through $x_0$ is called a \under{tangent vector}.

\paragraph{Definition 5.1.4: tangent space}
\under{Tangent space} to the surface $M$ at point $x_0$ is
$$T_{x_0} M = \{ \text{all tangent vectors to $M$ at $x_0$} \} \\
= \{v \in \real^n: v = \frac{d}{ds} |_{s=0} \, x(s)\}$$
where $x(s)$ is a differentiable curve on $M$ s.t. $x(0) = x_0$

\paragraph{Remarks}
The zero vector is contained in all tangent spaces.

\paragraph{Definition 5.1.5: T-space}
$$T_{x_0} = \{ x\in \real^n: x^T \nabla h_i(x_0) = 0 \, \forall i\} = Span\{ \nabla h_1(x_0), \hdots, \nabla h_k(x_0)\}^\perp$$

\paragraph{Definition 5.1.6: regular point}
$x_0 \in M$ is a \under{regular point} (of the constraints) if $\{ \nabla h_1(x_0), \hdots. \nabla h_k(x_0) \}$ are linearly independent.
\paragraph{Remark}
If there is only one constraint $h$, then $x_0$ is regular if and only if $\nabla h(x_0) \neq 0$.

\paragraph{When does the T-space equivalent to the tangent space?}
\textcolor{red}{When $x_0$ is a regular point (of the constraints).}

\paragraph{Theorem 5.1.7}
Suppose $x_0$ is a regular point s.t. $ M = \{ x \in \real^n: h_i(x) = 0 \, \forall i\}$. Then
$$T_{x_0}M = T_{x_0}$$

\paragraph{Lemma 5.1.8}
$f, h_1, \hdots, h_k \in C^1$ on open $\Omega \subseteq \real^n$ \\
$ M = \{ x \in \real^n: h_i(x) = 0 \, \forall i\}$ \\
Suppose $x_0 \in M$ is a local minimum of $f$ on $M$, then
$$\nabla f(x_0) \perp T_{x_0}M \iff \nabla f(x_0)\cdot v = 0$$ for all $v \in T_{x_0}M$

\subsection{Lagrange Multipliers: 1st order necessary condition for local minimum}
$f, h_1, \hdots, h_k \in C^1$ on open $\Omega \subseteq \real^n$. \\
Let $x_0$ be a regular point of the constraints $ M = \{ x \in \real^n: h_i(x) = 0 \, \forall i\}$. \\
Suppose $x_0$ is a local minimum of $f$ on $M$, then $\exists \lambda_1, \hdots, \lambda_k \in \real$ s.t.
$$\nabla f(x_0) + \lambda_1 \nabla h_1(x_0) + \hdots + \lambda_k \nabla h_k(x_0) = 0$$
\begin{proof}
	$x_0$ regular implies that $$\tanspace = \tspace = Span\{\nabla h_1(x_0), \hdots, \nabla h_k(x_0)\}^\perp$$
	By Lemma 5.1.8, $x_0$ is a loc min implies that $$\nabla f(x_0) \perp \tanspace$$
	Then $$\nabla f(x_0) \in (\tanspace)^\perp = Span\{\nabla h_i (x_0)\}^{\perp\perp} = Span\{ \nabla h_i(x_0)\}$$
	Then 
	$$\nabla f(x_0) = -\lambda_1 \nabla h_1(x_0) - \hdots - \lambda_k \nabla h_k(x_0)$$
	for some $\lambda_i \in \real$
\end{proof} 

\subsection{2nd order necessary condition for local minimum}
$f, h_1, \hdots, h_k \in \textcolor{red}{C^2}$ on open $\Omega \subseteq \real^n$. \\
Let $x_0$ be a regular point of the constraints $ M = \{ x \in \real^n: h_i(x) = 0 \, \forall i\}$. \\
Suppose $x_0$ is a local minimum of $f$ on $M$, then
\begin{enumerate}
	\item $$\nabla f(x_0) + \sum_{i=1}^k \lambda_i \nabla h_i(x_0) = 0$$
	for some $\lambda_i \in \real$
	\item \textcolor{blue}{$$\nabla^2 f(x_0) + \sum \lambda_i \nabla^2 h_i(x_0) \geq 0$$ on $T_{x_0}M$}
\end{enumerate}

\subsection{2nd order sufficient condition for local minimum}
$f, h_1, \hdots, h_k \in \textcolor{red}{C^2}$ on open $\Omega \subseteq \real^n$. \\
Let $x_0$ be a regular point of the constraints $ M = \{ x \in \real^n: h_i(x) = 0 \, \forall i\}$. \\
If $\exists \lambda_i \in \real$ s.t.
\begin{enumerate}
	\item $$ \nabla f(x_0) + \sum \lambda_i \nabla h_i(x_0) = 0$$
	\item $$ \nabla^2 f(x_0) + \sum \lambda_i \nabla^2 h_i(x_0) > 0$$ on $T_{x_0} M$
\end{enumerate}
Then $x_0$ is a strict local minimum.\\\\
\begin{proof}
	Recall that (2) means $[\nabla^2 f(x_0) + \sum \lambda_i h_i(x_0)]$ is pos-def on $T_{x_0}M$. \\
	Then $ \exists a > 0$ s.t. $v^T [\nabla^2 f(x_0) + \sum \lambda_i h_i(x_0)] v \geq a||v||^2$ for all $v \in \tanspace$. \\
	Let $x(s) \in M$ be a curve s.t. $x(0) = x_0$ and $v = x'(0)$. \\
	WLOG, $||x'(0)|| = 1$.\\
	By 2nd order Taylor, 
	\begin{align*}
		f(x(s)) - f(x(0)) &= s\frac{d}{ds}|_{s=0} f(x(s)) + \frac{1}{2} s^2 \frac{d}{ds^2} |_{s=0} f(x(s)) + o(s^2)\\
		&= s\frac{d}{ds}|_{s=0} [f(x(s)) + \sum_i \lambda_i h_i(x(s)) ]+ \frac{1}{2} s^2 \frac{d}{ds^2}|_{s=0} [f(x(s)) + \sum_i \lambda_i h_i(x(s)) ] + o(s^2) \tag{$\sum_i \lambda_i h_i(x(s)) = 0$}\\
		&=  s[\nabla f(x_0) + \sum \lambda_i \nabla h_i(x_0)] \cdot x'(0) + \frac{1}{2}s^2x'(0)^T[\nabla^2f(x_0) + \sum \lambda_i \nabla^2h_i(x_0)] x'(0) + o(s^2) \\
		&= 0 + \frac{1}{2} s^2 v^T[\nabla^2f(x_0) + \sum \lambda_i \nabla^2h_i(x_0)]v + o(s^2) \\
		&\geq \frac{1}{2}s^2 a||v||^2 + o(s^2) \\
		&= \frac{1}{2}s^2 a + o(s^2) \\
		&= s^2 [\frac{a}{2} + \frac{o(s^2)}{s^2}] > 0
	\end{align*}
	for small $s >0$, since $\frac{a}{2} > 0$ and $\underset{s \rightarrow 0}{\lim} \frac{o(s^2)}{s^2} = 0$ \\
	Then $f(x(s)) > f(x_0)$ for small $s > 0$
	Then $x_0$ is a strict local min of $f$. 
\end{proof}


\section{Optimization with Inequality Constraints}
\paragraph{Problem}
open $\Omega \subseteq \real^n$\\
$f: \Omega \rightarrow \real$ \\
$h_1, \hdots, h_k: \Omega \rightarrow \real$ \\
$g_1, \hdots, g_l: \Omega \rightarrow \real$ \\
$$\begin{cases}
	\min f(x) \\
	x \in \Omega \text{ subject to } 
\begin{cases}
	h_1(x) = 0, \hdots, h_k(x) = 0 \\
	g_1(x) \leq 0, \hdots, g_l(x) \leq 0
\end{cases}
\end{cases}(*)$$

\paragraph{Definition 1: activeness}
Let $x_0$ satisfy the constraints. \\
We say that the constraint $g_i(x) \leq 0$ is \under{active} at $x_0$ if $g_i(x_0) = 0$. \\ It is \under{inactive} at $x_0$ if $g_i(x_0) < 0$.\\
\paragraph{Definition 2: regular point}
Suppose for some $l' \leq l$:
$$g_1(x) \leq 0, \hdots, g_{l'}(x) \leq 0; \, g_{l' + 1}(x) \leq 0, \hdots, g_l(x) \leq 0$$
where $g_1, \hdots g_{l'}$ active and the rest inactive. \\
We say that $x_0$ is a \under{regular point} of the constraints if \\
$\{\nabla h_1(x_0), \hdots, \nabla h_k(x_0), \nabla g_1(x_0), \hdots, \nabla g_{l'}(x_0)\}$ is linearly independent.

\subsection{Kuhn-Tucker conditions: 1st order necessary condition for local minimum}
open $\Omega \subseteq \real^n$\\
$f: \Omega \rightarrow \real$ \\
$h_1, \hdots, h_k, g_1, \hdots, g_l: C^1 \in \Omega$ \\
Suppose $x_0 \in \Omega$ is a regular point of the constraints which is a local minimum, then
\begin{enumerate}
	\item $$\nabla f(x_0) + \sum_{i=1}^k \lambda_i \nabla h_i(x_0) + \sum_{j=1}^l \mu_j \nabla g_j(x_0) = 0$$ for some $\lambda_i \in \real$ and \textcolor{red}{$\mu_j \geq 0$}
	\item $\mu_j g_j(x_0) = 0$
\end{enumerate}

\paragraph{Remark 1}
Given $x_0$, 
$$
\begin{cases}
g_j(x) \leq 0 \text{ active at } x_0 \implies g_j(x_0) = 0 \implies \mu_jg_j(x_0) = 0 \\
g_j(x) \leq 0 \text{ inactive at } x_0 \implies g_j(x_0) < 0 \implies \mu_j = 0
\end{cases}
$$
$\implies \mu_j = 0$ for all inactive $g_j$ at $x_0$

\paragraph{Remark 2}
It is possible for an active constraint to have zero multiplier.

\paragraph{Remark 3}
$\mu_j \geq 0$ because $\nabla f$ and $\nabla g$ have opposite directions at a local minimum $x_0$. \\
$$\nabla f(x_0) + \mu \nabla g(x_0) = 0 \implies \nabla f(x_0) = -\mu \nabla g(x_0) \implies -\mu < 0 \implies \mu > 0 $$
\textcolor{red}{Is this true?}

\paragraph{Idea of proof}
$x_0$ is a local min of $f$ subject to (*) \\
$\implies x_0$ is a local min for equality constraints $h_1(x) = 0, \hdots, h_k(x) = 0 +$ active inequality constraints $g_1(x) \leq 0, \hdots, g_{l'}(x) \leq 0$ \\
$\implies x_0$ is a local min for $h_1(x) = 0, \hdots, h_k(x) = 0 + g_1(x) = 0, \hdots, g_{l'}(x) = 0$ 
$\implies \nabla f(x_0) + \sum_{i=1}^k \lambda_i \nabla h_i(x_0) + \sum_{j=1}^{l'} \mu_j \nabla g_j(x_0) = 0$\\
for some $\lambda_i \in \real$ and $\mu_j \in \real$.\\
Let $\mu_j = 0$ for $j = l' + 1, \hdots, l$,
then 
$$\nabla f(x_0) + \sum_{i=1}^k \lambda_i \nabla h_i(x_0) + \sum_{j=1}^{\textcolor{red}{l}} \mu_j \nabla g_j(x_0) = 0$$

\subsection{2nd order necessary conditions for local minimum}
Open $\Omega \subseteq \real^n, f, h_1, \hdots, h_k, g_1, \hdots, g_l \in C^2$.
Let $x_0$ be a regular point of the constraints:
$$(\dagger)\begin{cases}
	h_1(x) = \hdots = h_k(x_0) = 0 \\
	g_1(x), \hdots g_l(x_0) \leq 0
\end{cases}$$
Suppose $x_0$ is a local min of $f$ subject to ($\dagger$).
Then, $\exists \lambda_i \in \real, \mu_j \geq 0$ s.t.
\begin{enumerate}
	\item $\nabla f(x_0) + \sum_{i=1}^k \lambda_i \nabla h_i(x_0) + \sum_{j=1}^l \mu_j \nabla g_j(x_0) = 0$
	\item $\mu_j g_j(x_0) = 0 $ for all $j$
	\item \textcolor{red}{$[\nabla^2 f(x_0) + \sum \lambda_i \nabla^2h_i(x_0) + \sum \mu_j \nabla^2 g_h(x_0)]]$ is positive semi definite on tangent space to active constraints at $x_0$.}
\end{enumerate}
\begin{proof}
	$x_0$ local min for ($\dag$) \\
	$\implies x_0$ local min for only active constraints at $x_0$. \\
	$\implies \begin{cases}
		h_i(x) = 0 \,\, \forall i \\
		g_j(x) = 0 \,\, j=1, \hdots, l'
	\end{cases}$ \\
	$\implies [\nabla^2 f(x_0) + \sum \lambda_i \nabla^2h_i(x_0) + \sum \mu_j \nabla^2 g_h(x_0)]$ pos semi def on tangent space to active constraints. 
\end{proof}

\subsection{2nd order sufficient conditions}
Open $\Omega \subseteq \real^n, f, h_i, g_j \in C^2$ on $\Omega$. \\
\under{Problem:}
$$\begin{cases}
	\min &f(x) \\
	\text{subject to }&\begin{cases}h_i(x) = 0 \\
	g_j(x) \leq 0 \end{cases}
\end{cases}$$
Suppose $\exists x_0$ feasible and $\lambda_i, \mu_j \in \real$ s.t. 
\begin{enumerate}
	\item $\nabla f(x_0) + \sum_{i=1}^k \lambda_i \nabla h_i(x) + \sum_{j=1}^l \mu_j \nabla g_j(x_0) = 0$
	\item $\mu_j g_j(x_0) =0$ all $j$
\end{enumerate}
If the Hessian matrix, $L(x_0) = \nabla^2 f(x_0) + \sum_{i=1}^k \lambda_i \nabla^2 f(x_0) + \sum_{i=1}^l \mu_j \nabla^2 g_j(x)$ is pos def on $\tilde{\tilde{T}}_{x_0}$-space of ``strongly active" constraints at $x_0$. \\
Then $x_0$ is a strict local min. \\

\paragraph{Remarks}
\begin{enumerate}
	\item $$\text{Active constraints at $x_0$} \begin{cases}
	h_i(x) = 0 \quad i=1,\hdots,k \\
	g_j(x) \leq 0 \quad j = 1,\hdots, l' \implies g_j(x_0) = 0
	\end{cases}$$

	\item $$\text{\textcolor{red}{Strongly active constraints at $x_0$}} 
	\begin{cases}
	h_i(x) = 0 \quad i=1,\hdots,k \\
	g_j(x) \leq 0 \quad j = 1,\hdots, l'' \,\,\, \text{$g_j(x)$ is active at $x_0 \text{ and } \mu_j > 0$}
	\end{cases}$$
	$l'' \leq l' \leq l$
	\item $$\tilde{\tilde{T}}_{x_0} = \{ v\in \real^n | v\cdot \nabla h_i(x_0) = 0 \text{ all $i$} \text{ and } v \cdot \nabla g_j(x_0) = 0 \text{ for all $j = 1, \hdots, l''$} \}$$
	\item strongly active $\subseteq$ active \\
	      $\implies \tilde{\tilde{T}}_{x_0}$ = (strongly active)$^\perp \supseteq$ (active)$^\perp$ = $\tilde{T}_{x_0}$
\end{enumerate}
\begin{proof}
	(details see another pdf by prof)
	Suppose $x_0$ is \textcolor{red}{NOT} a (strict) local min.\\
	\under{claim:} $\exists$ unit vector $v \in \real$ s.t.
	\begin{enumerate}
		\item $\nabla f(x_0) \cdot v \leq 0$
		\item $\nabla h_i(x_0) \cdot v = 0 \quad i = 1, \hdots, k$
		\item $\nabla g_j(x_0) \cdot v \leq 0 \quad j = 1, \hdots, l'$
	\end{enumerate}
	\under{proof of claim}: [] \\
	\under{claim:} $\nabla g_j(x)\cdot v = 0$ for $j = 1, \hdots, l''$ \\
	\under{proof of claim:} [] \\
	$\implies$ contradiction! \\
	\under{claim:} $\exists$ unit vector $v \in \real$ s.t.
	\begin{enumerate}
		\item $\nabla f(x_0) \cdot v \leq 0$
		\item $\nabla h_i(x_0) \cdot v = 0 \quad i = 1, \hdots, k$
		\item $\nabla g_j(x_0) \cdot v = 0 \quad j = 1, \hdots, l''$
	\end{enumerate}
	\under{proof of claim:} [] 
\end{proof}


\section{Different Computation Methods for Solving Optimum}
\subsection{Newton's Method}
$x_0 \in I$ start \\
$x_{n+1} = x_0 - \frac{f'(x_0)}{f''(x_0)}$

\paragraph{Theorem}
Let $f \in C^3$ on $I$.\\
Suppose $x_* \in I$ satisfies $f'(x_*) = 0$ and $f''(x_*) \neq 0$ (\textcolor{red}{$x_*$ is a non-degenerate (non-singular) critical point}).\\
Then the sequence of points $\{x_n\}$ generated by Newton's method 
$$x_{n+1} = x_n  - \frac{f'(x_0)}{f''(x_0)}$$ converges to $x_*$ if $x_0$ is sufficiently close to $x_*$.

\paragraph{Why do we need this method?}
In real life, we may not know the real function formula. We only have data, using which we can approximate the function formula. In a way, Newton's method is true ``applied mathematics".

\paragraph{Proof of Theorem}
Let $g(x) = f'(x)$ so that $x_{n+1} = x_n - \frac{g(x_n)}{g'(x_n)}$ \\
By $g \in C^2, \exists \alpha$ s.t. $|g'(x_1)| > \alpha \, \forall x_1$ and $|g''(x_2)| < \frac{1}{\alpha} \,\, \forall x_2$ in a neighbourhood of $x_*$ (choose $\alpha$ small enough).

\begin{align}
	x_{n+1} - x_* &= x_n - \frac{g(x_n)}{g'(x_n)} - x_* \\
	&= x_n - x_* - \frac{g(x_n) - g(x_*)}{g'(x_n)} &\text{($g(x_*) = 0$)}\\
	&= \frac{-[g(x_n) - g(x_*) - g'(x_n)(x_n - x_*)]}{g'(x_n)} \\
	&= \frac{1}{2}\frac{g''(\xi)}{g'(x_n)}(x_n - x_*)^2 \\
	|x_{n+1} - x_*| &= \frac{1}{2}\frac{g''(\xi)}{g'(x_n)}|x_n - x_*|^2 < \frac{1}{2\alpha^2}|x_n - x_*|^2 &\text{(in small neighbourhood of $x_*$)} \\
	\rho &:= \frac{1}{2\alpha^2} |x_0 - x_*| &\text{(choose $x_0$ sufficiently close to $x_*$ s.t. $\rho < 1$)} \\
	|x_1 - x_*| &< \frac{1}{2\alpha^2}|x_0 - x_*|^2 \\
	&= \frac{1}{2\alpha^2}|x_0 - x_*||x_0 - x_*| \\
	&= \rho|x_0 - x_*| \\
	|x_2 - x_*| &< \frac{1}{2\alpha^2}|x_1 - x_*|^2 \\
	&< \frac{1}{2\alpha^2}\rho^2|x_0 - x_*|^2 \\
	&= \frac{1}{2\alpha^2}|x_0 - x_*| \rho^2 |x_0 - x_*| \\
	&< \rho^2 |x_0 - x_*| &\text{($\rho < 1$)} \\
	|x_n - x_*| &< \rho^n |x_0 - x_*| \underset{n \rightarrow \infty}{\rightarrow} 0 \\
	\implies x_n &\rightarrow x_*
\end{align}
\under{proof of (4)}: \\
By 2nd order MVT,
$$g(x) = g(y) + g'(y)(x-y) + \frac{1}{2}g''(\xi) (x-y)^2$$
for some $\xi \in [x, y]$. \\
Let $x = x_*$ and $y = x_n$, then
$$g(x_*) = g(x_n) + g'(x_n)(x_* - x_n) + \frac{1}{2}g''(\xi)(x_* - x_n)^2$$
$$\implies -[g(x_n) - g(x_*) - g'(x_n)(x_n - x_*)] = \frac{1}{2}g''(\xi)(x_n - x_*)^2 $$
\qed
\paragraph{Newton's Method (generalized)}
$f: \Omega \underset{open}{\subseteq} \real^n \rightarrow \real$ and $f \in C^3$ on $\Omega$ \\
$x_0 \in \Omega$ \\
$x_{n+1} = x_n - [\nabla^2 f(x_n)]^{-1} \nabla f(x_n)$ \\
(The algorithm requires $\nabla^2 f(x_n)$ invertible and stops when $\nabla f(x_n) = 0$)
\paragraph{Note}
Newton's method may fail to converge even if $f(x)$ has a unique global min $x_*$ and $x_0$ is arbitrarily close to $x_*$
\paragraph{Remark}
Newton's method, if converge, converges to
\begin{enumerate}
	\item local min
	\item local max
	\item saddle point
\end{enumerate}

\example{Newton's Method on Quadratic Function}
Let $Q$ be a symmetric $n \times n$ invertible matrix.
Define quadratic form $f(x) := \frac{1}{2}x^TQx: \real^n \rightarrow \real$. Then the optima is $x = 0$. \\
Let $x_0 \in \real^n$, then $$x_1 := x_0 - \nabla^2 f(x_0)^{-1} \nabla f(x_0) = x_0 - Q^{-1}Qx_0 = 0$$
\red{Newton's method converges in one iteration.}

\subsection{Method of Steepest Descent (Gradient Method)}
$f: \Omega \underset{open}{\subseteq} \real^n \rightarrow \real,C^1$ \\
Recall: Direction of steepest ascent at $x_0$ is given by the direction of gradient $\nabla f(x_0)$
\paragraph{Algorithm of steepest descent}
$x_0 \in \Omega$ \\
$$x_{k+1} = x_k - \alpha_k \nabla f(x_k)$$
where $\alpha_k \geq 0$ satisfying $f(x_k - \alpha_k \nabla f(x_k)) = \underset{\alpha \geq 0}{\min} f(x_k - \alpha \nabla f(x_k))$ \\
(keep going until you find the minimum) \\
\paragraph{Fact: algorithm is descending}
If $\nabla f(x_k) \neq 0$, then $f(x_{k+1}) < f(x_k)$\\
Why? $f(x_{k+1}) = f(x_k - \alpha_k \nabla f(x_k)) \leq f(x_k - \alpha \nabla f(x_k))$ for all 
$ 0 < \alpha \leq \alpha_k$ \\
\under{Recall:} $\frac{d}{ds}|_{s=0} f(x_k - s\nabla f(x_k)) = \nabla f(x_k) \cdot (-\nabla f(x_k)) = - |\nabla f(x_k)|^2 < 0$ \\
$\implies f(x_{k+1}) \leq f(x_k - \alpha \nabla f(x_k)) < f(x_k)$ for small $\alpha$

\paragraph{Fact: the method of steepest descent moves perpendicular steps}
\begin{align}
	(x_{k+2} - x_{k+1}) \cdot (x_{k+1} - x_k) &= (-\alpha_{k+1}\nabla f(x_{k+1}))\cdot(-\alpha_k \nabla f(x_k)) \\
	&= \alpha_k \alpha_{k+1} \nabla f(x_{k+1})\cdot \nabla f(x_k) \\
\end{align}
If $\alpha_k = 0$, then we are done. \\
If $\alpha_k \neq 0$, then 
\begin{align}
	\nabla f(x_{k+1}) &= \underset{\alpha \geq 0}{\min}\, f(x_k - \alpha \nabla f(x_k)) \\
\implies \frac{d}{d\alpha}|_{\alpha = \alpha_k} f(x_k - \alpha \nabla f(x_k)) &= (- \nabla f(x_k)) \cdot \nabla f(x_k - \alpha_k \nabla f(x_k)) = 0 \\
\implies \alpha_k \alpha_{k+1} \nabla f(x_{k+1})\cdot \nabla f(x_k) &= 0
\end{align}

\paragraph{Note}
This method is not the most efficient. May take infinite steps to converge.

\paragraph{Theorem (Convergence of Steepest Descent)}
$f \in C^1$ on $\Omega \underset{open}{\subseteq} \real^n$ \\
Let $\{ x_k\}$ be sequence generated by steepest descent. \\
$x_{k+1} = x_k - \alpha_k \nabla f(x_k)$ \\
If $\{ x_k\}$ is ``bounded in $\Omega$" (i.e. $\exists$ compact set $K \subset \Omega$ s.t. $x_k \in K$ for all $k$)\\
Then every convergent subsequence of $\{ x_k\}$ converges to a critical point $x_* \in \Omega$ of $f: \nabla f(x_*) = 0$
\begin{proof}
	$x_k \in$ compact $K \implies$ subsequence $x_{k_i} \rightarrow x_* \in K$ \\
	Since $f(x_0) \geq f(x_1) \geq f(x_2) \geq \hdots$ and $f(x_{k_i}) \searrow f(x_*)$\\
	Suppose by contradiction that $\nabla f(x_*) \neq 0$ \\
	$x_{k_i} \rightarrow x_* \implies  \nabla f(x_{k_i}) \rightarrow \nabla f(x_*)$\\
	Let $y_{k_i} = x_{k_i} - \alpha_{k_i} \nabla f(x_{k_i}) = x_{k_i + 1}$. Then $y_{k_i} \rightarrow y_*$. Then
	\begin{align}
		f(y_{k_i}) &= f(x_{k_i + 1} ) = \underset{\alpha \geq 0}{\min} f(x_i - \alpha \nabla f(x_{k_i}))  \\
		f(y_{k_i}) &\leq f(x_{k_i} - \alpha \nabla f(x_{k_i})) \text{ for all $\alpha \geq 0$}\\
		\underset{i \rightarrow \infty}{\lim} f(y_*) &\leq f(x_* - \alpha \nabla f(x_*)) \text{ for all $\alpha \geq 0$} \\
		f(y_*) &\leq  \underset{\alpha \geq 0}{\min} f(x_* - \alpha \nabla f(x_*)) < f(x_*) \\
		f(y_*) &< f(x_*)\\
	\end{align}
	But $f(y_*) = \underset{i \rightarrow \infty}{\lim}f(y_{k_i}) = \underset{i \rightarrow \infty}{\lim}f(x_{k_i + 1}) = f(x_*)$, so we have a contradiction. 
\end{proof}

\paragraph{Steepest descent: Quadratic case}
Let $f$ follow the general quadratic form
$$f(x) = \frac{1}{2} x^T Q x - b^T x$$
where $b, x \in \real^n$ and  $Q$ is an $n \times n$ positive definite matrix. \\
Let $0 < \lambda = \lambda_1 \leq \lambda_2 \leq \hdots \leq \lambda_n = \Lambda$ be eigenvalues of Q. \\
Recall that if Q pos-def, then there is a unique minimum $x_*$ such that $Qx_* - b = 0 \iff  x_* = Q^{-1}b$\\
Define
$q(x) := \frac{1}{2}(x - x^*)^TQ(x - x^*) = f(x) + const$ \\
Note that  $q(x) \geq 0$ and $q(x_*) = 0$. \\
Define $g(x) := Qx - b  = \nabla q(x) = \nabla f(x)$\\
So using the method of steepest descent: 
$$x_{k+1} = x_k - \alpha_k g(x_k)$$
\tb{Derive the formula for $\alpha_k$:} \\
$\alpha_k$ minimizes $f(x_k - \alpha g(x_k))$
\begin{align}
	0 &= \frac{d}{d\alpha}|_{\alpha = \alpha_k} f(x_k - \alpha g(x_k)) \\
	&= \nabla f(x_k - \alpha_k g(x_k)) \cdot (-g(x_k)) \\
	&= - [Q (x_k - \alpha_k g(x_k)) - b] \cdot (g(x_k)) \\
	&= -(Q x_k - b) \cdot g(x_k) \\
	&= - |g(x_k)|^2 + \alpha_k g(x_k)^T Q g(x_k) \\
	\implies \alpha_k &= \frac{|g(x_k)|^2}{g(x_k)^TQg(x_k)}\\
	\implies x_{k+1} &= x_k - \alpha_k g(x_k) \\
	&= x_k - \frac{|g(x_k)|^2}{g(x_k)^T Q g(x_k)} g(x_k)
\end{align}

\tb{Claim:} \\
$$ q(x_{k+1}) = \left( 1 - \frac{|g(x_k)|^4}{(g(x_k)^TQ g(x_k))(g(x_k)^TQ^{-1}g(x_k))}\right)g(x_k)$$
\begin{proof}
\begin{align}
	q(x_{k+1}) &= q(x_k - \alpha_k g(x_k)) \\
	&= \frac{1}{2} (x_k - \alpha_k g(x_k) - x_*)^T Q (x_k - \alpha_k g(x_k) - x_*) \\
	&= \frac{1}{2} (x_k - x_* - \alpha_k g(x_k))^T Q ((x_k - x_*) - \alpha_k g(x_k)) \\
	&= \frac{1}{2} (x_k - x_*)^T Q (x_k - x_*) - \alpha_k g(x_k)^T Q (x_k - x_*) + \frac{1}{2} \alpha_k^2 g(x_k)^T Q g(x_k) \\
	&= q(x_k) - \alpha_k g(x_k)^T Q (x_k - x_*) + \frac{1}{2} \alpha_k^2 g(x_k)^T Q g(x_k) \\
	\implies q(x_k) - q(x_{k+1}) &= -\frac{1}{2} \alpha_k^2 g(x_k)^T Q g(x_k) + \alpha_k g(x_k)^T Q (x_k - x_*) \\
	y_k := x_k - x_* \\
	\frac{q(x_k) - q(x_{k+1})}{q(x_k)} &= \frac{-\frac{1}{2}\alpha_k^2 g(x_k)^T Q g(x_k) + \alpha_k g(x_k)^T Q y_k}{\frac{1}{2}y_k^T Q y_k}  \\
	&= \frac{2\alpha_k g(x_k)^T Q y_k - \alpha_k^2 g(x_k)^T Q g(x_k)}{y_k^T Q y_k} \\
	(&g_k := g(x_k) = Q x_k - b = Qx_k - Q x_* = Q (x_k - x_*) = Q y_k \implies y_k = Q^{-1}g_k) \\
	&= \frac{2\alpha_k|g_k|^2 - \alpha_k^2 g_k^T Q g_k}{g_k^TQ^{-1}g_k}\\ 
	&= \frac{2\frac{|g_k|^4}{g_k^TQg_k}  - \frac{|g_k|^4}{g_k^TQg_k}}{g_k^TQ^{-1}g_k}\\
	&= \frac{|g_k|^4}{(g_k^TQ g_k)(g_k^TQ^{-1}g_k)} \tag{$\alpha_k = \frac{|g(x_k)|^2}{g(x_k)^T Q g(x_k)}$} \\
	\implies q(x_k) - q(x_{k+1}) &= \left( \frac{|g_k|^4}{(g_k^TQ g_k)(g_k^TQ^{-1}g_k)} \right ) q(x_k) \\
	\implies q(x_{k+1}) &= q(x_k)\left( 1 - \frac{|g_k|^4}{(g_k^TQ g_k)(g_k^TQ^{-1}g_k)} \right) \\
	&\leq (1 - \frac{4 \lambda \Lambda}{(\lambda + \Lambda)^2}) q(x_k) \tag{By Kantorovich Inequality} \\
	\implies q(x_{k+1}) &\leq \frac{(\Lambda - \lambda)^2}{(\lambda + \Lambda)^2} q(x_k)
\end{align}

\end{proof}

\paragraph{Kantorovich Inequality}
$Q: n\times n$ positive definite symmetric matrix \\
$\lambda = \lambda_1 \leq \lambda_2 \leq \hdots \leq \lambda_n = \Lambda$ \\
For any $v \in \real^n$:
$$ \frac{|v|^4}{(v^TQv)(v^TQ^{-1}v)} \geq \frac{4\lambda \Lambda}{(\lambda + \Lambda)^2} $$


\paragraph{Theorem: Steepest Descent in Quadratic Case}
For any $x_0 \in \real^n$, method of steepest descent converges to the unique min point $x_*$ of $f$. \\
Furthermore, for $q(x) := \frac{1}{2} (x - x_*) Q (x - x_*)$, where Q symmetric positive definite and $0 < \lambda = \lambda_1 \leq \lambda_2 \leq \hdots \leq \lambda_n = \Lambda$,  $$q(x_{k+1}) \leq \frac{(\Lambda - \lambda)^2}{(\lambda + \Lambda)^2} q(x_k)$$
Let $r := \frac{(\Lambda - \lambda)^2}{(\lambda + \Lambda)^2}$, then
$$q(x_k) \leq r^k q(x_0)$$ for all k. As $k \rightarrow \infty$, $q(x_k) \rightarrow 0$.

\paragraph{Notes}
\begin{enumerate}
	\item $x_k \in \{ x \in \real^n| q(x) \leq r^k q(x_0)\} = SL_k$\\ (sublevel set of function $q(x)$)\\
	Note that $SL_k$ is strictly decreasing. Furthermore, note that $x_*$ is the only point satisfying the inequality at the limit:
	$$q(x_*) = 0 = \underset{k \rightarrow \infty}{\lim} q(x_0)$$
	Therefore, $\underset{k \rightarrow \infty}{\lim} SL_k = \{0\}$, and $x_k \rightarrow x_*$. 
	\item $r = (\frac{(\Lambda - \lambda)}{(\lambda + \Lambda)})^2 = (\frac{\Lambda / \lambda - 1}{\Lambda / \lambda + 1})^2$ depends only on the ratio $\frac{\Lambda}{\lambda}$ = ``condition number of Q" \\
	\tb{case $\frac{\Lambda}{\lambda} = 1$}$\implies r = 0 \implies 0 \leq q(x_1) \leq 0\cdot q(x_0) \implies q(x_1) = 0 \implies x_1 = x_*$\\
	(\red{Gradient descent converges to the unique global minimum in only one iteration.})\\
	\tb{case $\frac{\Lambda}{\lambda} >> 1$} $\implies r \simeq 1$ \\(\textcolor{red}{worst case, converges very flow})
\end{enumerate}

\subsection{Method of Conjugate Direction}
\paragraph{Motivation}
Method of conjugate directions is designed for quadratic functions with form $f(x) = \frac{1}{2}x^TQx - b^Tx$. For other functional forms, one can approximate the function using quadratic form firstly and then apply method of conjugate directions.

\paragraph{Definition: Q-orthogonality}
Let Q be a symmetric matrix. Two vectors $d, d' \in \real^n$ are \under{Q-orthogonal} (or \under{Q-conjugate}) if $$d^TQd' = 0$$
A finite set of $d_0, \hdots, d_k$ is called \under{Q-orthogonal set} if $d_i^TQd_j = 0$ for all $i \neq j$.

\paragraph{Example 1}
Q is an identity matrix. $d, d'$ are Q-orthogonal iff they are orthogonal.

\paragraph{Example 2}
If $d, d'$ are two eigenvectors with different eigenvalues, then they are Q-orthogonal. \\
\begin{proof}
Suppose $Qv = \lambda v$ and $Qw = \lambda' w$ so $\lambda \neq \lambda'$
	\begin{align}
		<v, Qw> &= <v, \lambda'w> = \lambda'<v,w> \\
		&= <Q^Tv, w> = <Qv, w> = <\lambda v, w> = \lambda <v,w>\\
		&\implies (\lambda - \lambda')\langle v, w \rangle  = 0
	\end{align}
	Since $(\lambda - \lambda') \neq 0$, them we have $\langle v, w \rangle  = 0$.
	$$\implies v^TQw = \langle v, Qw \rangle = \lambda \langle v, w \rangle  = 0$$
\end{proof}

\paragraph{Example 3}
If $Q$ is an $n \times n$ symmetric matrix, then there exists an orthogonal basis of eigenvectors $d_0, \hdots, d_{n-1}$ \\
\tb{Claim: } They are also Q-orthogonal. \\
\begin{proof}
	$d_i^TQd_j = d_i^T(\lambda d_j) = \lambda d_i^Td_j = 0$ 
\end{proof}

\paragraph{Proposition}
Let $Q$ be a symmetric positive definite matrix. Let $d_0, \hdots, d_k$ be a set of (non-zero) Q-orthogonal vectors. Then \textcolor{red}{$d_0, \hdots, d_k$ are linearly independent.} \\
\begin{proof}
	Assume $\alpha_0 d_0 + \hdots + \alpha_kd_k = 0$ for $\alpha_i \in \real$. \\
	Multiply the whole equation by $d_i^T Q$:
	$$\alpha_0 \underbrace{d_i^T Qd_0}_{=0} + \hdots + \alpha_i \underbrace{d_i^TQd_i}_{>0} + \hdots + \alpha_k \underbrace{d_i^T Qd_k}_{=0} = 0$$
	which implies $\alpha_i d_i^TQd_i = 0$ and $\alpha_i$ = 0. \\
	This is true for every $i$. Therefore $d_0, \hdots, d_k$ are linearly independent.
\end{proof}

\paragraph{Lemma (Theorems covered so far)}
\begin{enumerate}
	\item $d_i, d_j$ are $Q$-orthogonal if $d_i^TQd_j = 0$;
	\item Eigen-vectors with different eigenvalues are $Q$-orthogonal;
	\item Matrix $Q$ symmetric $\implies$ there exists an orthogonal basis $\implies$ the set of basis is $Q$-orthogonal as well;
	\item $Q$-orthogonal vectors are linearly independent.
\end{enumerate}

\paragraph{Example 4} (Special case: Method of Conjugate Direction on Quadratic Functions). Let $Q$ be a positive definite symmetric $n \times n$ matrix. The problem is
$$\min f(x) = \frac{1}{2}x^TQx - b^Tx$$
Recall that the unique global minimum is $x^* = Q^{-1}b$. \\
Let $d_0, d_1, \hdots, d_{n-1}$ be non-zero Q-orthogonal vectors. \\
Note that they are linearly independent by the previous theorem. \\
Therefore, they form a basis of $\real^n$. \\
The global minimum can be represented as 
$$x^* = \sum_{j=0}^{n-1} \alpha_j d_j, \, \alpha_j \in \real$$
For every $j$, the following holds:
$$d_j^TQx^* = \alpha_jd_j^TQd_j$$
$$\implies \alpha_j = \frac{d_j^TQx^*}{d_j^TQd_j}$$
\paragraph{Algorithm: Method of Conjugate Directions}
Let $Q$ be a positive definite symmetric $n \times n$ matrix. and $\{d_j\}_{j=0}^{n-1}$ be a set of non-zero $Q$-orthogonal vectors, note that they form a basis of $\real^n$. \\
Given initial point $x_0 \in \real^n$, the method of conjugate direction generates a sequence of points $\{x_k\}_{k=0}^n$ as the following:
$$x_{k+1} \leftarrow x_k + \alpha_k d_k$$
$$\alpha_k := -\frac{\langle g_k, d_k\rangle}{d_k^TQd_k} \,\, g_k := \nabla f(x_k)$$
\paragraph{Theorem} Given the method of conjugate, the sequence of points generated eventually reaches the global minimum. That is, $x_{n} = x^*$. \\
\begin{proof}
	Let $x^*, x_0 \in \real^n$, consider
	\begin{align}
		x^* - x_0 &= \sum_{j=0}^{n-1}\beta_j d_j \\
		\iff x^* &= x_0 + \sum_{j=0}^{n-1} \beta_j d_j \\
		d_j^TQ(x^* - x_0 &= d_j^TQ(\sum_{j=0}^{n-1} \beta_j d_j) \\
		&= \beta_j d_j^T Q d_j \\
		\implies \beta_j &= \frac{d_j^TQ(x^* - x_0)}{d_j^TQd_j}
	\end{align}
Note that the algorithm generates the sequence as following:
\begin{align}
	x_k &= x_0 + \sum_{j=0}^{k-1} \alpha_j d_j \\
	\implies (x_k - x_0) &= \sum_{j=0}^{k-1} \alpha_j d_j \\
	\implies d_k^TQ(x_k - x_0) &= \sum_{j=0}^{k-1} \alpha_j d_k^T Q d_j = 0
\end{align}
Therefore,
\begin{align}
	\beta_k &= \frac{d_k^TQ(x^* - x_0)}{d_k^TQd_k} \\
	&= \frac{d_k^TQ(x^* - x_0) - d_k^TQ(x_k - x_0)}{d_k^TQd_k} \\
	&= \frac{d_k^TQ(x^* - x_k)}{d_k^TQd_k} \\
	&= \frac{d_k^T(Qx^* - Qx_k)}{d_k^TQd_k} \\
	&= \frac{d_k^T(b - Qx_k)}{d_k^TQd_k} \tag{The first order necessary condition suggests $Qx^* = b$} \\
	&= -\frac{d_k^T(Qx_k - b)}{d_k^TQd_k} \\
	&= -\frac{d_k^T\nabla f(x_k)}{d_k^TQd_k} \tag{Assuming $f$ is quadratic} \\
	&= \alpha_k
\end{align}
Consequently,
\begin{align}
	x^* &= x_0 + \sum_{j=0}^{n-1}\beta_j d_j \\
	&= x_0 + \sum_{j=0}^{n-1} \alpha_j d_j \\
	&= x_n
\end{align}
\qed

\subsubsection{Geometric Interpretations of Method of Conjugate Directions}
\paragraph{Theorem} Let $f \in C^1 (\Omega, \real)$, where $\Omega$ is a convex subset of $\real^n$, then $x_0$ is a local minimum of $f$ on $\Omega$ if and only if 
$$\nabla f(x_0) \cdot (y - x_0) \geq 0 \, \forall y\in \Omega$$
\paragraph{Example} Now consider the special case in which $\Omega$ is an affine hyperplane, that is, 
$$\Omega = \{ x \in \real^n: cx + b = 0\}$$
where $\dim(\Omega)$ is $n-1$. \\
Note that for every $y \in \Omega, \nabla f(x_0)\cdot (y - x_0) \geq 0$. For any feasible direction $a:= y - x_0$ at point $x_0$, by definition of hyperplane, $-a$ is a feasible direction as well. \\
Consequently, $a \cdot \nabla f(x_0) = 0$ for every feasible direction. That is, $\nabla f(x_0) \perp \Omega$.
\end{proof}

\paragraph{Geometric Interpretation}
Let $d_0, d_1, \hdots, d_{n-1}$ be a set of non-zero $Q$-orthogonal vectors in $\real^n$. Let $B_k = Span \{d_0, \hdots, d_{k-1}\}$ for $k = 0, 1, \hdots, n$. \\
\under{Note:}
\begin{itemize}
	\item $$B_0 = \{0\} \subseteq B_1 = \langle d_0 \rangle \subseteq B_2 = \langle d_0, d_1 \rangle \subseteq \hdots \subseteq B_n = \langle d_0, \hdots, d_{n-1} \rangle  = \real^n$$
	\item $$\dim B_k = k$$ 
	\item $$x_0 + B_0 \subseteq x_0 + B_1 \subseteq \hdots$$
\end{itemize}

\paragraph{Theorem}
The sequence $\{x_k\}$ generated from $x_0 \in \real^n$ by conjugate directions method has the property that $x_k$ minimizes $f(x) = \frac{1}{2}x^TQx - b^Tx$ on the affine hyperplane $x_0 + B_k$. \\
\begin{proof}
	Recall that $x_k$ is the minimizer of $f(x)$ on $x_0 + B_k \iff \nabla f(x_k) \perp x_0 + B_k$ \\
	\textcolor{red}{Enough to prove that $\nabla f(x_k) \perp B_k$.} \\\\
	\ti{Remarks: $x_0$ here is like a bias which shifts the subspace by a ``constant". Also, ``$\nabla f(x_k) \perp B_k$" here means that $\nabla f(x_k)$ is perpendicular to every basis vector of $B_k$.} \\\\
	We prove this by induction on $k$. \\
	\under{Notation:} $\nabla f(x_k) = Q x_k - b =: g_k$. \\
	\tb{Base case: k = 0} $B_0 = \{0\} \implies g_0 \perp B_0$ \\
	\tb{Inductive Step: } Assume that $g_k \perp B_k$, show $g_{k+1} \perp B_{k+1}$ \\
	Since $$x_{k+1} = x_k + \alpha_k d_k$$
	then $$\underbrace{Q_{x_{k+1}} - b}_{g_{k+1}} = \underbrace{Q_{x_k} - b}_{g_k} + \alpha_k Q d_k$$
\begin{align}
	g_{k+1}^TB_k &= \langle d_0, \hdots, d_{k-1} \rangle \\
	g_{k+1}^Td_k &= (\underbrace{g_k + \alpha_k Q d_k^T d_k}_{g_{k+1}})^Td_k \\
	&= g_k^Td_k + \alpha_k d_k^T Q d_k \\
	&= g_k^Td_k + (- \frac{g_k^Td_k}{d_k^TQd_k})d_k^TQd_k \\
	&= 0
\end{align}
This implies that $g_{k+1} \perp d_k$ \\
For $0 \leq i < k$, 
\begin{align}
g_{k+1}^T\cdot d_i &= (g_k + \alpha_k Q d_k)^T d_i \\
&= \underbrace{g_k^Td_i}_{= 0} + \underbrace{\alpha_k d_k^T Q d_i}_{= 0} \\
&= 0	
\end{align}
Therefore, $g_{k+1} \perp d_0, d_1, \hdots, d_k$\\
Hence $g_{k+1} \perp \langle d_0, d_1, \hdots, d_k \rangle = B_k$
\qed 

\paragraph{Corollary} $x_n$ minimizes $f(x)$ on $x_0 + B_n$ (which is $\real^n$) \\
i.e. $x_n = x^*$ 
\paragraph{Corollary} $0 \leq q(x_{k}) = \underset{x \in x_0 + B_k}{\min} q(x) \leq q(x_{k-1}) = \underset{x \in x_0 + B_{k-1}}{\min} q(x)$
\paragraph{Corollary}
\begin{align}
&\begin{cases}
	\min f(x) \\
	x \in x_0 + B_1
\end{cases} \\
\implies &\begin{cases}
	\min f(x_0 + td_0) \\
	t \in \real
\end{cases}\tag{Since $x_0 + B_1 = \{x_0 + td_0 | t \in \real \}$} \\
\implies &0 = \frac{d}{dt}|_{t=t_0} f(x_0 + td_0) = \nabla f(x_0 + t_0d_0) \cdot d_0 \tag{where $t_0$ is such that $x_1 = x_0 + t_0 d_0$}
\end{align}
\end{proof}

\section{Calculus of Variations}
Note: infinite dimensional optimization.
\paragraph{Comparison with finite dimensions}
\begin{center}
\begin{tabular}{ |c|c|c| } 
 \hline
 & finite dimensional & $\infty$-dimensional \\
 \hline
 problem & $\min \, f(x)$ & $\min \, F[u]$ \\
 \hline
 constraint & $x \in M$ & $u \in \mathcal{A}$ \\ 
 \hline
 note & set of points in $\real^n$ & space of functions \\
 \hline
\end{tabular}
\end{center}
\paragraph{Model model}
$$\mc{A} =  \{ u: [0,1] \rightarrow \real | u \in C^1 \, s.t. \, u(0) = u(1) = 1\}$$
\under{Note:} We call $F$ a ``Functional". It maps a function to a real number.
\paragraph{Notation}
Write $u(\cdot)$ for a function $u$.
\subsection{Example}
$F[u(\cdot)] = \frac{1}{2} \int_0^1 \{ u(x)^2 + u'(x)^2\} \, dx$.\\
$ \begin{cases}
	\min \, F[u(\cdot)] \\
	u(\cdot) \in \mc{A}
\end{cases}$\\
means: Find $u^*(\cdot) \in \mc{A}$ s.t. $F[u^*(\cdot)] \leq F[u(\cdot)]$ for all $u(\cdot) \in \mc{A}$.
\paragraph{Plan}
\begin{enumerate}
	\item We derive 1st order necessary conditions for a local min;
	\item Find a function $u^*(\cdot)$ satisfying these conditions;
	\item Check this candidate $u^*(\cdot)$ is in fact a minimizer.
\end{enumerate}
We reduce this problem to (many) 1-dimensional problems. \\
\tb{Step 1:} Derive 1st order necessary conditions for a local min \\
Fix $v(\cdot) \in C^1$ on $[0,1]$ s.t. $v(0) = 0 = v(1)$. \\
Suppose $u^*(\cdot) \in \mc{A}$ is a minimizer. \\
Notice that $u^*(\cdot) + sv(\cdot) \in \mc{A}\, \forall s \in \real$. \\
Let $f: \real \rightarrow \real$ s.t. $f(s) := F(u^*(\cdot) + sv(\cdot))$. \\
If $u^*(\cdot)$ minimizes $F$, then $s=0$ minimizes $f$, then $f'(0) = 0$. \\
Then $f(0) = F[u^*(\cdot)] \leq F[u^*(\cdot) + sv(\cdot)] = f(s)$ \\
\begin{align}
	f'(0) &= \frac{d}{ds}|_{s=0} \underbrace{F[u^*(\cdot) + sv(\cdot)]}_{f(s)} \\
	&= \frac{d}{ds}|_{s=0} \frac{1}{2} \int_0^1 \{ [u^*(x) + sv(x)]^2 + [u^{*'}(x) + sv'(x)]^2\} \, dx \\
	&= \frac{1}{2}\frac{d}{ds}|_{s=0} \int_0^1 \{ u^*(x)^2 + u^{*'}(x)^2\} \, dx + \frac{d}{ds}|_{s=0} s \int_0^1 \{u^*(x)v(x) + u^{*'}(x)v'(x)\} \, dx + \frac{d}{ds}|_{s=0} \frac{s^2}{2} \int_0^1 \{v(x)^2 + v'(x)^2\} \, dx \\
	&= \int_0^1 \{u^*(x)v(x) + u^{*'}(x)v'(x)\} \, dx
\end{align}
So far, if $u^*(\cdot)$ is a minimizer of $F$ over $\mc{A}$, then
$$\int_0^1 \{u^*(x)v(x) + u'^{*}(x)v'(x)\} \, dx = 0 \quad (\heartsuit)$$
for all $v(\cdot) \in C^1$ on $[0,1]$ and $v(0) = 0 = v(1)$.
We call this a ``primitive form of 1st order condition", and call $v(\cdot)$ the test functions. \\
Recall Integration by parts:
$$\int_0^1 w(x)v'(x) \, dx = w(x)v(x)|_0^1 - \int_0^1 w'(x)v(x) \, dx$$
\begin{align}
	(\heartsuit) &= \int_0^1 u^*(x) v(x) \, dx + \int_0^1 u'^{*}(x) v'(x) \, dx \\
	&=  \int_0^1 u^*(x) v(x) \, dx + \underbrace{u'^{*}(x)v(x)|_0^1}_{=0 \, (v(0) = v(1) = 0)} - \int_0^1 u''^{*}(x)v(x) \, dx \\
	&= \int_0^1 \left(u^*(x) - u''^{*}(x)\right) v(x) \, dx \\
	&= 0
\end{align}
For all test functions $v(\cdot)$.\\

\paragraph{Lemma: Fundamental Lemma of Calculus of Variations}
Suppose $g$ is continuous function on interval $[a,b]$. If
$$\int_a^b g(x)v(x) \, dx = 0$$ for all test functions $v(\cdot)$, then \\
$g(x) \equiv 0$ on $[a,b]$.\\
Then by Fundamental Lemma of Calculus of Variations, $(\heartsuit) \implies u^*(x) - u''^{*}(x) \equiv 0$, which is the 1st order necessary condition for $u^*(\cdot)$. \\
\tb{Step 2:} Find a function $u^*(\cdot)$ satisfying these conditions
\begin{align}
\begin{cases}
	u^*(x) = u''^{*}(x) \\
	u^*(0) = u^*(1) = 1
\end{cases} &\implies u^*(x) = c_1e^x + c_2e^{-x} \\
\begin{cases}
	1 = u^*(0) = c_1 + c_2 \\
	1 = u^*(1) = c_1e + c_2 \frac{1}{e}
\end{cases} &\implies c_1 = \frac{1}{e + 1}, c_2 = \frac{e}{e+1} \\
&\implies u^*(x) = \frac{1}{e + 1} e^x + \frac{e}{e+1}e^{-x}
\end{align}
\tb{Step 3:} check $u^*(\cdot)$ is in fact a global minimizer. \\
We derived that 
$$F[u^*(\cdot) + sv(\cdot)] = F[u^*(\cdot)] + \underbrace{s\int_0^1 \{ u^*(x)v(x) + u'^{*}(x)v'(x)\} \, dx}_{=0} + \underbrace{\frac{s^2}{2} \int_0^1 \{ v(x)^2 + v'(x)^2 \} \, dx}_{\geq 0}$$
$$F[u^*(\cdot)] \leq F[u^*(\cdot) + sv(\cdot)]$$
for all test functions $v(\cdot)$ and all $s \in \real$. In particular, let $s = 1$, then\\
$$F[u^*(\cdot)] \leq F[u^*(x) + v(\cdot)]$$
for all test functions $v(\cdot)$. In particular, let $v(\cdot) = u(\cdot) - u^*(\cdot)$, then
$$F[u^*(\cdot)] \leq F[u(\cdot)]$$ for all $u(\cdot) \in \mc{A}$. \\
\textcolor{red}{Note: The space of $v(\cdot)$ is a vector space, but $\mc{A}$ is not a vector space (since $u(\cdot) \neq 0$). It is a translate of a vector space.}
\paragraph{\blue{Lemma: Fundamental Lemma of Calculus of Variations}}
Suppose $g$ is continuous function on interval $[a,b]$. If
$$\int_a^b g(x)v(x) \, dx = 0$$ for all test functions $v(\cdot)$, then \\
$g(x) \equiv 0$ on $[a,b]$.\\
\begin{proof}
	Suppose for contradiction that $g(x) \not\equiv 0$ on $[a, b]$. \\
	WLOG, $g(x_0) > 0$ for some $x_0 \in (a,b)$. This implies that $g > 0$ on $(c, d) \subsetneq (a,b)$. \\
	Let $v(\cdot)$ be a continuous function s.t. 
	$$ v(\cdot) \begin{cases}
		> 0 \quad \text{on $(c,d)$} \\
		= 0 \quad \text{otherwise}
	\end{cases}$$
	Then
	$$\int_a^b g(x)v(x)\, dx = \int_c^d \underbrace{g(x) v(x)}_{>0} \, dx >0$$ which leads to a contradiction.
\end{proof}

\subsection{Classical Problem: the Brachistochrone}
Galileo (1638): Find the curve connecting A and B on which a point mass moves without fiction under the influence of gravity in the least time possible.\\
Johann Bernoulli (1696): Revisit the problem and sent invitations \\
6 correct solutions sent (1697): \\
Leibniz, Johann, \textcolor{blue}{Jacob}, l'Hospital, Von Tschinhaus, Anonymous $\rightarrow$ Newton (*)\\
\textcolor{blue}{This answer is the beginning of Calculus of Variations}


\subsection{General class of problems in Calculus of Variations}
$$\mc{A} = \{ u: [a,b] \rightarrow \real | u \in C^1, u(a) = A, u(b) = B\}$$
$$F[u(\cdot)] = \int_a^b L(x,u(x),u'(x))\,dx$$ 
where $L(x,z,p):[a,b] \times \real \times \real \rightarrow \real$

\paragraph{Model example}
$L(x, z, p) = \frac{z^2 + p^2}{2}, \quad F[u(\cdot)] = \int_0^1 \frac{u(x)^2 + u'(x)^2}{2} \, dx$\\
\under{Notation:} 
$$\begin{cases}
	L_z (x,z,p) = \frac{\partial}{\partial z} L(x,z,p) \\
	L_p (x,z,p) = \frac{\partial}{\partial p} L(x,z,p)
\end{cases}$$

\paragraph{Definition}
Given $u(\cdot) \in \mc{A}$, suppose $\exists$ function $g(\cdot)$ on $[a,b]$ s.t.
$$\frac{d}{ds}|_{s=0} F[u(\cdot) + sv(\cdot)] = \int_a^b g(x) v(x) \, dx$$ for all test functions $v(\cdot)$, then $g(\cdot)$ is called the \under{variational derivative} of $F$ at $u(\cdot)$, \\denoted by $\frac{\delta F}{\delta u}(u)(\cdot)$ or $\frac{\delta F}{\delta u}(u)$ or $\frac{\delta F}{\delta u}$.

\paragraph{Analogy}
\tb{In finite dimensions}:
\begin{align}
\frac{d}{ds}|_{s=0}f(u+sv) &= \textcolor{red}{\nabla f(u)\cdot v} \\
&= \sum_i \nabla f(u)_i v_i	
\end{align}
for all $v \in \real^n$ \\
\tb{In Calculus of Variations ($\infty$ dimensional)}:
\begin{align}
	\frac{d}{ds}|_{s=0} F[u(\cdot) + sv(\cdot)] &= \int_a^b \textcolor{red}{\frac{\delta F}{\delta u}(u)(x)} v(x) \, dx \tag{where possible} \\
	&\sim \sum_x \frac{\delta F(u)}{\delta u} (x) v(x) \tag{a kind of an infinite sum}
\end{align}

\paragraph{Model example}
$L(x, z, p) = \frac{z^2 + p^2}{2}, \quad F[u(\cdot)] = \int_0^1 \frac{u(x)^2 + u'(x)^2}{2} \, dx$\\
$$\frac{d}{ds}|_{s=0} F[u(\cdot) + sv(\cdot)] = \hdots = \int_0^1 \textcolor{blue}{[u(x) - u''(x)]}v(x) \, dx$$ for all test functions $v(\cdot)$\\
Therefore 
$\frac{\delta F}{\delta u}(u)(x) = \textcolor{blue}{u(x) - u''(x)}$

\paragraph{Lemma}(1st order necessary conditions satisfied by a solution $u^*(\cdot) \in C^1$)
$$\mc{A} = \{ u: [a,b] \rightarrow \real | u \in C^1, u(a) = A, u(b) = B\}$$
If $u^*(\cdot) \in \mc{A}$ minimizes $F$ over $\mc{A}$, and if $\frac{\delta F}{\delta u}(u^*)(\cdot)$ exists and is continuous, then it must satisfy
$$\frac{\delta F}{\delta u}(u^*)(\cdot) \equiv 0$$
\begin{proof}
	note: $u^*(\cdot) + sv(\cdot) \in \mc{A}$ \\
	If $u^*(\cdot)$ is a minimizer of $F$, then
	$$F[u^*(\cdot) \leq F[u^*(\cdot) + sv(\cdot)]$$
	for all test functions $v$. \\
	Define $f(s) := F[u^*(\cdot) + sv(\cdot)]$, then $f(0) \leq f(s)$ for all $s \in \real$. \\
	Then 
	\begin{align}
		\int_a^b \frac{\delta F}{\delta u}(u^*)(\cdot)v(x) \, dx
		&= \frac{d}{ds}|_{s=0} F[u^*(\cdot) + sv(\cdot)] \tag{$\frac{\partial F}{\partial u}(u^*)(\cdot)$ exists} \\
		&= \frac{d}{ds}|_{s=0} f(s)\\
		&= f'(0) = 0 \tag{0 is the global minimize of $f$}
	\end{align}
	This implies that $\frac{\delta F}{\delta u}(u^*)(\cdot) \equiv 0$.
\end{proof}

\paragraph{Theorem: Leibniz Integral Rule}
Let $f(x,t)$ be a function such that both $f(x,t)$ and its partial derivative  $\frac{\partial}{\partial x} f(x,t)$ is continuous w.r.t. $t$ and $x$ in some region of the $(x,t)$-plane, including $a(x) \leq t \leq b(x), x_0 \leq x \leq x_1$. Also suppose that the functions $a(x)$ and $b(x)$ are both continuous and both have continuous derivatives for $x_0 \leq x \leq x_1$. Then, for $x_0 \leq x \leq x_1$,
$$\frac{d}{dx}\left(\int_{a(x)}^{b(x)} f(x,t) \, dt\right) = f(x, b(x))\cdot \frac{d}{dx} b(x) - f(x, a(x)) \cdot \frac{d}{dx} a(x) + \int_{a(x)}^{b(x)} \frac{\partial}{\partial x} f(x,t) \, dt$$
Note that if $a(x)$ and $b(x)$ are constants rather than functions of $x$, we have a special case of Leibniz's rule:
$$\frac{d}{dx} \left( \int_a^b f(x,t)\, dt\right) = \int_a^b \frac{\partial}{\partial x} f(x,t)\, dt$$

\paragraph{Theorem}
$$\mc{A} = \{ u: [a,b] \rightarrow \real | u \in C^1, u(a) = A, u(b) = B\}$$
$$F[u(\cdot)] = \int_a^b L(x,u(x),u'(x))\, dx$$
where $L \in C^2$. \\
Then if $u(\cdot) \in C^1$, then
$\frac{\delta F}{\delta u}(u)(\cdot)$ exists, is continuous, and
$$\frac{\delta F}{\delta u}(u)(x) = -\frac{d}{dx}[L_p(x,u(x),u'(x))] + L_z(x,u(x),u'(x))$$
	Let $v$ be a test function ($v(a) = v(b) = 0$)
	\begin{align}
		\frac{d}{ds} |_{s=0} F[u(\cdot) + sv(\cdot)] 
		&= \frac{d}{ds}|_{s=0}\int_a^b L(x, u(x) + sv(x), u'(x) + sv'(x)) \, dx \\
		&= \int_a^b \frac{d}{ds}|_{s=0} L(x,z,p)\, dx \tag{By Leibniz's rule}\\
		&= \int_a^b L_z(\cdot)v(x) + L_p(\cdot)v'(x)\, dx \\
		&= \int_a^b L_z(\cdot)v(x) \, dx + \int_a^b L_p(\cdot)v'(x)\, dx \\
		&= \int_a^b L_z(\cdot)v(x) \, dx + \textcolor{red}{L_p(\cdot)v(x)|_a^b} - \int_a^b \frac{d}{dx}L_p(\cdot)v(x) \, dx \tag{Integration by parts} \\
		&= \int_a^b [-\frac{d}{dx}L_p(\cdot) + L_z(\cdot)]v(x) \, dx \text{  $\forall$ test functions $v(\cdot)$} 
	\end{align}
	By the definition of variational derivative, it follows that 
	$$\frac{\delta F}{\delta u}(u)(x) = -\frac{d}{dx}[L_p(x,u(x),u'(x))] + L_z(x,u(x),u'(x))$$
	Furthermore, since $L(\cdot) \in C^2, -\frac{d}{dx}L_p(\cdot)$ and $L_z(\cdot)$ are  continuous. Moreover, $u(\cdot)$ and $u'(\cdot)$ are continuous, so is the composite function. Hence the variational derivative is continuous. 
\paragraph{Model example}
$L(x, z, p) = \frac{z^2 + p^2}{2}, \quad F[u(\cdot)] = \int_0^1 \frac{u(x)^2 + u'(x)^2}{2} \, dx$\\
$L_z(x,z,p) = z \implies L_z(x,u(x),u'(x)) = u(x)$ \\
$L_p(x,z,p) = p \implies L_p(x,u(x),u'(x)) = u'(x)$ \\
$$\frac{\delta F}{\delta u}(u(\cdot)) = -\frac{d}{dx}[u'(x)] + u(x) = -u''(x) + u(x)$$
If $u^*(\cdot) \in \mc{A}$ is a minimizer, then
$-u''(x) + u(x) = 0$
\example[min arclength]
We will show that the straight line gives the shortest path.\\
$$\min \, F[u(\cdot)] = \int_a^b(1+u'(x)^2)^\frac{1}{2}\,dx = \text{arclength of $u(\cdot)$}$$
$$\mc{A} = \{ u: [a,b] \rightarrow \real | u \in C^1, u(a) = A, u(b) = B\}$$
Then $L(x,z,p) = (1+p^2)^{\frac{1}{2}}, L_z = 0$ and $L_p = \frac{p}{(1+p^2)^\frac{1}{2}}$ \\
If $u^*(\cdot)$ is a minimizer, then
$$- \frac{d}{dx}\frac{u'(x)}{(1+u'(x))^\frac{1}{2}} \equiv 0$$
$\implies \frac{u'(x)}{(1+u'(x))^\frac{1}{2}} = const$ \\
$\implies u'(x)^2 = const(1 + u'(x)^2)$ \\
Then $u'(x) = \alpha$ for some $\alpha \in \real$. \\
Then $u(x) = \alpha x + \beta $ for some $\beta \in \real$.

\example[Surface Area of Revolution]
Suppose $u(\cdot) \in C^1$ on $[a,b]$, the surface area of rotating the curve $u$ connecting $a$ and $b$ can be computed as
$$F[u(\cdot)] = \int_a^b 2\pi u(x) \sqrt{1 + u'(x)} \, dx$$
For simplicity, assume $u > 0$. 
In this example, the space of feasible functions is
$$ \mc{A} = \{ u:[a,b] \rightarrow \real: u \in C^1, u(a) = A, u(b) = B, u > 0 \}$$
If $u(\cdot)$ solves the minimization problem, it must be the case that
$$\frac{\delta F}{\delta u}(u)(\cdot) \equiv 0 \quad (\dagger)$$
Notice
$$L(x,z,p) = 2\pi z \sqrt{1 + p^2}$$
$$L_z(x,z,p) = 2\pi \sqrt{1 + p^2}$$
$$L_p(x,z,p) = 2\pi z \frac{p}{\sqrt{1 + p^2}}$$
\tb{Claim: } the family of $u(\cdot) = \beta \cosh(\frac{x - \alpha}{\beta})$ solves the necessary condition $\dagger$. \\
\textbf{Instance 1} When $a=0, b=1, A=B=1$, plugging in the initial condition gives
		\begin{align}
			\begin{cases}
				\beta \cosh \left(\frac{0-\alpha}{\beta}\right) &= 1 \\
				\beta \cosh \left(\frac{1-\alpha}{\beta}\right) &= 1
			\end{cases}
		\end{align}
		solving above system of equations provides the solution. \\
		\textbf{Instance 2} When $a=0, b=1, A=1, B=0$, plugging in these initial conditions gives
		\begin{align}
			\begin{cases}
				\beta \cosh \left(\frac{0-\alpha}{\beta}\right) &= 1 \\
				\beta \cosh \left(\frac{1-\alpha}{\beta}\right) &= 0
			\end{cases}
		\end{align}
		because $\cosh > 0$, the second equation suggests $\beta=0$, but in this case the first equation would never hold. \\
		Therefore, there is no solution to this calculus of variation. \\
		In face, the surface area is minimized by
		\begin{align}
			u(x) = \begin{cases}
				1 &\text{ if } x = 0 \\
				0 &\text{ otherwise}
			\end{cases}
		\end{align}

\subsection{Euler-Lagrange Equations in $\real^n$}
	\paragraph{Setup}
	\begin{align}
		F[u(\cdot)] &= \int_a^b L(x, u(x), u'(x))\ dx \\
		u &: [a, b] \to \real^n \\
		L(x, z, p) &: [a, b] \times \real^n \times \real^n \to \real \\
		\mc{A} &:= \left\{
			u: [a, b] \to \real^n: u \in C^1, u(a) = \textbf{A}, u(b) = \textbf{B}
		\right\} 
	\end{align}

	\begin{theorem}[Euler-Lagrange Equations in Vector Forms]
		\begin{align}
			- \frac{d}{dx} \nabla_p L(x, z, p) + \nabla_z L(x, z, p) &= \textbf{0} \in \real^n \quad (\dagger)
		\end{align} 
	\end{theorem}
	
	\begin{example}[Classical Lagrangian Mechanics]
		\begin{align}
			V(x): \real^n \to \real &\text{ potential energy} \\
			\frac{1}{2} m \norm{v}_2^2 &\text{ kinetic energy} \\
			L(t, x, v):= \frac{1}{2} m \norm{v}_2^2 - V(x) &\text{ difference between KE and PE}
		\end{align}
		Consider a path $x(t)$ in $\real^n$, define objective function as
		\begin{align}
			F[x(\cdot)] &= \int_a^b L(t, x(t), x'(t))\ dt \\
			&= \int_a^b \frac{1}{2} m \norm{\dot{x}(t)}_2^2 - V(x(t))\ dt
		\end{align}
		The Euler-Lagrange equation in vector form implies
		\begin{align}
			- \frac{d}{dt} \nabla_{(3)} L(t, x(t), \dot{x}(t))
			+ \nabla_{(2)} L(t, x(t), \dot{x}(t)) &= 0 \\
			\implies - \frac{d}{dt} m \dot{x}(t) - \nabla V(x(t)) &= 0 \\
			\implies m \ddot{x}(t) &= \nabla V(x(t))\quad (\dagger \dagger)
		\end{align}
	\end{example}
	
	\begin{remark}
		($\dagger \dagger$) is often referred to as \emph{Newton's second law}: object moves along the path on which the total conversion between kinetic and potential energies is minimized.
	\end{remark}
	
	\begin{example}[3-Dimensional Pendulum]
		Suppose the pendulum is moving on a path such that the total conversion between kinetic and potential energies is minimized, that is
		\begin{align}
			\min \int_a^b L(t)\ dt &= \int_a^b \frac{1}{2} m (\dot{x}(t)^2 + \dot{y}(t)^2 + \dot{z}(t)^2) - m g z(t)\ dt
		\end{align}
		with the restriction that $\norm{\textbf{x}(t)} = \ell$, where $\ell$ is the radius of the sphere. \\
		The restriction can be embodied by framing the problem using spherical coordinates:
		\begin{align}
			x &:= \ell \cos \varphi \sin \theta \\
			y &:= \ell \sin \varphi \sin \theta \\
			z &:= - \ell \cos \theta
		\end{align}
		where the path of motion can be characterized using $(\theta(t), \varphi(t))$. \\
		The objective function is therefore
		\begin{align}
			L\left( t, \begin{pmatrix} \theta(t) \\ \varphi(t)	\end{pmatrix}, \begin{pmatrix} \dot{\theta}(t) \\ \dot{\varphi}(t) \end{pmatrix} \right) &= \frac{1}{2} m \ell^2 (\dot{\theta}^2 + \dot{\varphi}^2 \sin^2 (\theta)) + mg\ell \cos \theta
		\end{align}
		So the Euler-Lagrange equation can be written as
		\begin{align}
			- \frac{d}{dt} \nabla_{(3)} L \left( t, \begin{pmatrix} \theta(t) \\ \varphi(t)	\end{pmatrix}, \begin{pmatrix} \dot{\theta}(t) \\ \dot{\varphi}(t) \end{pmatrix} \right)
			+ \nabla_{(2)} L\left( t, \begin{pmatrix} \theta(t) \\ \varphi(t)	\end{pmatrix}, \begin{pmatrix} \dot{\theta}(t) \\ \dot{\varphi}(t) \end{pmatrix} \right)
			= \textbf{0}
		\end{align}
	\end{example}
	\subsection{Equality constraints}
	\subsubsection{Isoperimetric constraints}
	\paragraph{Recall: finite dimensional case}
	\begin{align}
		f, g: \real^n \rightarrow \real \\
		\begin{cases}
			\underset{x \in \real^n}{\min} &f(x) \\
			&g(x) = const
		\end{cases}
	\end{align}
	Suppose regular point $x_*$ ($\nabla g(x_*) \neq 0$) is a minimizer.
	Then $\exists \lambda \in \real$ s.t.
	$\nabla f(x_*) + \lambda \nabla g(x_*) = 0$ (by Lagrange multipliers)
	\remark $x_*$ minimizes $f + \lambda g$. The Lagrange multipliers convert the original problem to an unconstrained optimization problem $L(x, \lambda) = f(x) + \lambda g(x)$.
	\paragraph{Infinite dimensional case}
	\begin{align}
		F[u(\cdot)] &= \int_a^b L^F(x, u(x), u'(x))\,dx \\
		G[u(\cdot)] &= \int_a^b L^G(x, u(x), u'(x))\,dx \\
	\end{align}
	$$		\begin{cases}
			\underset{u(\cdot) \in \mc{A}}{\min} &F[u(\cdot)] \\
			&G[u(\cdot)] = const
			\end{cases}$$
	Suppose regular point $u_*(\cdot)$ ($\frac{\delta G}{\delta u}(u_*) \neq 0$) is a minimizer, then $\exists \lambda \in \real$ s.t.
	$$\red{\frac{\delta F}{\delta u}(u_*) + \lambda \frac{\delta G}{\delta u}(u_*) \equiv 0} $$
	\remark $u_*$ minimizes $F + \lambda G$.
	
	\example
	\begin{align}
		\mc{A} &:= \{u: [-a, a] \rightarrow \real, u \in C^1, u(-a) = u(a) = 0 \}\\
		F[u(\cdot)] &= \int_a^b u(x) \, dx \\
		G[u(\cdot)] &= \int_a^b \sqrt{1 + u'(x)} \, dx = l > 0 \quad\text{note that $G$ is arg length}\\
		&\begin{cases}
			\underset{u \in \mc{A}}{\min} &(-F)[u(\cdot)] \\
			&G[u(\cdot)] = l
		\end{cases}
	\end{align}
	Let $u_*(\cdot)$ be a minimizer, then
	$$\frac{\delta F}{\delta u} = - \frac{d}{dx}L_p^F + L_z^F$$
	$$\frac{\delta G}{\delta u} = - \frac{d}{dx}L_p^G + L_z^G$$
	Then Euler-Lagrange equations suggests
	$$- \frac{d}{dx}L_p^F + L_z^F + \lambda(- \frac{d}{dx}L_p^G + L_z^G) = 0$$
	$$\implies \lambda^2 \frac{u'_*(x)^2}{1 + u'_*(x)^2} = (C_1 - x)^2 (\dagger)$$
	\paragraph{Claim:} Solution $u_*(\cdot)$ to ($\dagger$) satisfies
	$$(x-C_1)^2 + (u_*(x) - C_2)^2 = \lambda^2$$
	So that the graph of $u_*(\cdot)$ lies on a circle, and our solution is the semi-circle that has length $l$.\\
	\emph{Check}: 
		\begin{align}
			\frac{d}{dx} \left[ 2(x - C_1) + 2(u_*(x) - C_2)u_*'(x) \right] &= 0
		\end{align}
		which implies
		\begin{align}
			u'_*(x) &= - \frac{x - C_1}{u_*(x) - C_2} \\
			\implies u'_*(x)^2 &= \frac{(x - C_1)^2}{(u_*(x) - c_2)^2}\quad (\S)
		\end{align}
		Also, 
		\begin{align}
			(u_*'(x)^2)(u_*(x) - C_2)^2  &= (x - C_1)^2 + (u_*(x) - C_2)^2 = \lambda^2 \\
			\implies (u_*(x) - C_2)^2 &= \frac{\lambda^2}{1 + u'_*(x)^2}\quad (\S \S)
		\end{align}
		Combine $(\S)$ and $(\S \S)$,
		\begin{align}
			\frac{\lambda^2}{1 + u_*'(x)^2} u_*'(x)^2 &= (x - C_1)^2
		\end{align}
	It is possible to solve for $u$:
	$$\begin{cases}
		x=-a, y=0, (-a-C_1)^2 + (0-C_2)^2 = \lambda^2 \\
		x=+a, y=0, (+a-C)1)^2 + (0-C_2)^2 = \lambda^2 \\
		\int_{-a}^a \sqrt{1 + u'(x)^2} \, dx = l
	\end{cases}$$
\subsubsection{Holonomic constraints}
\paragraph{Setup(3-Dim Special Case)}
Minimize
$$F[x(\cdot), y(\cdot), z(\cdot)] = \int_a^b L(t, x(t), y(t), z(t), \dot{x}(t), \dot{y}(t), \dot{z}(t))\, dt$$
with constraint
$$H(x(t), y(t), z(t)) \equiv 0$$
\theorem[Euler-Lagrange Equations]
Let $\vx_*(t) := \begin{pmatrix}
	x_*(t) \\
	y_*(t) \\
	z_*(t)
\end{pmatrix}$ be the minimizer subject to the constraint, then
$$\begin{pmatrix}
	\frac{\delta F}{\delta x}[x_*(\cdot), y_*(\cdot), z_*(\cdot)](t) \\
	\frac{\delta F}{\delta y}[x_*(\cdot), y_*(\cdot), z_*(\cdot)](t) \\
	\frac{\delta F}{\delta z}[x_*(\cdot), y_*(\cdot), z_*(\cdot)](t) \\
\end{pmatrix} + \lambda(t)
\begin{pmatrix}
	H_x[x_*(\cdot), y_*(\cdot), z_*(\cdot)](t)\\
	H_y[x_*(\cdot), y_*(\cdot), z_*(\cdot)](t)\\
	H_z[x_*(\cdot), y_*(\cdot), z_*(\cdot)](t)\\
\end{pmatrix} = 0 \quad \forall t\in \real
$$
where $\lambda: [a,b] \rightarrow \real$ is a function.
	
	
	
	
	
	
	
	
	
	
	
	
	
	
	
	
	
	
	
	
	
	
	
	
	

\end{document}