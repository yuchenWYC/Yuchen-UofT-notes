\documentclass[11pt]{article}

% Libraries.
\usepackage{amsmath}
\usepackage{amssymb}
\usepackage{pgfplots}
\usepackage{graphicx}
\usepackage{enumitem}
\usepackage{hyperref}
\usepackage{fancyhdr}
\usepackage{perpage}
\usepackage{float}

% Property settings.
\MakePerPage{footnote}
\pagestyle{fancy}
\lhead{Notes by Y.W.}

% Commands
\newcommand{\ti}[1]{\textit{#1}}
\newcommand{\tb}[1]{\textbf{#1}}
\newcommand{\mb}[1]{\mathbb{#1}}
\newcommand{\under}[1]{\underline{#1}}
\newcommand{\proof}[0]{\textit{\underline{proof:} }}
\newcommand{\litran}[0]{$T: V \rightarrow W$ }
\newcommand{\slitran}[0]{Let $ T: V \rightarrow W$ be a linear transformation }
\newcommand{\mt}[0]{$[T]_\alpha^\beta$ }
\newcommand{\qed}[0]{$\hfill\blacksquare$}
\newcommand{\real}[0]{\mathbb{R}}
\newcommand{\vx}[0]{\tb{x}}
\newcommand{\vy}[0]{\tb{y}}
\newcommand{\vz}[0]{\tb{z}}
\newcommand{\vo}[0]{\tb{0}}
\newcommand{\vu}[0]{\tb{u}}
\newcommand{\trans}[3]{{#1}: {#2} \rightarrow {#3}}

% Attr.
\title{CSC263 \\ Lecture Notes}
\author{Yuchen Wang}
\date{\today}

\begin{document}
	\maketitle
	\tableofcontents
	\newpage
	\section{Notes On AVL Trees}
	\paragraph{Definition of AVL Trees} A binary tree is \ti{height-balanced} if the heights of the left and right subtrees of \ti{every} node \textcolor{red}{differ by at most one}. An AVL tree is a height-balanced binary search tree.
	\paragraph{Remark} By convention, the height of an empty tree is $-1$; the height of a tree consisting of a single node is $0$.
	\paragraph{Definition of balance factor} Let $h_R$ and $h_L$ be the heights of the right and left subtrees of a node $m$ in a binary tree respectively. The \ti{balance factor} of $m$, $BF[m]$, is defined as $BF[m] = h_R - h_L$. \newline
	For an AVL tree, the balanced factor of any node is $-1, 0$ or $+1$. \newline
	\begin{enumerate}
		\item if $BF[m] = +1, m$ is right heavy
		\item if $BF[m] = -1, m$ is left heavy
		\item if $BF[m] = 0, m$ is balanced
	\end{enumerate}
	In AVL trees we will store $BF[m]$ in each node $m$
	\paragraph{Algorithm Search}
	Treat T as an ordinary binary search tree
	\paragraph{Algorithm Insert}
	First insert $x$ in $T$ as in ordinary binary search trees: trace a path from the root downward, and insert a new node with key $x$ in it in the proper place, so as to preserve the binary search tree property. This may destroy the integrity of our AVL tree in that 
	
\end{document}