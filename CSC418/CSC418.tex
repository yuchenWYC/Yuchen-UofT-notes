\documentclass[11pt]{article}
% Libraries.

\usepackage{dsfont}
\usepackage{amsmath}
\usepackage{amssymb}
\usepackage{esint}
\usepackage[margin=3cm]{geometry}
%\usepackage{pgfplots}
\usepackage{graphicx}
\usepackage{enumitem}
\usepackage{hyperref}
\usepackage{fancyhdr}
\usepackage{perpage}
\usepackage[dvipsnames, pdftex]{xcolor}
\usepackage{float}
\usepackage{xargs}
\usepackage{/Users/raina/Desktop/uoft-notes/raina}
\usepackage[
	colorinlistoftodos,
	prependcaption,
	textsize=tiny
]{todonotes}


% Property settings.
\MakePerPage{footnote}
\pagestyle{headings}

% Attr.
\title{CSC418\\ Lecture Notes}
\author{Yuchen Wang}
\date{\today}

\begin{document}
    \maketitle
    \tableofcontents
    \newpage

\section{Ray Tracing}
\subsection{Shading}
\notation
The important variables in light reflection are \red{unit vectors}\\
\under{Light direction} $\vl$: a unit vector pointing toward the light source;\\
\under{View direction} $\vv$: a unit vector pointing toward the eye or camera;\\
\under{Surface normal} $\vn$: a unit vector perpendicular to the surface at the point where reflection is taking place.

\subsubsection{Lambertian Shading}
An observation by Lambert in the 18th century: the amount of energy from a light source that falls on an area of surface depends on the angle of the surface to the light.
\definition[Lambertian shading model]
The vector $\vl$ is computed by subtracting the intersection point of the ray and the surface from the light source position.\\
The pixel color
$$L = k_dI\max(0, \vn \cdot \vl)$$
where $k_d$ is the \ti{diffuse coefficient}, or the surface color; and $I$ is the intensity of the light source.
\begin{figure}[H]
	\centering
	\includegraphics[scale=0.7]{p1}
	\caption{Geometry for Lambertian shading}
\end{figure}
\remark
Because $\vn$ and $\vl$ are unit vectors, we can use $\vn \cdot \vl$ as a convenient shorthand for $\cos \theta$. This equation applies separately to the three color channels.
\remark
Lambertian shading is \ti{view independent:} the color of a surface does not depend on the direction from which you look. Therefore it does not produce any highlights and leads to a very matte, chalky appearance.

\subsubsection{Blinn-Phong Shading}
A very simple and widely used model for specular highlights by Phong (1975) and J.F.Blinn (1976).
\paragraph{Idea}
Produce reflection that is at its brightest when $\vv$ and $\vl$ are symmetrically positioned across the surface normal, which is when mirror reflection would occur; reflection then decreases smoothly as the vectors move away from a mirror configuration.\\
Compare the half vector $\vh$ with $\vn$: if $\vh$ is near the surface normal, the specular component should be bright and vice versa.
\definition[Blinn-Phong shading model]
\begin{align*}
	\vh &= \frac{\vv + \vl}{\norm{\vv + \vl}} \\
	L &= k_d I \max(0, \vn \cdot \vl) + k_s I \max(0, \vn \cdot \vh)^p
\end{align*}
where $k_s$ is the \ti{specular coefficient}, or the specular color of the surface, and $p > 1$.

\subsubsection{Ambient Shading}
A heuristic to avoid black shadows is to add a constant component to the shading model, one whose contribution to the pixel color depends only on the \blue{object hit}, with no dependence on the \blue{surface geometry} at all, as if surfaces were illuminated by ambient light that comes equally from everywhere.
\definition[simple shading model / Blinn-Phong model with ambient shading]
$$L = k_a I_a + k_d I \max (0, \vn \cdot \vl) + k_s I \max (0, \vn \cdot \vh)^n$$
where $k_a$ is the surface's ambient coefficient or ``ambient color", and $I_a$ is the ambient light intensity.

\subsubsection{Multiple Point Lights}
\property[superposition]
The effect by more than one light source is simply the sum of the effects of the light sources individually.
\definition[extended simple shading model]
$$L = k_a I_a + \sum_{i=1}^N[k_dI_i\max(0, \vn \cdot \vl_i) + k_sI_i \max(0, \vn \cdot \vh_i)^p]$$
where $I_i, \vl_i$ and $\vh_i$ are the intensity, direction, and half vector of the $i$-th light source.

\subsection{A Ray-Tracing Program}




















\end{document}

