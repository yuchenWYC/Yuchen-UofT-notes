\documentclass[11pt]{article}

% Libraries.
\usepackage{amsmath}
\usepackage{amssymb}
\usepackage{pgfplots}
\usepackage{graphicx}
\usepackage{enumitem}
\usepackage{hyperref}
\usepackage{fancyhdr}
\usepackage{perpage}
\usepackage{float}

% Property settings.
\MakePerPage{footnote}
\pagestyle{fancy}
\lhead{Notes by Yuchen Wang}

% Commands
\newcommand{\ti}[1]{\textit{#1}}
\newcommand{\tb}[1]{\textbf{#1}}
\newcommand{\mb}[1]{\mathbb{#1}}
\newcommand{\under}[1]{\underline{#1}}

% Attr.
\title{MAT224 Linear Algebra II \\ Lecture Notes}
\author{Yuchen Wang}
\date{\today}

\begin{document}
	\maketitle
	\tableofcontents
	\newpage
	\section{Vector Spaces}
	\subsection{Bases And Dimension (Jan 17)}
	\paragraph{Definition} A subset S of vector space V is called a \ti{basis} of V if V = Span(S) and S is linearly independent.
	\paragraph{Examples} 
	\begin{enumerate}
		\item the standard basis S = \{\tb{$e_1$},...,\tb{$e_n$}\} in $\mb{R}^n$, since every vector $(a_1, ..., a_n) \in \mb{R}^n$ may be written as the linear combination $(a_1,..., a_n) = a_1e_1+ ... + a_ne_n$
		\item The vector space $\mb{R}^n$ has many other bases as well. e.g., in $\mb{R}^2$, consider the set $S = \{(1,2),(1,-1)\}$, which is l.i.
		\item Let $V = P_n(\mb{R})$ and consider $ S = \{1, x, x^2, ..., x^n\}$, which is a basis of V.
		\item The empty subset, $\emptyset$, is a basis of the vector space consisting only of a zero vector, $\{\tb{0}\}$.
	\end{enumerate}
	\paragraph{Theorem 1.6.3} Let V be a vector space, and let S be a nonempty subset of V. Then S is a basis of V iff every vector $\tb{x} \in V$ may be written uniquely as a linear combination of the vectors in S. \newline
	\under{\ti{Proof:}}
	\paragraph{Theorem 1.6.6} Let V be a vector space that has a finite spanning set, and let S be a linearly independent subset of V. Then there exists a basis S' of V, with $S \subset S'$
	\paragraph{Lemma 1.6.8} Let S be a linearly independent subset of V and let $x \in V$, but $x \notin S$. Then $S \cup \{\tb{x}\}$ is l.i. iff $\tb{x} \notin Span(S)$.
	\paragraph{Insight} the number of vectors in a basis is, in a rough sense, a measure of "how big" the space is.
	\paragraph{Theorem 1.6.10} Let V be a vector space and let S be a spanning set for V, which has m elements. Then no linearly independent set in V can have more than m elements.
	\paragraph{Corollary 1.6.11} Let V be a vector space and let S and S' be two bases of V, with m and m' elements, respectively. Then m = m'.
	\paragraph{Definitions 1.6.12}
	\begin{enumerate}
		\item If V is a vector space with some finite basis(possibly empty), we say V is \under{\it{finite-dimentional}}.
		\item Let V be a finite-dimensional vector space. The dimension of V, denoted dim(V), is the number of vectors in a (hence any) basis of V.
		\item If $V = \{\tb{0}\}$, we define dim(V) = 0.
	\end{enumerate}
	\paragraph{Examples}
	\begin{enumerate}
		\item For each n, dim($\mb{R}^n$) = n, since the standard basis contains n vectors.
		\item dim($P_n(\mb{R})) = n + 1$, since a basis for $P_n(\mb{R})$ contains n + 1 functions.
		\item The vector spaces $P(\mb{R}), C^1(\mb{R})$ and $C(\mb{R})$ are not finite-dimensional. We say that such spaces are \under{\it{infinite-dimentional}}.
	\end{enumerate}
	\paragraph{Corollary 1.6.14} Let W be a subspace of a finite-dimensional vector space V. Then $dim(W) \leq dim(V).$ Furthermore, $	dim(W) = dim(V)$ iff $W = V$.
	\paragraph{Corollary 1.6.15} Let W be a subspace of $\mb{R}^n$ defined by a system of homogeneous linear equations. Then dim(W) is equal to the number of free variables in the corresponding echelon form system.
	\paragraph{Theorem 1.6.18} Let $W_1$ and $W_2$ be finite-dimensional subspaces of a vector space V. Then $$dim(W_1 + W_2) = dim(W_1) + dim(W_2) - dim(W_1 \cap W_2)$$
	\newpage
	\section{Linear Transformations}
	\subsection{Linear Tranformations}
	
	
\end{document}
