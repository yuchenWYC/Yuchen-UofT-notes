\documentclass[11pt]{article}
% Libraries.

\usepackage{amsmath}
\usepackage{amssymb}
\usepackage{esint}
\usepackage[margin=3cm]{geometry}
%\usepackage{pgfplots}
\usepackage{graphicx}
\usepackage{enumitem}
\usepackage{hyperref}
\usepackage{fancyhdr}
\usepackage{perpage}
\usepackage[dvipsnames, pdftex]{xcolor}
\usepackage{float}
\usepackage{xargs}
\usepackage{/Users/raina/Desktop/uoft-notes/raina}
\usepackage[
	colorinlistoftodos,
	prependcaption,
	textsize=tiny
]{todonotes}


% Property settings.
\MakePerPage{footnote}
\pagestyle{headings}

% Attr.
\title{MAT337\\ Lecture Notes}
\author{Yuchen Wang}
\date{\today}

\begin{document}
    \maketitle
    \tableofcontents
    \newpage

\section{Real Numbers}
\subsection{Discussion: The Irrationality of $\sqrt{2}$}
If we make natural numbers $\mb{N}$ closed under subtraction, we obtain 
$$\mb{Z} = \{ \hdots, -1, 0, 1, \hdots \}$$
If we take the closure of $\mb{Z}$ under division by non-zero numbers, we obtain
$$\mb{Q} = \{\frac{m}{n}: m \in \mb{Z}, n \in \mb{N}, (m,n) = 1 \}$$
\remark
$(m, n) = 1$ means that if $d \in \mb{N}$ divides both $m$ and $n$, then $d = 1$.

\theorem
There is no $r \in \mb{Q}$ s.t. $r^2 = 2$.

\begin{proof}
	Assume for contradiction that there are $m \in \mb{Z}. n \in \mb{N}$ s.t. $\frac{m}{n} = \sqrt{2}$ and $(m,n) = 1$. \\
	Then $m^2 = 2n^2$ so that $m^2$ is an even complete square. \\
	Suppose $m = p_1 \hdots p_r$ where $p_i$s are prime numbers. Then $2n^2 = m^2 = p_1^2 \hdots p_r^2 \implies p_i^2 = 2^2$. \\
	Then $4 | m^2$ and $2|n^2$, so $n$ has to be even. Therefore both $m$ and $n$ are even. \\
	Then $2 | m$ and $2|n$, which leads to a contradiction that if $d \in \mb{N}$ divides both $m$ and $n$, then $d = 1$.
\end{proof}

\subsection{Preliminaries}
\definition[set]
A \under{set} is any collection of objects.
\definition[function]
Given two sets $A$ and $B$, a \under{function} from $A$ to $B$ is a rule or mapping that takes each element $x \in A$ and associates with it a single element of $B$. In this case, we write $(f: A \rightarrow B)$. It is the set of pairs $(A, B) \in A \times B$ s.t.
\begin{enumerate}
	\item If $(x,y_1) \in f$ and $(x,y_2) \in f$, then $y_1 = y_2$.
	\item For all $x \in A$, there is some $y \in B$ s.t. $f(x) = y$.
\end{enumerate}
The set $A$ is said to be the \under{domain} of $f$. The \under{range} of $f$ is not necessarily equal to $B$ but refers to the subset of $B$ given by $\{ y \in B: y = f(x)$ for some $x \in A \}$.

\example[absolute value function]
For every $x$, 
$$|x| = \begin{cases}
	x \quad & x \geq 0 \\
	-x \quad & x < 0
\end{cases}$$

\theorem[triangle inequality]
$$|x+y| \leq |x| + |y|$$
\begin{proof}
	\begin{align*}
		(x + y)^2 &= x^2 + y^2 + 2xy \\
		&\leq |x|^2 + |y|^2 + 2|x||y| \\
		&= (|x| + |y|)^2 \\
		\implies |x + y| &= \sqrt{(x+y)^2}\\
		&\leq \sqrt{(|x| + |y|)^2} \\
		&= ||x| + |y||\\
		&= |x| + |y|
	\end{align*}
\end{proof}

\definition[maximum and minimum]
Assume set $X \subseteq \real$. Then the maximum (minimum) of $X$ is an element $a \in X$ s.t. for all $x \in X, x \leq a (x \geq a)$.

\definition[least upper bound / supremum]
The \under{least upper bound} of X (denoted by $\sup(X)$) is a real number $a \in \real$ s.t.

\begin{enumerate}
	\item For all $x \in X, x \leq a$ (this means that $a$ is an upper bound for $X$)
	\item If $b$ is an upper bound for $X$, then $a \leq b$
\end{enumerate}

\example
\begin{align*}
	\max([0, 1]) &= 1 \\
	\min([0, 1]) &= 0 \\
	\sup((0,1)) &= 1 \\
	\sup(\real), \sup(\mb{N}) \, DNE
\end{align*}
\subsection{The axiom of completeness}

\definition[initial segment]
$X \subseteq \mb{Q}$ is said to be an \under{initial segment} if
\begin{enumerate}
	\item $X \neq \emptyset$
	\item For all $x, y \in \mb{Q}$, if $x < y$ and $y \in X$, then $x \in X$.
	\item $X \neq \mb{Q}$
\end{enumerate}
\tb{Alternative definition: } Let $(A, \leq)$ be a well-ordered set. Then the set
$$\{a \in A: a < k\}$$ for some $k \in A$ is called an initial segment of $A$.

\definition[real numbers]
$\real = \{\sup(X): X$ is an initial segment of $\mb{Q}\}$ \\

\lemma[supremum] Suppose $A \subseteq \real$ and $s \in \real$ is an upper bound for $A$. If $\forall \epsilon > 0, \exists a \in A, a + \epsilon > s$, then $s = \sup(A)$
\begin{proof}
($\impliedby$) Assume for contradiction that $t \in \real$ is an upper bound for $A$ and $t < s$. \\
Let $\epsilon = \frac{s - t}{2}$. Obviously $\epsilon > 0$. \\
But then $\forall a \in A, a + \epsilon \leq t + \epsilon < s$, which is a contradiction. \\
($\implies$) Assume for contradiction that $\epsilon_0 > 0$ and $\forall a \in A, a + \epsilon \leq S$)\\
Then $\forall a \in A, a \leq S - \epsilon_0$. \\
So $s - \epsilon_0$ is an upper bound for $A$, which is a contradiction that $a + \epsilon > s$. 	
\end{proof}

\theorem[\blue{the Axiom of Completeness}] If $X \subset \real$ is bounded above, then $X$ has a least upper bound.
\begin{proof}
	For $x \in X$, let $Ax$ be the initial segment of $\mb{Q}$ corresponding to $x$. \\
	Since $X$ is bounded above, pick $b \in \real$ s.t. $\forall x \in X, x < b$. Then $b \notin \underset{x \in X}{\cup} Ax$. Note that $\underset{x \in X}{\cup} Ax$ is an initial segment of $\mb{Q}$. Then $\sup(\underset{x \in X}{\cup} Ax)$ is $\sup(X)$.
\end{proof}


\subsection{Consequences of Completeness}
\definition[nested sequence of sets]
Assume $\langle A_n: n \in \mb{N}\rangle$ is a sequence of sets. \\
$\langle A_n: n \in \mb{N}\rangle$ is said to be \under{nested} if
$$A_{n+1} \subseteq A_n$$

\theorem[Nested Interval Property]
Assume $\langle I_n: n \in \mb{N}\rangle$ is a nested sequence of \blue{closed intervals of $\real$}. Then $$\underset{n}{\cap} I_n \neq \emptyset$$
\begin{proof}
Let $[a_n, b_n] = I_n$ where $a_n, b_n \in \real$. \\
Since $\langle I_n | n \in \mb{N} \rangle$ is nested,
$$a_n \leq a_{n+1} \leq b_{n+1} \leq b_n \quad (\dagger)$$ for all $n \in \mb{N}$\\
Let $A = \{ a_n: n \in \mb{N} \}$.\\
Note that $b_1$ is an upper bound for $A$. So $A$ has a supremum in $\real$. \\
We claim that $\sup(A) \in \underset{n}{\cap} I_n$.\\
By $(\dagger)$, for all $n \in \mb{N}, \sup(A) \leq b_n$\\
Obviously, for all $n \in \mb{N}, \sup(A) \geq a_n$\\
So $\forall n \in \mb{N}, a_n \leq \sup(A) \leq b_n$. \\
Therefore $\forall n \in \mb{N}, \sup(A) \in [a_n, b_n]$. 
\end{proof}

\example
$$\underset{n \in \mb{N}}{\cap} (0, \frac{1}{n}) = \emptyset$$
$$\underset{n \in \mb{N}}{\cap} [0, \frac{1}{n}] = \{0\}$$

\theorem[Archimedian Property]
We have
\begin{enumerate}
	\item For every $y \in \real$, there is $n \in \mb{N}$ s.t. $y \leq n$.
	\item For every $y > 0$, there is $n \in \mb{N} s.t. \frac{1}{n} < y$.
\end{enumerate}
\begin{proof}
	(1) Assume for contradiction that $\mb{N}$ is bounded in $\real$. \\
	Let $\alpha = \sup(\mb{N})$. Then there is a natural number $n \in \mb{N}$ s.t. $n > \alpha - 1$. \\
	But then $n + 1 > (\alpha - 1) + 1 = \alpha$, which is a natural number greater than $\alpha$, contradiction. \\
	(2) Exercise.
\end{proof}

\theorem[density of $\mb{Q}$ in $\real$]
For every two real numbers $a$ and $b$ with $a < b$, there exists a rational number $r$ satisfying $a < r < b$.
\begin{proof}
Let $n \in \mb{N}$ s.t. $\frac{1}{n} < b - a, 1 < nb - na$. \\
Let $m \in \mb{Z}$ s.t. $na < m < nb$. \\
Then $a < \frac{m}{n} < b$. \\
Pick $r = \frac{m}{n}$ and we are done.
\end{proof}

\subsection{Cardinality}
``The size of a set"
\subsubsection{1-1 Correspondence}
\definition[one-to-one and onto] A function $f: A \rightarrow B$ is \under{one-to-one (1-1)} if $a_1 \neq a_2$ in $A$ implies that $f(a_1) \neq f(a_2)$ in $B$. The function $f$ is \under{onto} if, given any $b \in B$, it is possible to find an element $a \in A$ for which $f(a) = b$.

\proposition
If $f: A \rightarrow B$ and $g: B \rightarrow C$ is 1-1, then $g \circ f: A \rightarrow C$ is 1-1.


\remark
If a function $f: A \rightarrow B$ is both 1-1 and onto, then there is a 1-1 correspondence between two sets.

\definition[the same cardinality] The set $A$ \under{has the same cardinality as $B$} if there exists $f: A \rightarrow B$ that is $1-1$ and onto. In this case, we write $A \sim B$.

\proposition
If $A \sim B, B \sim C$, then $A \sim C$

\proposition
If $Card(A) \leq Card(B) \leq Card(C)$, then $Card(A) \leq Card(C)$

\subsubsection{Countable Sets} 
A set $A$ is \under{countable} if $\mb{N} \sim A$. An infinite set that is not countable is called an \under{uncountable} set.

\theorem
The set $\mc{Q}$ is countable.

\begin{proof}
	Set $A_1 = \{0\}$ and for each $n \geq 2$, let $A_n$ be the set given by
	$$A_n = \{ \pm \frac{p}{q}: \text{where }p,q \in \mb{N} \text{ are in lowest terms with} p + q = n \}$$ 
	e.g. $A_2 = \{ \frac{1}{1}, \frac{-1}{1}\}, A_3 = \{\frac{1}{2}, \frac{-1}{2}, \frac{2}{1}, \frac{-2}{1}\}$
	
\begin{figure}[H]
	\centering
	\includegraphics[scale=0.6]{p1.png}
\end{figure}
\noindent The above correspondence is onto because every rational number appears in the correspondence exactly once. The above correspondence is 1-1 because $A_N$ were constructed to be disjoint so that no rational number appears twice.
\end{proof}

\theorem
The set $\real$ is uncountable.

\begin{proof}
	Assume for contradiction that there does exist a bijection function $f: \mb{N} \rightarrow \real$. \\
	Let $x_1 = f(1), x_2 = f(2)$ and so on. Then since $f$ is onto, can write
	\begin{equation}
			\real = \{x_1, x_2, x_3, x_4, \hdots\}
	\end{equation}
	and be confident that every real number appears somewhere on the list. \\
	We will now use the Nested Interval Property to produce a real number that is not there.
	Let $I_1$ be a closed interval that does not contain $x_1$. given an interval $I_n$, construct $I_{n+1}$ to satisfy $I_{n+1} \subseteq I_n$ and $x_{n+1} \notin I_{n+1}$. \\
	If $x_{n_0}$ is some real number from the list in $(1)$, then we have $x_{n_0} \notin I_{n_0}$, and it follows that
	$$ x_{n_0} \notin \cap_{n=1}^\infty I_n$$
	Since we are assuming that the list in $(1)$ contains every real number, then
	$$\cap_{n=1}^\infty I_n = \emptyset$$
	However, the NIP asserts that $\cap_{n=1}^\infty I_n \neq \emptyset$, which is a contradiction.
\end{proof}

\theorem If $A \subseteq B$ and $B$ is countable, then $A$ is either countable or finite.

\theorem We have\\
(i) If $A_1, A_2, \hdots, A_m$ are countable sets, then the union $A_1 \cup A_2 \cup \hdots \cup A_m$ is countable. \\
(ii) If $A_n$ is a countable set for each $n \in \mb{N}$, then $\cup_{n=1}^\infty A_n$ is countable.

\theorem The open interval $(0,1) = \{x \in \real: 0 < x <1\}$ is uncountable.

\subsection{Cantor's Theorem}
\notation
Given a set $A$, the power set $P(A)$ refers to the collection of all subsets of $A$.

\theorem[Cantor's Theorem]
Given any set $A$, there does not exist a function $f: A \rightarrow P(A)$ that is onto.
\begin{proof}
	Assume, for contradiction, that $f: A \rightarrow P(A)$ is onto. For each element $a \in A$, $f(a)$ is a particular subset of $A$. The assumption that $f$ is onto means that every subset of $A$ appears as $f(a)$ for some $a \in A$. To arrive at a contradiction, we will produce a subset $B \subseteq A$ that is not equal to $f(a)$ for any $a \in A$.\\
	Construct $B$ using the following rule. For each element $a \in A$, consider the subset $f(a)$. This subset of $A$ may contain the element $a$ or it may not. This depends on the function $f$. If $f(a)$ does not contain $a$, then we include $a$ in our set $B$: Let 
	$$B = \{a \in A: a \notin f(a)\}$$
	Since we have assumed that our function $f: A \rightarrow P(A)$ is onto, it must be that $B = f(a')$ for some $a' \in A$.\\
	\tb{Case 1} $a' \in B$ \\
	Then $a' \notin f(a') = B$, a contradiction. \\
	\tb{Case 2} $a' \notin B$ \\
	Then $a' \in f(a') = B$, a contradiction.
\end{proof}


\theorem[Schr\"oder-Bernstein Theorem]
If there are 1-1 functions $f: A \rightarrow B$ and $h: B \rightarrow A$, then there is a bijection $g: A \rightarrow B$.

\begin{proof}
	\tb{Claim:} the statement of the theorem is equivalent to the following:\\
	If $B \subseteq A$ and $f: A \rightarrow B$ is $1-1$, then there is a bijection $g: A \rightarrow B$. \quad (*) \\\\
	\tb{proof of claim:}
	theorem $\implies$ (*): \\
	Take $h: X \rightarrow Y$ with $h(x) = x$, then $X \subseteq Y$. \\
	(*) $\implies$ theorem: \\
	Let $f: A \rightarrow B$ and $h: B \rightarrow A$ be 1-1 functions, as in the theorem. We need to show that there is bijection $g: A \rightarrow B$. \\
	Notice that $A \subseteq h(B)$ and $h \circ f: A \rightarrow h(B) $ is a 1-1 function. So by (*), there is a bijection $g_0: A \rightarrow h(B)$. \\
	But $h: B \rightarrow h(B)$ is also a bijection. So $g = h^{-1} \circ g_0: A \rightarrow B$ is a bijection (using the fact that bijections are closed under compositions). \\\\
	Now it suffices to prove (*). \\
	Assume set $X \subseteq Y$ and $f: Y \rightarrow X$. Let $W = \bigcup_{n=0}^\infty f^n(Y \setminus X)$.\\
	Define $g: Y \rightarrow X$ by:
	\begin{itemize}
		\item If $y \in W$, then $g(y) = f(y)$
		\item If $y \in Z:= Y \setminus W$, then $g(y) = y$
	\end{itemize}
	We need to show that $g: Y \rightarrow X$ is a well-defined bijection. \\
	Since $f$ is 1-1, for all $m < n$, $f^m(Y \setminus X) \cap f^n(Y \setminus X) = \emptyset$\\
	Note that 
	\begin{align*}
		Y \setminus W &= Y \setminus \bigcup_{n=0}^\infty f^n(Y \setminus X) \\
		&= [Y \setminus (Y \setminus X)] \setminus \bigcup_{n=1}^\infty f^n(Y \setminus X) \\
		&= X \setminus \bigcup_{n=1}^\infty f^n(Y \setminus X)
	\end{align*}
	Therefore for all $y \in Y, g(y) \in X$. \\
	(Show $g$ is 1-1) Now assume $y_1, y_2 \in Y$ and $g(y_1) = g(y_2)$. We show that $y_1 = y_2$. \\
	\tb{Case 1} $y_1, y_2 \in W$ \\
	Then $g(y_1) = g(y_2) \implies f(y_1) = f(y_2) \implies y_1 = y_2$. \\
	\tb{Case 2} $y_1 \in W$ but $y_2 \in Y\setminus W$ \\
	Then $g(y_1) = g(y_2) \implies f(y_1) = y_2$ \\
	Note that if $y_1 \in W$, then for some $n \geq 0, y_1 \in f^n(Y \setminus X)$ \\
	Then $y_2 \in f^{n+1}(Y \setminus X) \subseteq W$ \\
	So $y_2 \in W$, which leads to a contradiction. \\
	\tb{Case 3} $y_1, y_2$ are both in $Z:= Y \setminus W$\\
	Then $g(y_1) = g(y_2) \implies y_1 = y_2$. \\
	Therefore by case 1,2,3, $g$ is 1-1.\\
	(Show $g$ is onto) Let $x \in X$. We need to find $y \in Y$ s.t. $g(y) = X$. \\
	If $x \in Z$, take $y = x$. \\
	If $x \in \bigcup_{n=1}^\infty f^n(Y \setminus X)$, then fix $n \in \mb{N}$ s.t. $x \in f^n(Y \setminus X)$. \\
	But $f^n(Y \setminus X) = f(f^{n-1}(Y \setminus X))$ \\
	Pick $y \in f^{n-1}(Y \setminus X)$ s.t. $f(y) = x$. \\
	Then $y \in W$ and $g(y) = x$. Therefore $g$ is onto. 
\end{proof}

\section{Sequences and Series}
\subsection{The Limit of a Sequence}
\definition[sequence]
A \under{sequence} is a function whose domain is $\mb{N}$.


\definition
Let $(X, d)$ be a metric space. A sequence $(X_n) \subseteq X$ \under{converges} to an element $x \in X$ if $\forall \epsilon > 0, \exists N \in \mb{N}, n \geq N \implies d(x_n, x) < \epsilon$. \\\\
\tb{Key property: } If $\underset{n \rightarrow \infty}{\lim} x_n = x, \underset{n \rightarrow \infty}{\lim} x_n = y$, then $x = y$.
\begin{proof}
	WTS $d(x,y) = 0$ \\
	Let $\epsilon > 0$. We will show that $d(x,y) < \epsilon$. \\
	Since $\underset{n \rightarrow \infty}{\lim} x_n = x$, then $\exists N_1, \forall n \geq N_1, d(x_n, x) < \frac{\epsilon}{2}$ \\
	Since $\underset{n \rightarrow \infty}{\lim} x_n = y$, then $\exists N_2, \forall n \geq N_2, d(x_n, y) < \frac{\epsilon}{2}$ \\
	Take $n \geq \max(N_1, N_2)$, then $d(x,y) \leq d(x_n, x) + d(x_n, y) < \frac{\epsilon}{2} + \frac{\epsilon}{2} = \epsilon$.
\end{proof}

\proposition Suppose $(X, d)$ is a metric space, $(X, \tau)$ is a topological space, and $F \subseteq X$. If $\underset{n \rightarrow \infty}{\lim} x_n = x$, $(x_n) \subseteq F$ and $F$ is closed, then $x \in F$.
\begin{proof}
	Suppose $x \notin F$, i.e., $x \in X \setminus F$. \\
	Since $F$ is closed, then $X \setminus F$ is open, so there is $\epsilon > 0$ s.t. $B_\epsilon(x) \subseteq X \setminus F$. \\
	Let $N$ be such that $\forall n \geq N, d(x_n, x) < \epsilon$.\\
	 Then $x_n \in B_\epsilon(x)$, which implies that $(x_n) \subseteq X \setminus F$, a contradiction.
\end{proof}

\proposition Suppose $(X, d)$ is a metric space and $F \subseteq X$. If $F$ is not closed, then there exists $(x_n) \subseteq F$ and $x \notin F$ s.t. $\underset{n \rightarrow \infty}{\lim} x_n = x$.

\begin{proof}
	If $F$ is not closed, then $X \setminus F$ is not open, so there is $x \in X \setminus F$ s.t. $B_\epsilon(x) \not \subseteq X \setminus F$ for all $\epsilon > 0$. \\
	Take $x_n \in B_{1/n}(x) \setminus (X \setminus F)= B_{1/n}(x) \cap F$ for each $n \in \mb{N}$, then $(x_n) \subseteq F$ and $\underset{n \rightarrow \infty}{\lim} x_n = x$.
\end{proof}

\definition[Cauchy sequence]
 A sequence $(x_n)$ in a metric space $(x_n)$ in a metric space $(X, d)$ is a \under{Cauchy sequence} if $\forall \epsilon >0, \exists N \in \mb{N}, m, n \geq N \implies d(x_m, x_n) < \epsilon$.
 
\proposition
A convergent sequence is Cauchy. 
\begin{proof}
	Let $(x_n)$ be a convergent sequence, so that $\underset{n \rightarrow \infty}{\lim} x_n = x$. To check $(x_n)$ is Cauchy, let $\epsilon > 0$. We need to find $N$ s.t. $\forall m, n \geq N, d(x_n, x_m) < \epsilon$.\\
	Apply $\underset{n \rightarrow \infty}{\lim} x_n = x$ to $\frac{\epsilon}{2}$, we get $N$ s.t. $\forall n \geq N, d(x, x_n) < \frac{\epsilon}{2}$. \\
	Notice that $N$ works for Cauchy: \\
	Take $m, n \geq N$, then $$d(x_n, x_m) \leq d(x_n, x) + d(x, x_m) < \frac{\epsilon}{2} + \frac{\epsilon}{2} = \epsilon$$
\remark
When $X = \real$ with the usual metric, A Cauchy sequence is convergent (the converse is true). \\
In general not true. For example, $X = \real \setminus \{0\}, d(x,y) = |x - y|, (x_n) = \frac{1}{n}$.

\end{proof}



\definition[monotone sequence]
$(x_n) \subseteq \real$ is \under{monotone} if either $x_n \leq x_m, n \leq m$, or $x_n \geq x_m, n \leq m$.

\theorem[Monotone Subsequence Theorem]
Every sequence $(x_n) \subseteq \real$ has a monotone subsequence.
\todo{prove this}

\fact If $a_n \leq b_n$ for all $n$, $a = \limit a_n, b = \limit b_n$, then 
$$a \leq b$$

\begin{proof}
	Suppose for contradiction that $a > b$. Let $\epsilon = \frac{a - b}{2}$.\\
	We know $\exists N_1$ s.t. $a_n \in B_\epsilon(a)$ for $n \geq N_1$ and $\exists N_2$ s.t. $b_n \in B_\epsilon(b)$ for $n \geq N_2$. Take $n > \max(N_1, N_2)$, then we have
	$$b_n < \frac{a+b}{2} < a_n$$
	which is a contradiction.
\end{proof}

\theorem[Algebraic limit theorem] Suppose $a = \limit a_n, b = \limit b_n$, then:
\begin{enumerate}
	\item $a + b = \limit (a_n + b_n)$
	\item $ab = \limit a_nb_n$
	\item $\frac{a}{b} = \limit \frac{a_n}{b_n}$, and $b \neq 0$.
\end{enumerate}

\fact Monotone bounded sequence $(x_n)$ converges to its supremum or infimum.
\begin{proof}
	We only prove the supremum case. \\
	Fix $\epsilon > 0$, let $s = \sup\{x_n: n \in \mb{N}\}$. We have 
	$s - \epsilon < s$ and thus $s - \epsilon$ is not an upper bound of $(x_n)$. Therefore, there is $N$ s.t. $x_N > s-\epsilon$. \\
	Take $n \geq N$, then we have
	$$x_n \geq x_N > s-\epsilon$$
	Therefore, we have $|x_n - s| < \epsilon$.
\end{proof}

\definition[limit supremum]
We define
$$\underset{n \rightarrow \infty}{\lim \sup}\, x_n = \inf\{y_m:m \in \mb{N}\}$$
where $y_m = \sup\{x_n: n\geq m\}$. \\
Alternatively,
$$\underset{n \rightarrow \infty}{\lim \sup}\, x_n = \underset{m \rightarrow \infty}{\lim} \underset{n \geq m}{\sup} x_n$$
\definition[limit infimum]
$$\underset{n \rightarrow \infty}{\lim \inf}\, x_n = \sup\{z_m:m \in \mb{N}\}$$
where $z_m = \inf\{x_n: n\geq m\}$.\\
Alternatively,
$$\underset{n \rightarrow \infty}{\lim \inf}\, x_n = \underset{m \rightarrow \infty}{\lim} \underset{n \geq m}{\inf} x_n$$

\subsection{Series}
\definition 
We define
$$S_n = \sum_{k=1}^n a_k, \quad \limit S_n = \sum_{k=1}^\infty a_k$$
We call $\sum_{k=1}^\infty a_k$ a \under{summable series} if the limit exists,
 i.e.,
 $$\exists A, \forall \epsilon > 0, \exists N s.t. \forall n \geq N, |S_n - A| < \epsilon$$

\property[Cauchy criterion for series] $\sum_{k=1}^\infty$ is \under{summable} iff 
$$\forall \epsilon > 0, \exists N \text{ s.t. } \forall n \geq m \geq N, |S_n - S_m| = \left|\sum_{k=m+1}^n a_k \right |< \epsilon $$

\corollary If $\sum_{k=1}^\infty a_k$ is summable, then $|a_k| \rightarrow 0$.
\begin{proof}
	We have $|a_k| = |s_k - s_{k-1}| < \epsilon$ for $k > N$.
\end{proof}

\example
$\sum_{k=1}^\infty \frac{1}{k^2}$ is summable.
\begin{proof}
	\begin{align*}
		S_m &= 1 + \frac{1}{4} + \frac{1}{9} + \hdots + \frac{1}{m^2} \\
		&< 1 + \frac{1}{2\cdot1} + \frac{1}{3\cdot 2} + \hdots + \frac{1}{m(m-1)} \\
		&= 1 + (1-\frac{1}{2}) + (\frac{1}{2} - \frac{1}{3}) + \hdots + (\frac{1}{m-1} - \frac{1}{m}) \\
		&= 1 + 1 - \frac{1}{m} \\
		&< 2
	\end{align*}
\end{proof}

\example
$\sum_{k=1}^\infty \frac{1}{k} = \infty$
\begin{proof}
	We have
	\begin{align*}
		\sum_{k=1}^\infty \frac{1}{k} &= (1/2) + (1/3 + 1/4) + (1/5 + 1/6 + 1/7 + 1/8) + \hdots \\
		&= 1 + (1/2) + (1/4 + 1/4) + (1/8 + 1/8 + 1/8 + 1/8) + \hdots \\
		&= 1 + 1/2 + 1/2 + 1/2 + \hdots \\
		\rightarrow \infty
	\end{align*}
\end{proof}

\theorem[Algebraic limit theorem for series] Suppose $\sum_{k=1}^\infty a_k = A, \sum_{k=1}^\infty b_k = B, c \in \real$, then
\begin{enumerate}
	\item $\sum_{k=1}^\infty ca_k = cA$
	\item $\sum_{k=1}^\infty (a_k + b_k) = A + B$
\end{enumerate}

\begin{proof}
	(1) We want to show $\forall \epsilon > 0, \exists N$ s.t. $\forall n \geq N, \left | \sum_{k=1}^\infty ca_k - cA\right | < \epsilon$. \\
	We know $\forall \epsilon_0 > 0, \exists N_{\epsilon_0}$ s.t. $\forall n \geq N_{\epsilon_0}, \bignorm{\sum_{k=1}^\infty a_k - A} < \epsilon_0$.\\
	Take $\epsilon_0 = \frac{\epsilon}{|c|}$, then we have
	$$\bignorm{\sum_{k=1}^\infty ca_k - cA} = |c|\bignorm{\sum_{k=1}^\infty a_k - A} < |c|\cdot\frac{\epsilon}{|c|} = \epsilon$$
\end{proof}

\property[Order comparison test] Suppose $b_k \geq a_k \geq 0, \forall k$.\\
\begin{enumerate}
	\item If $\sum_{k=1}^\infty b_k < \infty$ , then $\sum_{k=1}^\infty a_k < \infty$.
	\item If $\sum_{k=1}^\infty a_k = \infty$ , then $\sum_{k=1}^\infty b_k = \infty$.
\end{enumerate}

\definition[geometric series]
We call a series a \under{geometric series} if it is of the form
$$\sum_{k=1}^\infty ar^k$$
Note that the geometric series converges to $\frac{a}{1-r}$ whenever $r^m \rightarrow 0$ iff $|r| < 1$.

\definition[absolutely convergence]
$\sum_{k=1}^\infty a_k$ is \under{absolutely convergent} if $\sum_{k=1}^\infty |a_k| < \infty$. 

 \definition[conditionally convergence]
$\sum_{k=1}^\infty a_k$ is \under{conditionally convergent} if $\sum_{k=1}^\infty a_k < \infty$, but $\sum_{k=1}^\infty |a_k| = \infty$

\example[alternating series]
$\sum_{k=1}^\infty \frac{(-1)^{k+1}}{k} < \infty$ but $\sum_{k=1}^\infty \bignorm{\frac{(-1)^{k+1}}{k}} = \sum_{k=1}^\infty \frac{1}{k} = \infty$

\property[Absolute convergence test]
If $\sum_{k=1}^\infty |a_k| < \infty$, then $\sum_{k=1}^\infty a_k < \infty$.

\begin{proof}
	We use Cauchy criterion for $\sum_{k=1}^\infty a_k$: we want to show
	$$\forall \epsilon > 0, \exists N \text{ s.t. } \forall n \geq m\geq N, \bignorm{\sum_{k = m+1}^n a_k} < \epsilon$$
	Let $\epsilon > 0$.\\
	Since $\sum_{k=1}^\infty |a_k| < \infty$, then we know that $\exists N$ s.t. $\forall n \geq m \geq N$,
	$$\bignorm{\sum_{k=1}^n|a_k| - \sum_{k=1}^m|a_k|} < \epsilon$$
	Then
	\begin{align*}
		\bignorm{\sum_{k = m+1}^n a_k} &= \bignorm{\sum_{k=1}^na_k - \sum_{k=1}^ma_k}\\
		&\leq \sum_{k=1}^n|a_k| - \sum_{k=1}^m|a_k| \\
		&\leq \bignorm{\sum_{k=1}^n|a_k| - \sum_{k=1}^m|a_k|}\\
		&< \epsilon
	\end{align*}
\end{proof}

\property[Alternating series test] Suppose $a_1 \geq a_2 \geq \hdots \geq 0$, $\underset{k \rightarrow \infty}{\lim}a_k = 0$, then $\sum_{k=1}^\infty (-1)^{k+1} a_k < \infty$.

\begin{proof}
	We want to show $\{S_n\} = \{\sum_{k=1}^n (-1)^{k+1}a_k\}$ is Cauchy:
	$$\forall \epsilon > 0, \exists N \text{ s.t. } \forall m, n \geq N, |S_n - S_m| < \epsilon$$
	Let $\epsilon > 0$.\\
	Suppose $n > m$, then $|S_n - S_m| = |a_{m+1} - a_{m+2} + \hdots + (-1)^{n-m+1}a_n|$. \\
	Since $(a_n)$ is a non-negative decreasing sequence, then
	\begin{align*}
		a_{m+1} - a_{m+2} + \hdots + (-1)^{n-m-1}a_n &= a_{m+1} - (a_{m+2} - a_{m+3}) - (a_{m+4} - a_{m+5}) - \hdots \\
		&\leq a_{m+1}
	\end{align*}
	Since $\underset{k \rightarrow \infty}{\lim}a_k = 0, \exists N$ s.t. $\forall m + 1 \geq N, a_{m+1} < \epsilon$. \\
	Thus $0 \leq |S_n - S_m| \leq a_{m+1} < \epsilon$.	
	
\end{proof}

\property[Ratio test] Given $\sum_{k=1}^\infty a_k$ s.t. $a_k \neq 0$ for all $k$. \\
If $\limit \bignorm{\frac{a_{n+1}}{a_n}} = r < 1$, then $\sum_{k=1}^\infty |a_k|< \infty$ 
\begin{proof}
	Define $S := \{ n\in \mb{N}: \bignorm{\frac{a_{n+1}}{a_n}} \geq r'\}$, then S contains finitely many elements of $\mb{N}$. (If $S$ were to be infinite set, if we take $\epsilon = r' - r$, then $\bignorm{\frac{a_{n+1}}{a_n}} - r \geq r' - r$ for infinitely many terms which contradicts that $r$ is the point of convergence.\\
	Therefore, $S' = \{n \in \mb{N}: \bignorm{\frac{a_{n+1}}{a_n} < r'}$ contains all but finitely many elements of $\mb{N}$. Let $N = 1 + \max S$, then $\forall n \geq N,\bignorm{\frac{a_{n+1}}{a_n} < r'} < r' \implies |a_{n+1}| < r'|a_n|$. \\
	Since $0 < r' < 1, \sum_{n=1}^\infty (r')^n$ converges which implies $|a_N|\sum_{n=1}^\infty (r')^n$ converges. 
	We have $\sum_{n=1}^\infty |a_n| = \sum_{n=1}^N |a_n| + \sum_{n = N+1}^\infty |a_n| < C + |a_N|\sum_{n=N+1}^\infty (r')^{n-N}$ converges, by comparison test. Hence $\sum_{n=1}^\infty |a_n|$ converges.
\end{proof}
\todo{understand the last two lines of the proof}

\definition[rearrangement] Let $\sum_{k=1}^\infty a_k$ be a series. A series $\sum_{k=1}^\infty b_k$ is called a \under{rearrangement} of $\sum_{k=1}^\infty a_k$ if $\forall n, !\exists k$ s.t. $b_k = a_n$.


\section{Metric Spaces and the Baire Category Theorem}
\subsection{Basic Definitions}
\definition[metric and metric space] Given a set $X$, a function $d: X \times X \rightarrow \real$ is a \under{metric} on $X$ if for all $x, y \in X$:
\begin{enumerate}
	\item $d(x,y) \geq 0$ with $d(x,y) = 0$ if and only if $x = y$;
	\item $d(x,y) = d(y,x)$;
	\item for all $z \in X, d(x, y) \leq d(x,z) + d(z,y)$
\end{enumerate}
A \under{metric space} is a set $X$ together with a metric $d$.

\example
The set $\real$ considered with $d: \real^2 \rightarrow [0, \infty), (x,y) \mapsto |x - y|$ is a metric space.

\example
In general, $\real^n$ considered with the Euclidean distance is a metric space.
$$d(\vx, \vy) = \sqrt{\sum_{i=1}^n (x_i - y_i)^2}$$

\example
Let $x$ be a set. The \under{discrete metric} $d$ on $X$ is defined by
$$d(x,y) = \begin{cases}
	0 \quad x = y \\
	1 \quad x \neq y
\end{cases}$$

\paragraph{Fact} If $(X, d)$ is a metric space, $d'(x,y) = \max\{1, d(x,y)\}$ for all $x, y \in X$, then $(X, d')$ is also a metric space.

\example
Let $X = \{f: A \rightarrow \real\}$ \\
$$d(f,g) = \sup\{|f(x) - g(x)|: x \in A\}$$ if the supremum exists.

\definition
Let $(X, d_1)$ and $(Y, d_2)$ be metric spaces. A function $f: X \rightarrow Y$ is \under{continuous at $x \in X$} if $\forall \epsilon > 0, \exists \delta > 0, d_1(x,y) < \delta \implies d_2(f(x), f(y)) < \epsilon$.

\subsection{Topology on Metric Spaces}

%\remark
%$(X, \tau), \tau \subseteq P(X), E \subseteq X$
%\begin{enumerate}
%	\item $\bar{E}$ = minimal closed superset of $E$ = $\bigcap\{H: H \text{ closed }, H \supseteq E\}$
%	\item $E^\circ$ = maximal open subset of $E$ = $\bigcup\{U: U \text{ open }, U \subseteq E\}$
%\end{enumerate}
%
%\example $(X, d)$ is a metric space, $\tau_d$ is the topology determined by $d$: $U \in \tau_d$ iff $\forall x \in U, \exists \epsilon > 0, B_\epsilon(x) \subseteq U$ \\
%$ F \subseteq X$
%
%\begin{align}
%	\bar{F} &= \{ x \in X: \forall \epsilon > 0, B_\epsilon(x) \cap F \neq \emptyset\} \\
%	&= \bigcap \{ H: H \text{ closed } H \supseteq F\} \\
%	&= \{ \underset{n \rightarrow \infty}{\lim} x_n:  (x_n) \subseteq F, \underset{n \rightarrow \infty}{\lim} x_n \text{ exists } \}
%\end{align}

%TODO: I really, really cannot understand professor's proof...



%\begin{proof}
%\tb{Claim 1: } $$\bar{F} = \{ x \in X: \forall \epsilon > 0, B_\epsilon(x) \cap F \neq \emptyset\} := H$$
%%Note that $H$ is closed and $\supseteq F$ \\
%%Then $U = X \setminus H$ is open. Take $y \in X \setminus H$. \\
%%We need to find $\epsilon > 0$ s.t. $B_\epsilon(y) \subseteq X \setminus H$\\
%%Note that for a given $\epsilon > 0$, $$B_\epsilon(y) \cap F \neq \emptyset \equiv B_\epsilon(y) \cap \bar{F} \neq \emptyset$$
%%Note that $X \setminus B_\epsilon(y) \supseteq F$ is a closed superset of $F$.\\
%Suppose for contradiction that $H$ is not the minimal superset of $F$, so there is a closed set $H \subsetneq H$ s.t. $H' \supseteq F$. \\ 
%Notice that $V = X \setminus H'$ is open. \\
%Then take $y \in H \setminus H' \subseteq V$. So there is $\epsilon > 0, B_\epsilon(y) \subseteq V$. But then $B_\epsilon(y) \cap H' = \emptyset, B_\epsilon(y) \cap F = \emptyset, y \notin H \implies $ contradiction.
%
%
%\end{proof}

%
%\begin{align}
%	F^\circ &= \{ x \in X: \exists \epsilon > 0, B_\epsilon(x) \cap F \neq \emptyset\} \\
%	&= \bigcup \{B_\epsilon(x): \epsilon > 0, x \in F, B_\epsilon(x) \subseteq F \} \\
%\end{align}



\definition[open ball]
An \under{open ball} (or \under{$\epsilon$-neighbourhood}) with radius $r$ and center $x$ is
$$B_r(x) = \{y \in X: d(x, y) < r\}$$

\definition[open set]
A set $U \subseteq X$ is \under{open} iff
$$\forall x \in U, \exists \epsilon > 0 \text{ s.t. } B_\epsilon(x) \subseteq U$$

\example $B_\epsilon(x)$ is open.
\begin{proof}
	Fix $x \in X$ and $\epsilon > 0$. We want to show: $\forall y \in B_\epsilon(x), \exists \delta > 0$ s.t. $B_\delta(y) \subseteq B_\epsilon(x)$. \\
	Take $y \in B_\epsilon(x)$, then $d(x,y) < \epsilon$. Take $\delta = \epsilon - d(x,y) > 0$. Take any $z \in B_\delta(y)$, we have
	$$d(x,z) \leq d(x,y) + d(y,z) < d(x,y) + \epsilon - d(x,y) = \epsilon$$
	Thus $z \in B_\epsilon(x)$ so $B_\delta(y) \subseteq B_\epsilon(x)$.
\end{proof}
\definition[topological space] A \under{topological space} is a pair $(X, \tau)$, where $X$ is a set and $\tau$ a subset of the power set of $X$ which we call open such that
\begin{enumerate}
	\item $\emptyset, X \in \tau$
	\item $U_1, \hdots, U_n \in \tau \implies \bigcap_{i=1}^n U_i \in \tau$
	\item $U_1, \hdots, U_n \in \tau \implies \bigcup_{i=1}^n U_i \in \tau$
\end{enumerate}

\example
$(X, \{\emptyset, X \})$

\example
$(X, P(X))$ is a \under{discrete topological space}, where $P(X)$ is the power set of $X$.
\example
Given $(X, d)$ a metric space, define $\tau_d: $ a set $U \in \tau_d \iff \forall x \in U, \exists \epsilon > 0, B_\epsilon(x) \subseteq U$. Then $\tau_d$ is a topology.
\begin{proof}
	(1) First, $\emptyset, X \in \tau_d$ since $\forall x \in \emptyset, B_1(x) \subseteq \emptyset$ and $\forall x \in X, B_1(x) \subseteq X$.\\
	Then suppose $U_1, \hdots, U_n \in \tau_d$.\\
	(2) we want to show:
	$$U = \bigcap_{i=1}^n U_i \in \tau_d \iff \forall x \in U, \exists \epsilon > 0 \text{ s.t. } B_\epsilon(x) \subseteq U$$
	Since $x \in U$, then $\forall i = 1, \hdots, n, x \in U_i: \exists \epsilon_i > 0$ s.t. $B_{\epsilon_i}(x) \subseteq U_i$. \\
	Take $\epsilon = \underset{1 \leq i \leq n}{\min} \epsilon_i$, thus $B_\epsilon(x) \subseteq U_i \, \forall i$. Hence $B_\epsilon(x) \subseteq U_i \subseteq U$. \\
	(3) We also want to show:
	$$\bigcup_{i=1}^n U_i \in \tau_d \iff \forall x \in U, \exists \epsilon > 0 \text{ s.t. } B_\epsilon(x) \subseteq U$$
	Let $x \in U$, then there is some $U_i$ s.t. $x \in U_i$. Since $U_i \in \tau_d$, then $\exists \epsilon >0$ s.t. $B_\epsilon(x) \subseteq U_i \subseteq U$.\\
	Therefore, $\tau_d$ is a topology.
\end{proof}

\definition
A \blue{subset} $F$ of a topological space $(X, \tau)$ is \under{closed} if $X \setminus F$ is open. \\
\property Given a topological space $(X, \tau)$ and a subset $F$ of it, we have: 
\begin{enumerate}
	\item $\emptyset, X$ are closed
	\item If $F_1, \hdots, F_n$ are closed, then $\bigcup_{i=1}^n F_i$ is closed
	\item If $F_1, \hdots, F_n$ are closed, then $\bigcap_{i=1}^n F_i$ is closed
\end{enumerate}

\definition[topological closure and interior]
Given a topological space $(X, \tau)$, where $\tau \subseteq P(X)$, and a set $F \subseteq X$, the \under{topological closure} of $F$ is the minimal closed superset of $F$, i.e.,
$$\bar{F} = \bigcap \{H: H \text{ is closed}, H \supseteq F\}$$
The \under{interior} of $F$ is the maximal open subset of $F$, i.e.,
$$F^\circ = \bigcap \{U: U \text{ is open}, U \subseteq F\}$$

\example
Given $(X, d)$ a metric space, define $\tau_d: $ a set $U \in \tau_d \iff \forall x \in U, \exists \epsilon > 0, B_\epsilon(x) \subseteq U$. Suppose $F \subseteq X$, then
$$\bar{F} = \{x \in X: \forall \epsilon > 0, B_\epsilon(x) \cap F \neq \emptyset\} = \{\limit x_n: (x_n) \subseteq F, \limit x_n \text{ exists}\}$$
and
$$F^\circ = \{x \in X: \exists\epsilon >0, B_\epsilon(x)\subseteq F\} = \bigcup \{B_\epsilon(x): \epsilon >0, x \in F, B_\epsilon(x) \subseteq F\}$$

\subsection{Compactness and Bolzano-Weierstrass Theorem}
\definition[compactness]
A subset $K$ of a metric space $(X, d)$ is \under{compact} if every sequence in $K$ has a convergent subsequence that converges to a limit in $K$.

\example
$(\real, |x - y|)$ is not compact (e.g. $(x_n) = n$)

\example
$([0,1], |x-y|)$ is compact.

\property If $(X, d)$ is compact, then it is bounded, i.e. $\exists M$ s.t. $x, y \in X, d(x,y) \leq M$.
\property If $Y \subseteq X, (X, d)$ is a metric space, and $(Y, d)$ is compact, then $Y$ is closed in $X$.

\property If $K_1 \supseteq K_2 \supseteq \hdots$ are compact and nonempty subsets of $X$, then $K = \bigcap_{n=1}^\infty K_n$ is compact and nonempty.

\theorem[Bolzano-Weierstrass theorem] A subset $Y$ of $\real$ is \blue{compact} iff \blue{closed and bounded}. \\
\tb{Alternative formation: } Every \blue{bounded} subsequence contains a \blue{convergent subsequence.}
\remark The theorem is true for $\real^n$ but is false for infinite dimension.

\theorem[Heine-Borel Theorem]
Let $K$ be a subset of a metric space $(X, d)$. The following statements are equivalent:
\begin{enumerate}
	\item $K$ is compact.
	\item $K$ is closed and bounded.
	\item \blue{Every open cover $K \subseteq \bigcup_{i\in I} U_i$ for $K$ has a finite subcover $K \subseteq \bigcup_{l=1}^n U_{i_l}$.}

\end{enumerate}

\subsection{Completeness of Metric Spaces}
\definition[completeness of metric spaces]
A metric space $(X, d)$ is \under{complete} if every Cauchy sequence in $X$ converges to an element of $X$.

\example
$\real, d(x,y) = |x - y|$
\example
$(X, d), d$ discrete metric.
\example 
$C[0,1], d(f,g) = \underset{x \in [0, 1]}{\sup} |f(x) - g(x)| = ||f - g||_{\infty}$
\example
$(\mb{N}^{\mb{N}}, d), d((x_n), (y_n)) = \frac{1}{\min\{n: x_n \neq y_n\}}$
\\
where $\mb{N}^{\mb{N}} = \{x: \mb{N} \rightarrow \mb{N}\}$.


\subsection{Perfect Sets}
\definition[perfect set]
Let $(X,d)$ be a metric space. $P \subseteq X$ is \under{perfect} if it is closed, nonempty, and for every open $U \subseteq X, U \cap P$ is not empty and has at least two elements.

\example
$S = [0,1] \cup \{\frac{3}{2}\} \cup [2,3]$ is not perfect.

\property Perfect subsets $P$ of a complete metric space are not countable.

\example[Cantor set] Let $C_0$ be the closed interval $[0,1]$, and define $C_1$ to be the set that results when the open middle third is removed; that is,
$$C_1 = C_0 \setminus (\frac{1}{3}, \frac{2}{3}) = [0, \frac{1}{3}] \cup [\frac{2}{3}, 1]$$
Now construct $C_2$ in a similar way by removing the open middle third of each of the two components of $C_1$:
$$C_2 = ([0, \frac{1}{9}] \cup [\frac{2}{9}, \frac{1}{3}]) \cup ([\frac{2}{3}, \frac{7}{9}] \cup [\frac{8}{9}, 1])$$
Continue this process inductively. For each $n = 0,1,2,\hdots$, we get a set $C_n$ consisting of $2^n$ closed intervals each having length $(\frac{1}{3})^n$. Finally, we define the \under{Cantor set} C to be the intersection
$$C = \bigcap_{n=0}^\infty C_n$$

\remark As follows
\begin{itemize}
	\item Since we are always removing open middle thirds, then at each stage, endpoints are never removed. Thus, C at least contains the endpoints of all of the intervals that make up each of the sets $C_n$.
	\item The Cantor set has zero length.
	\item The Cantor set is uncountable, with cardinality equal to the cardinality of $\real$.
\end{itemize}

\subsection{Separated and Connected Sets}
\definition[separated sets]
Let $(X,d)$ be a metric space, $A \neq \emptyset, B \subseteq X$. $A$ and $B$ are \under{separated} if $\bar{A} \cap B = \bar{B} \cap A = \emptyset$.

\definition[connected sets]
A set $C \subseteq X$ is \under{connected} if for every decomposition $C = A \cup B$ s.t. $A, B \neq \emptyset$, $A$ and $B$ are not separated, i.e. $\bar{A} \cap B \neq \emptyset$ or $\bar{B} \cap A \neq \emptyset$.

\property
$C \subseteq \real$ is connected iff $$\forall a,b \in C, [a,b] \subseteq C$$
\begin{proof}
	Let $C = A \cup B, a_0 \in A, b_0 \in B, a_0 < b_0$. We define $I_0 = [a_0, b_0], c_0 = \frac{a_0 + b_0}{2}$. Define $I_1 = [a_0, c_0], \hdots$. We
	have $x \in \bar{A} \cap B$ or $\bar{B} \cap A$. 
	\todo{Is this complete?} 
\end{proof}


\subsection{Baire's Theorem}

\definition[dense]
A set $A \subseteq X$ is \under{dense} in the metric space $(X, d)$ if $\bar{A} = X$. 

\definition[nowhere-dense]A subset $E$ of a metric space $(X, d)$ is \under{nowhere-dense} in $X$ if $\bar{E}^\circ$ is empty.\\
i.e., A nowhere-dense set of a metric space is a set whose closure has empty interior.

\remark
It is a set whose elements are not tightly clustered anywhere.

\example
$\mb{Z}$ is nowhere-dense in $\real$.

\example
$S = \{\frac{1}{n}: n \in \mb{N}\}$ is nowhere-dense in $\real$.\\
$\bar{S} = S \cup \{0\}$, which has empty interior.


\theorem[Baire's Theorem]
The set of real numbers $\real$ cannot be written as the countable union of nowhere-dense sets.

\remark
Baire's Theorem asserts that the only way to make $\real$ from a countable union of arbitrary sets is for the closure of at least one of these sets to contain an interval.

\subsection{The Baire Category Theorem}
\theorem Let $(X, d)$ be a complete metric space, and let $\{O_n\}$ be a countable collection of dense, open subsets of $X$. Then, $\bigcap_{n=1}^\infty\{O_n\}$ is not empty.
\todo{prove this}
\theorem[Baire Category Theorem]
A complete metric space cannot be written as the countable union of nowhere-dense sets.
\todo{prove this}

\remark
This result is called the Baire Category Theorem because it creates two categories of size for subsets in a metric space:
\begin{enumerate}
	\item A set of ``first category" is one that can be written as a countable union of nowhere-dense sets. These are the small, intuitively ``thin" subsets of a metric space.
	\item If our metric space is complete, then it is necessarily of ``second category", meaning it cannot be written as a countable union of nowhere-dense sets.
\end{enumerate}

\theorem
The set
$$D = \{f \in C[0,1]: f'(x) \text{ exists for some } x \in [0, 1]\}$$ is a set of first category in $C[0,1]$.



\section{Functional Limits and Continuity}
\subsection{Functional Limits}
\definition Let $A \subseteq \real, a \in \overline{A \setminus \{a\}}$ ($a$ is an accumulation point of $A$). Let $f: A \rightarrow \real$, define $\underset{x \rightarrow a}{\lim} f(x) = L$ iff
$$\forall \epsilon > 0, \exists \delta > 0 \text{ s.t. } 0 < |x - a| < \delta \implies |f(x) - L| < \epsilon$$

\property[Sequential criterion for functional limits]
$a \in \overline{A \setminus \{a\}}, f:A \rightarrow \real$. The following are equivalent:
\begin{enumerate}
	\item $\underset{x \rightarrow a}{\lim} f(x) = L$
	\item $\forall (x_n) \subseteq A \setminus \{a\}, x_n \rightarrow a \implies f(x_n) \rightarrow L$
\end{enumerate}
\begin{proof}
	We prove $(1) \implies (2)$: \\
	Assume $\underset{x \rightarrow a}{\lim} f(x) = L$, take arbitrary $(x_n) \subseteq A \setminus \{a\}$ s.t. $x_n \rightarrow a$. \\
	Let $\epsilon > 0$, then $\exists \delta > 0$ s.t. $0 < |x - a| < \delta \implies |f(x) - L| < \epsilon$. \\
	Also, $\exists N$ s.t. $n \geq N \implies |x_n - a| < \delta$. \\
	Therefore, if $|x_n - a| < \delta$, then $|f(x_n) - L| < \epsilon$. 
\end{proof}

\theorem[Algebraic Limit Theorem for functional limits]
Suppose $f, g: A \rightarrow \real, a \in \overline{A \setminus \{a\}}$.\\
Suppose $\underset{x \rightarrow a}{\lim} f(x) = L, \underset{x \rightarrow a}{\lim} g(x) = M$. Then we have
\begin{enumerate}
	\item $\underset{x \rightarrow a}{\lim} cf(x) = cL$
	\item $\underset{x \rightarrow a}{\lim} (f(x) + g(x)) = L + M$
	\item $\underset{x \rightarrow a}{\lim} (f(x)g(x)) = LM$
	\item $\underset{x \rightarrow a}{\lim} (\frac{f(x)}{g(x)}) = \frac{L}{M}$ when $M \neq 0$.
\end{enumerate}

\property[Divergence criterion] Suppose $f: A \rightarrow \real, a \in \overline{A \setminus \{a\}}$ $\underset{x \rightarrow a}{\lim} f(x)$ does not exist if there are two sequences $(x_n), (y_n) \subseteq A \setminus \{a\}$ s.t. $x_n \rightarrow a, y_n \rightarrow a, \limit f(x_n) = L, \limit f(y_n) = M$ exist but $L \neq M$.

\example
Let $A = \real^+, f(x) = \sin(\frac{1}{x})$. Let $a_n = \frac{1}{2n\pi}, b_n = \frac{1}{2n\pi + \frac{\pi}{2}}$. \\
Then we have $a_n, b_n \rightarrow 0$. Besides, $\limit f(a_n) = 0, \limit f(b_n) = 1$. Hence $\underset{x \rightarrow 0^+}{\lim} \sin(\frac{1}{x})$ does not exist.

\definition
Suppose $f: A \rightarrow \real, x \in A \setminus \{a\}$. We define $\underset{x \rightarrow a}{\lim} f(x) = \infty$ iff
$$\forall M > 0, \exists \delta > 0 \text{ s.t. } 0 < |x - a| < \delta \implies f(x) > M$$


\subsection{Continuous Functions}
\definition[continuity]
 Suppose $(X, d_X), (Y, d_Y)$ are metric spaces. $f: X \rightarrow Y$ is \under{continuous} at $a \in X$ if 
 $$\forall \epsilon > 0, \exists \delta > 0 \text{ s.t. } x \in B_{\delta}^X(a) \implies f(x) \in B_{\epsilon}^Y(f(a))$$
 \remark
 Note that for $X = Y = \real, d(x,y) = |x - y|$, so that we can write
 $$\forall \epsilon > 0, \exists \delta > 0 \text{ s.t. } |x - a| < \delta \implies |f(x) - f(a)| < \epsilon$$
 i.e. $$\underset{x \rightarrow a}{\lim} f(x) = f(a)$$

\definition[continuous function]
$f: X \rightarrow Y$ is \under{continuous} if it is continuous at every point $a \in X$.

\property
The following are equivalent:
\begin{enumerate}
	\item $f$ is continuous at $a$
	\item $\underset{x \rightarrow a}{\lim} f(x) = f(a)$
	\item $\forall (x_n) \subseteq A, x_n \rightarrow a \implies f(x_n) \rightarrow f(a)$.
\end{enumerate}

\corollary $f$ is discontinuous at $a$ if there is a sequence $(x_n) \rightarrow a$ s.t. $\limit f(x_n) \neq f(a)$.

\remark
\red{Note that we may have $\underset{x \rightarrow a}{\lim} f(x)$ exists but $f$ is discontinuous at $a$.}

\theorem[Algebraic Continuity Theorem] Suppose $f, g: A \rightarrow \real$ are continuous at $a \in A, c \in \real$. We have
\begin{enumerate}
	\item $cf(x)$ is continuous at $a$
	\item $f(x) \pm g(x)$ is continuous at $a$
	\item $f(x)g(x)$ is continuous at $a$
	\item $\frac{f(x)}{g(x)}$ is continuous at $a$ if $g(a) \neq 0$
\end{enumerate}

\theorem Suppose $f: A \rightarrow B \subseteq \real, g: B \rightarrow \real$. \\
$(g \circ f)(x) = g(f(x))$ is continuous at $a \in A$ whenever $f$ is continuous at $a$ and $g$ is continuous at $f(a)$.

\theorem Suppose $(X, d_X), (Y, d_Y)$ are metric spaces and $f: X \rightarrow Y$ is continuous. \\
If $K \subseteq X$ is compact, then its image $f[K] = \{f(x): x \in K\}$ is compact.

\theorem Suppose $(X, d_X), (Y, d_Y)$ are metric spaces. If $F \subseteq Y$ is closed in $Y$, then $f^{-1}(F)$ is closed in $X$.

\theorem[Extreme Value Theorem]
If $f: K \rightarrow \real$ is continuous, $K$ is compact, then $\exists x_1, x_2 \in K$ s.t. $\forall x \in K$,
$$f(x_1) \leq f(x) \leq f(x_2)$$
\begin{proof}
	Let $H = f[K] = \{f(x): x \in K\} \subseteq \real$, which is compact. Since compact subsets of $\real$ are bounded, then let $y_2 = \sup(H)$. \\
	We have $y \leq y_2$ for all $y \in H$ and $\forall \epsilon > 0, \exists y \in H$ s.t. $y_2 - \epsilon < y \leq y_2$.\\
	Take $\epsilon = \frac{1}{n}$, then we have some $z_n \in H$ s.t. $y_2 - \frac{1}{n} < z_n \leq y_2$. \\as
	Now we find $a_n \in k$ s.t. $f(a_n) = z_n, n = 1,2,\hdots$ \\
	\red{By theorem, we have $a_{n_k} \rightarrow x_2$, then $f(x_2) = \underset{k \rightarrow \infty}{\lim}f(a_{n_k}) = y_2$.}
	\todo{Which theorem?}
\end{proof}

\subsection{Continuous Functions on Compact Sets}
\subsubsection{Uniform Continuity}
\definition[uniform continuity]
We say function $f: A \rightarrow \real$ is \under{uniformly continuous} on $A$ if 
$$\forall \epsilon > 0, \exists \delta > 0, x, y \in A \land |x - y| < \delta \implies |f(x) - f(y)| < \epsilon$$

\example $f(x) = x^2$ is not uniformly continuous.
\begin{proof}
	WTS $\exists \epsilon > 0, \forall \delta > 0, \exists x, y \in \real$ s.t. $|x - y| < \delta $ and $|f(x) - f(y)|  \geq \epsilon$.\\
	Let $\epsilon = 1, \delta > 0$. \\
	Choose $y = x + \frac{1}{2}\delta$, so that $|x - y| < \delta$. \\
	\begin{align*}
		f(y) - f(x) &= y^2 - x^2 \\
		&= (x + \frac{1}{2}\delta)^2 - x^2 \\
		&= x^2 + \delta x + \frac{1}{4}\delta^2 - x^2 \\
		&= \delta x + \frac{1}{4}\delta^2
	\end{align*} 
	If $x > \frac{1}{\delta}$, then $f(y) - f(x) > 1$.
\end{proof}

\property[\label{fails to be uniformly continuous iff}]
Function  $f: A \rightarrow \real$ fails to be uniformly continuous iff 
$\exists \epsilon_0 > 0, \exists (x_n), (y_n) \subseteq A$ s.t. $\limit |x_n - y_n| = 0 \land \forall n, |f(x_n) - f(y_n)| \geq \epsilon_0$.
\begin{proof} 
($\Leftarrow$) Obvious.\\
($\Rightarrow$) Assume $f$ is not uniformly continuous. \\
Then $\exists \epsilon_0 > 0$ s.t. $\forall \delta > 0, \exists x_n, y_n \in \real$ s.t. $|x_n - y_n| < \delta$ and $|f(x_n) - f(y_n)| \geq \epsilon_0$.\\
Then this is true for $\delta \in \mb{N}$ as well. \\
For each $n \in \mb{N}$, let $\delta = \frac{1}{n}$, and pick $x_n, y_n$ as above. Then it is obvious that $\limit |x_n - y_n| = 0$ and $\forall n, |f(x_n) - f(y_n)| \geq \epsilon_0$.
\end{proof}

\property[Continuous functions on compact sets are uniformly continuous] Assume $f: K \rightarrow \real$ is continuous and $K$ is compact, then $f$ is uniformly continuous on $K$.
\begin{proof}
	Assume for a contradiction that $f: K \rightarrow \real$ is continuous and $K$ is compact, but $f$ is not uniformly continuous. Then by Property \ref{fails to be uniformly continuous iff}, $\exists \epsilon_0 > 0, (x_n), (y_n) \subseteq K$ s.t. $\limit |x_n - y_n| = 0$ and $\forall n, |f(x_n) - f(y_n)| \geq \epsilon_0$. \\ 
	Since $K$ is compact, then $(x_n)$ has a subsequence $(x_{n_k})$ s.t. $x_{n_k} \rightarrow x \in K$.\\
	Moreover, $(y_{n_k})$ has a subsequence $(y_{n_{k_m}})$ s.t. $y_{n_{k_m}} \rightarrow y \in K$.\\
	Let $x_m' = x_{n_{k_m}}, y_m' = y_{n_{k_m}}$, then $x_m' \rightarrow x, y_m' \rightarrow y$.\\
	Since $\underset{m \rightarrow \infty}{\lim} |x_m' - y_m'| = 0$, thus $x = y$.\\
	Then
	\begin{align*}
		|f(x_m') - f(y_m')| & \geq \epsilon_0 \\
		\implies \underset{m \rightarrow \infty}{\lim}|f(x_m') - f(y_m')| & \geq \epsilon_0 \\
		\implies |f(x) - f(y)| &\geq \epsilon_0 \\
		\implies 0 &\geq \epsilon_0
	\end{align*}
	which is a contradiction.
\end{proof}

\definition A function $f: A \rightarrow \real$ is said to be \under{Lipschitz} if $\exists M \in \mb{N}$ s.t. $\forall x \neq y \in A$,
$$\bignorm{\frac{f(x) - f(y)}{x - y}} < M$$

\property Lipschitz functions are uniformly continuous.
\begin{proof}
	Let $f: A \rightarrow \real$ be Lipschitz on $A$. Then for every $\epsilon > 0$, take $\delta < \frac{\epsilon}{M}$. \\
	Then if $|x - y| < \delta$, then 
	\begin{align*}
		|f(x) - f(y)| &< M|x - y| \\
		&< M\frac{\epsilon}{M}\\
		&= \epsilon
	\end{align*}
	So $f$ is uniformly continuous.
\end{proof}
\remark
The converse does not hold.
\property[Continuous image of connected sets is connected] If $f: E \rightarrow \real$ is continuous and $E$ is connected, then $f(E)$ is connected.
%\begin{proof}
%	Assume for a contradiction that $f(E)$ is not connected.\\
%	Fix $A, B \subseteq f(E)$ s.t. $\bar{A} \cap B = \emptyset = \bar{B} \cap A$ and $f(E) = A \cup B$.\\
%	Let $C = f^{-1}(A), D = f^{-1}(B)$. Note that $C \cup D = E$ because $f$ is a function.
%\end{proof}

\subsection{Sets of Discontinuity}
Let $f: \real \rightarrow \real, D_f = \{x \in \real: f \text{ is not continuous at }x\}$.
\example[$D_f = \emptyset$] $f$ is continuous
\example[$D_f = \real$] $f(x) = \begin{cases}
	1, & x\in \mb{Q}\\
	0, & x \in \real \setminus \mb{Q}
\end{cases} $
\example Given a countable set $A = \{a_1, \hdots\}$, define $f(a_n) := \frac{1}{n}$ and $f(x) = 0, \forall x \notin A$. Then we have $D_f = A$.
\fact There is no $f: \real \rightarrow \real$ s.t. $D_f = \real \setminus \mb{Q}$.

\definition[$F_\sigma$-set] A subset $F$ of $\real$ is a \under{$F_\sigma$-set} if $F = \bigcup_{n=1}^\infty F_n$ s.t. $F_n$ is closed for all $n$.

\definition[$\alpha$-continuity] Let $\alpha > 0, f:\real \rightarrow \real, a \in \real$. $f$ is \under{$\alpha$-continuous} at $a$ if
$$\exists \delta > 0 \text{ s.t. } x, y \in (a - \delta, a + \delta) \implies |f(x) - f(y)| < \alpha$$
Note that $f$ is continuous at $a$ iff $f$ is $\alpha$-continuous at a for all $a > 0$.

\property For every $f: \real \rightarrow \real$, \red{the set $D_f$ is $F_\sigma$-set of $\real$.}
\unsure{red parts}

\definition Let $f: \real \rightarrow \real$. \\
$f$ is \under{removable discontinuous} if $\underset{x \rightarrow a}{\lim} f(x)$ exists but does not equal $f(a)$.\\
$f$ has a \under{jump} at $a$ if $\underset{x \rightarrow a^-}{\lim} f(x) \neq \underset{x \rightarrow a^+}{\lim} f(x)$. \\
If $\underset{x \rightarrow a}{\lim} f(x)$ does not exist for other reasons, we say $f$ is essential discontinuous.
\definition[monotonicity] $f: \real \rightarrow \real$ is \under{monotone} if either $x \leq y \implies f(x) \leq f(y)$ or $x \leq y \implies f(x) \geq f(y)$.

\property
Discontinuity of a monotone function $f$ is a jump. Moreover, $D_f$ is countable.

\section{the Derivative}
\subsection{Derivatives and the Intermediate Value Property}
\definition[derivative]
Let $f: \real \rightarrow \real, c \in \real$. Define the \under{derivative} of $f$ at $c$:
$$f'(c) = \underset{x \rightarrow c}{\lim} \frac{f(x) - f(c)}{x - c}$$
If $f'(c)$ exists, we say that $f$ is \under{differentiable at $c$}. If $f'(a)$ exists for all $a \in \real$, we say that $g$ is \under{differentiable on $\real$}.

\property If $f$ is differentiable at $c$, then $f$ is continuous at $c$.
\begin{proof}
	We have
	$$\underset{x \rightarrow c}{\lim} (f(x) - f(c)) = \underset{x \rightarrow c}{\lim} \frac{f(x) - f(c)}{x - c} \cdot (x - c) = f'(c)\cdot 0 = 0$$
\end{proof}

\theorem[Algebraic Differentiability Theorem] Suppose $f, g$ are differentiable, $a,c \in \real$. We have
\begin{enumerate}
	\item $(cf)'(a) = cf'(a)$
	\item $(f + g)'(a) = f'(a) + g'(a)$
	\item $(f \cdot g)'(a) = f'(a)g(a) + f(a)g'(a)$
	\item $\left( \frac{f}{g} \right)'(a) = \frac{f'(a)g(a) - f(a)g'(a)}{[g(a)]^2}$
\end{enumerate}

\theorem[Chain Rule] Let $f: A \rightarrow B, g: B \rightarrow \real, f(A) \subseteq B$ so that $g \circ f$ is defined. If $f$ is differentiable at $c$ and $g$ is differentiable at $f(c)$, then $g \circ f$ is differentiable at $a$ with 
	$$(g \circ f)'(c) = g'(f(c)) \cdot f'(c)$$

\theorem[Interior Extremum Theorem] If $f$ is differentiable on $(a,b)$, $f$ attains maximum at some $c \in (a,b)$, then $f'(c) = 0$.
\begin{proof}
	We have
	$$f'(c) = \underset{x \rightarrow c^-}{\lim} \frac{f(x) - f(c)}{x-c} \leq 0$$
	and 
	$$f'(c) = \underset{x \rightarrow c^+}{\lim} \frac{f(x) - f(c)}{x-c} \geq 0$$
	then $f'(c) = 0$.
\end{proof}

\theorem[Darboux's Theorem] If $f$ is differentiable on $[a,b]$ and $f'(a) < \alpha < f'(b)$ or $f'(a) > \alpha > f'(b)$, then $\exists c \in (a,b)$ s.t. $f'(c) = \alpha$.

\subsection{the Mean Value Theorems}
\theorem[Rolle's Theorem] Let $f:[a,b] \rightarrow \real$ be continuous on $[a,b]$ and differentiable on $(a,b)$. If $f(a) = f(b)$, then $\exists c\in (a,b)$ s.t. $f'(c) = 0$.
\begin{proof}
	By EVT, since $f$ is continuous on a compact set, then $f$ attains a maximum and a minimum. If both extremums occur at the endpoints, then $f$ is necessarily a constant function and $f'(x) = 0$ on $(a,b)$.\\
	If either the maximum or minimum occurs at some point $c \in (a,b)$, then it follows from the Interior Extremum Theorem that $f'(c) = 0$.
\end{proof}

\theorem[Mean Value Theorem] If $f: [a,b] \rightarrow \real$ is continuous on $[a,b]$ and differentiable on $(a,b)$, then $\exists c \in (a,b)$ s.t.
$$f'(c) = \frac{f(b) - f(a)}{b-a}$$
\begin{proof}
	Consider 
	$$d(x) = f(x) - \left[ \left(\frac{f(b) - f(a)}{b-a}\right) (x-a) + f(a) \right]$$
	We know $d$ is continuous on $[a,b]$ and differentiable on $(a,b)$. Also, $d(a) = d(b) = 0$. \\
	By Rolle's Theorem, $\exists c \in (a,b)$ s.t. $d'(c) = 0 \implies f'(c) = \frac{f(b) - f(a)}{b-a}$.
\end{proof}

\corollary If $f: (a,b) \rightarrow \real$ is differentiable and $f'(x) = 0$ for all $x \in (a,b)$, then $f$ is constant on $(a,b)$.
\begin{proof}
	Assume $x, y \in (a,b)$ and $x < y$. We set $c \in (x,y)$, then by Mean Value Theorem,
	$$0 = f'(c) = \frac{f(y) - f(x)}{y-x} \implies f(y) - f(x) = 0$$
\end{proof}
	
\corollary If $f: (a,b) \rightarrow \real$ is differentiable and $f'(x) = g'(x)$ for all $x \in (a,b)$, then $f(x) = g(x) + c$ for some $c \in \real$.
\begin{proof}
	Apply the previous corollary to the function $h(x) = f(x) - g(x)$.
\end{proof}

\theorem[Generalized Mean Value Theorem] If $f, g: [a,b] \rightarrow \real$ are continuous on $[a,b]$ and differentiable on $(a,b)$, then $\exists c \in (a,b)$ s.t.
$$[f(b) - f(a)]g'(c) = [g(b) - g(a)]f'(c)$$
If $g'$ is never zero on $(a,b)$, then
$$\frac{f'(c)}{g'(c)} = \frac{f(b) - f(a)}{g(b) - g(a)}$$
\begin{proof}
	Apply the Mean Value Theorem to the function $h(x) = [f(b) - f(a)]g(x) - [g(b)-g(a)]f(x)$.
\end{proof}

\theorem[L'Hospital's Rule: 0/0 case] Suppose $f, g$ are continuous on $I$ with $a \in I$ and are differentiable on $I \setminus \{a\}$. If $f(a) = g(a) = 0$ and $\forall x \neq a, g'(x) \neq 0$, then
$$\underset{x \rightarrow a}{\lim} \frac{f'(x)}{g'(x)} = L \implies \underset{x \rightarrow a}{\lim}\frac{f(x)}{g(x)} = L$$
\begin{proof}
	Since $\underset{x \rightarrow a}{\lim}\frac{f'(x)}{g'(x)} = L$, then for all $\epsilon > 0, \exists \delta > 0$ s.t.
	$$x \in (a - \delta, a + \delta) \implies \bignorm{\frac{f'(x)}{g'(x)} - L} < \epsilon$$
	By the Generalized Mean Value Theorem, for every $y \in (a, a + \delta), \exists x \in (a, y)$ s.t.
	$$\frac{f'(c)}{g'(c)} = \frac{f(b) - f(a)}{g(b) - g(a)} = \frac{f(y)}{g(y)}$$
	and thus
	$$\bignorm{\frac{f(y)}{g(y)} - L} = \bignorm{\frac{f'(x)}{g'(x)} - L} < \epsilon$$
\end{proof}

\theorem[L'Hospital's Rule: $\infty/\infty$ case] Suppose $f,g$ are differentiable on $(a,b)$ and $g'(x) \neq 0$ for all $x \in (a,b)$. If $\underset{x \rightarrow a}{\lim} g(x) = \infty$ or $-\infty$, then
$$\underset{x \rightarrow a}{\lim} \frac{f'(x)}{g'(x)} = L \implies \underset{x \rightarrow a}{\lim}\frac{f(x)}{g(x)} = L$$

\section{Sequences and Series of Functions}
\subsection{Uniform Convergence of a Sequence of Functions}
\definition[pointwise convergence] For each $n \in \mb{N}$, let $f_n$ be a function defined on a set $A \subseteq \real$. If $\forall x \in A, f_n(x) \rightarrow f(x)$ for some function $f$, then sequence $(f_n)$ of functions \under{converges pointwise} on $A$ to $f$.\\
 We can write $f_n \rightarrow f, \lim f_n = f$, or $\limit f_n(x) = f(x)$.
 
 \example Consider $f_n: \real \rightarrow \real$
 $$f_n(x) = \frac{x^2 + nx}{n}$$
 We can compute
 $$\limit f_n(x) = \limit \frac{x^2 + nx}{n} = \limit \frac{x^2}{n} + x = x$$
 Thus, $(f_n)$ converges pointwise to $f(x) = x$ on $\real$.
 
 \example Consider $f_n: [0,1] \rightarrow \real$
 $$f_n(x) = x^n$$
 If $0 \leq x < 1, x^n \rightarrow 0$. If $x = 1, x^n \rightarrow 1$. It follows that $f_n \rightarrow f$ pointwise on $[0,1]$ where
 $$f(x) = \begin{cases}
 	0, & 0\leq x < 1\\
 	1, & x = 1
 \end{cases}$$
 Note that pointwise convergent sequence of continuous functions may converge to a non-continuous function.
 
 \definition[uniformly convergence] Let $(f_n)$ be a sequence of functions defined on a set $A \subseteq \real$, then $(f_n)$ \under{converges uniformly} on $A$ to a limit function $f$ defined on $A$ if
 $$\forall \epsilon > 0, \exists N \text{ s.t. } \forall n \geq N, \forall x \in A, |f(x) - f_n(x)| < \epsilon$$
 \remark
 This is a \tb{stronger} notion of convergence.

\example Consider $f_n: \real \rightarrow \real$
 $$f_n(x) = \frac{x^2 + nx}{n}$$
 which converges pointwise on $\real$ to $f(x) = x$. But the convergence is not uniform, since
 $$|f_n(x) - f(x)| = \bignorm{\frac{x^2 + nx}{n} - x} = \frac{x^2}{n}$$
 In order to force $|f_n(x) - f(x)| < \epsilon$, we need $N < \frac{x^2}{\epsilon}$. Although it is possible to do for each $x \in \real$, there is no way to choose a single value of $N$ that will work for all values of $x$ at the same time. \\
 On the other hand, we can show that $f_n \rightarrow f$ uniformly on the set $[-b, b]$.

\property[Cauchy Criterion for Uniform Convergence]
A sequence of functions $(f_n)$ defined on a set $A \subseteq \real$ converges uniformly on $A$ iff
$$\forall \epsilon > 0, \exists N \text{ s.t. } \forall x \in A, \forall m,n \geq N, |f_n(x) - f_m(x)| < \epsilon$$

\theorem[Continuous Limit Theorem] Let $(f_n)$ be a sequence of functions defined on $A \subseteq \real$ that converges uniformly on $A$ to a function $f$. If each $f_n$ is continuous at $c \in A$, then $f$ is continuous at $c$.
\begin{proof}
	Let $\epsilon >0$ and fix $c \in A$. Choose $N$ s.t.
	$$|f_N(x) - f(x)| < \frac{\epsilon}{3}, \forall x \in A$$
	Since $f_N$ is continuous, then $\exists \delta > 0$ s.t.
	$$|x - c| < \delta \implies |f_N(x) - f_N(c)| < \frac{\epsilon}{3}$$
	Thus,
	\begin{align*}
		|f(x) - f(c) | &= |f(x) - f_N(x) + f_N(x) - f_N(c) + f_N(c) - f(c)|\\
		&\leq |f(x) - f_N(x)| + |f_N(x) - f_N(c)| + |f_N(c) - f(x)|\\
		&< \frac{\epsilon}{3} + \frac{\epsilon}{3} + \frac{\epsilon}{3} \\
		&= \epsilon
	\end{align*}
\end{proof}
Hence $f$ is continuous at $c \in A$.
\property (Algebraic Limit Theorem for Uniform Convergence) Suppose $(f_n), (g_n)$ are uniformly convergent on $A$, then
\begin{enumerate}
	\item $(cf_n + g_n)$ is uniformly convergent on $A$
	\item If $\exists M > 0$ s.t. $|f_n| \leq M$ and $|g_n| \leq M$, then $(f_ng_n)$ is uniformly convergent.
\end{enumerate}
\begin{proof}
(1) Obvious.\\
(2) Let $\epsilon > 0$. Since $(f_n), (g_n)$ are uniformly convergent on $A$, then $\exists N$ s.t. $\forall m, n \geq N, |f_n(x) - f_m(x)| < \frac{\epsilon}{2M}$ and $g_n(x) - g_m(x) < \frac{\epsilon}{2M}$. Using Cauchy criterion, we have
\begin{align*}
	|f_m(x)g_m(x) - f_n(x)g_n(x)| &= |f_m(x)g_m(x) -f_m(x)g_n(x) + f_m(x)g_n(x) - f_n(x)g_n(x)| \\
	&\leq |f_m(x)||g_m(x)-g_n(x)| + |g_n(x)||f_m(x) - f_n(x)|\\
	&\leq M(|g_m(x) - g_n(x)| + |f_m(x) - f_n(x)| \\
	&< M(\frac{\epsilon}{M}) \\
	&= \epsilon
\end{align*}
So $(f_ng_n)$ is uniformly convergent.
\end{proof}

\subsection{Uniform Convergence and Differentiation}
\theorem[Differentiable Limit Theorem] Let $f_n \rightarrow f$ pointwisely on $[a,b]$ and assume each $f_n$ is differentiable. If $(f_n')$ converges uniformly on $[a,b]$ to a function $g$, then the function $f$ is differentiable and $f' = g$.
\theorem Let $(f_n)$ be a sequence of differentiable functions defined on $[a,b]$ and assume $(f_n')$ converges uniformly on $[a,b]$. If $\exists x_0 \in [a,b]$ s.t. $f_n(x_0)$ is convergent, then $(f_n)$ converges uniformly on $[a,b]$.
\theorem Let $(f_n)$ be a sequence of differentiable functions defined on $[a,b]$ and assume $(f_n')$ converges uniformly on $[a,b]$. If $\exists x_0 \in [a,b]$ s.t. $f_n(x_0)$ is convergent, then $(f_n)$ converges uniformly on $[a,b]$. Moreover, the limit function $f = \lim f_n$ is differentiable and $f' = g$.


\subsection{Series of Functions}

\definition[pointwise convergence] For each $n \in \mb{N}$, let $f_n$ and $f$ be functions defined on a set $A \subseteq \real$. The infinite series
$$\sum_{n=1}^\infty f_n(x) = f_1(x) + f_2(x) + f_3(x) + \hdots$$
\under{converges pointwise} on $A$ to $f(x)$ if the sequence $s_k(x)$ of partial sums defined by
$$s_k(x) = f_1(x) + f_2(x) + \hdots + f_k(x)$$
converges pointwise to $f(x)$. \\
\definition[uniform convergence]
The series \under{converges uniformly} on $A$ to $f$ if the sequence $s_k(x)$ converges uniformly on $A$ to $f(x)$.\\
In either case, we write $f = \sum_{n=1}^\infty f_n$ or $f(x) = \sum_{n=1}^\infty f_n(x)$.





\end{document}

