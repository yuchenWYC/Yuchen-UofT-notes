\documentclass[11pt]{article}
% Libraries.
\usepackage{amsmath}
\usepackage{amssymb}
\usepackage{pgfplots}
\usepackage{graphicx}
\usepackage{enumitem}
\usepackage{hyperref}
\usepackage{fancyhdr}
\usepackage{perpage}
\usepackage{float}
\usepackage{esint}
\usepackage[margin=1.5cm]{geometry}
\usepackage{../raina}

% Property settings.
\MakePerPage{footnote}
\pagestyle{headings}

% Attr.
\title{STA414\\ Lecture Notes}
\author{Yuchen Wang}
\date{\today}

\begin{document}
    \maketitle
    \tableofcontents
    \newpage

\section{Chapter 1: Real Numbers}
\subsection{Discussion: The Irrationality of $\sqrt{2}$}
If we make natural numbers $\mb{N}$ closed under subtraction, we obtain 
$$\mb{Z} = \{ \hdots, -1, 0, 1, \hdots \}$$
If we take the closure of $\mb{Z}$ under division by non-zero numbers, we obtain
$$\mb{Q} = \{\frac{m}{n}: m \in \mb{Z}, n \in \mb{N}, (m,n) = 1 \}$$
\remark
$(m, n) = 1$ means that if $d \in \mb{N}$ divides both $m$ and $n$, then $d = 1$.

\theorem
There is no $r \in \mb{Q}$ s.t. $r^2 = 2$.

\begin{proof}
	Assume for contradiction that there are $m \in \mb{Z}. n \in \mb{N}$ s.t. $\frac{m}{n} = \sqrt{2}$ and $(m,n) = 1$. \\
	Then $m^2 = 2n^2$ so that $m^2$ is an even complete square. \\
	Suppose $m = p_1 \hdots p_r$ where $p_i$s are prime numbers. Then $2n^2 = m^2 = p_1^2 \hdots p_r^2 \implies p_i^2 = 2^2$. \\
	Then $4 | m^2$ and $2|n^2$, so $n$ has to be even. Therefore both $m$ and $n$ are even. \\
	Then $2 | m$ and $2|n$, which leads to a contradiction that if $d \in \mb{N}$ divides both $m$ and $n$, then $d = 1$.
\end{proof}

\subsection{Preliminaries}
\definition[set]
A \under{set} is any collection of objects.
\definition[function]
Given two sets $A$ and $B$, a \under{function} from $A$ to $B$ is a rule or mapping that takes each element $x \in A$ and associates with it a single element of $B$. In this case, we write $(f: A \rightarrow B)$. It is the set of pairs $(A, B) \in A \times B$ s.t.
\begin{enumerate}
	\item If $(x,y_1) \in f$ and $(x,y_2) \in f$, then $y_1 = y_2$.
	\item For all $x \in A$, there is some $y \in B$ s.t. $f(x) = y$.
\end{enumerate}
The set $A$ is said to be the \under{domain} of $f$. The \under{range} of $f$ is not necessarily equal to $B$ but refers to the subset of $B$ given by $\{ y \in B: y = f(x)$ for some $x \in A \}$.

\example[absolute value function]
For every $x$, 
$$|x| = \begin{cases}
	x \quad & x \geq 0 \\
	-x \quad & x < 0
\end{cases}$$

\theorem[triangle inequality]
$$|x+y| \leq |x| + |y|$$
\begin{proof}
	\begin{align*}
		(x + y)^2 &= x^2 + y^2 + 2xy \\
		&\leq |x|^2 + |y|^2 + 2|x||y| \\
		&= (|x| + |y|)^2 \\
		\implies |x + y| &= \sqrt{(x+y)^2}\\
		&\leq \sqrt{(|x| + |y|)^2} \\
		&= ||x| + |y||\\
		&= |x| + |y|
	\end{align*}
\end{proof}

\definition[maximum and minimum]
Assume set $X \subseteq \real$. Then the maximum (minimum) of $X$ is an element $a \in X$ s.t. for all $x \in X, x \leq a (x \geq a)$.

\definition[least upper bound / supremum]
The \under{least upper bound} of X (denoted by $\sup(X)$) is a real number $a \in \real$ s.t.

\begin{enumerate}
	\item For all $x \in X, x \leq a$ (this means that $a$ is an upper bound for $X$)
	\item If $b$ is an upper bound for $X$, then $a \leq b$
\end{enumerate}

\example
\begin{align*}
	\max([0, 1]) &= 1 \\
	\min([0, 1]) &= 0 \\
	\sup((0,1)) &= 1 \\
	\sup(\real), \sup(\mb{N}) \, DNE
\end{align*}
\subsection{The axiom of completeness}

\definition[initial segment]
$X \subseteq \mb{Q}$ is said to be an \under{initial segment} if
\begin{enumerate}
	\item $X \neq \emptyset$
	\item For all $x, y \in \mb{Q}$, if $x < y$ and $y \in X$, then $x \in X$.
	\item $X \neq \mb{Q}$
\end{enumerate}

\definition[real numbers]
$\real = \{\sup(X): X$ is an initial segment of $\mb{Q}\}$ \\
Properties of $\real$:
\begin{enumerate}
	\item $\real$ is an \red{ordered field}
	\item \q
\end{enumerate}

\lemma[supremum] Suppose $A \subseteq \real$ and $s \in \real$ is an upper bound for $A$. If $\forall \epsilon > 0, \exists a \in A, a + \epsilon > s$, then $s = \sup(A)$
\begin{proof}
($\impliedby$) Assume for contradiction that $t \in \real$ is an upper bound for $A$ and $t < s$. \\
Let $\epsilon = \frac{s - t}{2}$. Obviously $\epsilon > 0$. \\
But then $\forall a \in A, a + \epsilon \leq t + \epsilon < s$, which is a contradiction. \\
($\implies$) Assume for contradiction that $\epsilon_0 > 0$ and $\forall a \in A, a + \epsilon \leq S$)\\
Then $\forall a \in A, a \leq S - \epsilon_0$. \\
So $s - \epsilon_0$ is an upper bound for $A$, which is a contradiction that $a + \epsilon > s$. 	
\end{proof}

\theorem[\blue{the axiom of completeness}] If $X \subset \real$ is bounded above, then $X$ has a least upper bound.
\begin{proof}
	For $x \in X$, let $Ax$ be the initial segment of $\mb{Q}$ corresponding to $x$. \\
	Since $X$ is bounded above, pick $b \in \real$ s.t. $\forall x \in X, x < b$. Then $b \notin \underset{x \in X}{\cup} Ax$. Note that $\underset{x \in X}{\cup} Ax$ is an initial segment of $\mb{Q}$. Then $\sup(\underset{x \in X}{\cup} Ax)$ is $\sup(X)$.
\end{proof}


\subsection{Consequences of Completeness}
\definition[nested sequence of sets]
Assume $\langle A_n: n \in \mb{N}\rangle$ is a sequence of sets. \\
$\langle A_n: n \in \mb{N}\rangle$ is said to be \under{nested} if
$$A_{n+1} \subseteq A_n$$

\theorem[Nested Interval Property]
Assume $\langle I_n: n \in \mb{N}\rangle$ is a nested sequence of closed intervals of $\real$. Then $$\underset{n}{\cap} I_n \neq \emptyset$$
\begin{proof}
Let $[a_n, b_n] = I_n$ where $a_n, b_n \in \real$. \\
Since $\langle I_n | n \in \mb{N} \rangle$ is nested,
$$a_n \leq a_{n+1} \leq b_{n+1} \leq b_n \quad (\dagger)$$ for all $n \in \mb{N}$\\
Let $A = \{ a_n: n \in \mb{N} \}$.\\
Note that $b_1$ is an upper bound for $A$. So $A$ has a supremum in $\real$. \\
We claim that $\sup(A) \in \underset{n}{\cap} I_n$.\\
By $(\dagger)$, for all $n \in \mb{N}, \sup(A) \leq b_n$\\
Obviously, for all $n \in \mb{N}, \sup(A) \geq a_n$\\
So $\forall n \in \mb{N}, a_n \leq \sup(A) \leq b_n$. \\
Therefore $\forall n \in \mb{N}, \sup(A) \in [a_n, b_n]$. 
\end{proof}

\example
$$\underset{n \in \mb{N}}{\cap} (0, \frac{1}{n}) = \emptyset$$
$$\underset{n \in \mb{N}}{\cap} [0, \frac{1}{n}] = \{0\}$$

\theorem[Archimedian Property]
\begin{enumerate}
	\item For every $y \in \real$, there is $n \in \mb{N}$ s.t. $y \leq n$.
	\item For every $y > 0$, there is $n \in \mb{N} s.t. \frac{1}{n} < y$.
\end{enumerate}
\begin{proof}
	(1) Assume for contradiction that $\mb{N}$ is bounded in $\real$. \\
	Let $\alpha = \sup(\mb{N})$. Then there is a natural number $n \in \mb{N}$ s.t. $n > \alpha - 1$. \\
	But then $n + 1 > (\alpha - 1) + 1 = \alpha$, which is a natural number greater than $\alpha$, contradiction. \\
	(2) Exercise.
\end{proof}

\theorem[density of $\mb{Q}$ in $\real$]
For every two real numbers $a$ and $b$ with $a < b$, there exists a rational number $r$ satisfying $a < r < b$.
\begin{proof}
Let $n \in \mb{N}$ s.t. $\frac{1}{n} < b - a, 1 < nb - na$. \\
Let $m \in \mb{Z}$ s.t. $na < m < nb$. \\
Then $a < \frac{m}{n} < b$. \\
Pick $r = \frac{m}{n}$ and we are done.
\end{proof}

\subsection{Cardinality}
"The size of a set``
\subsubsection{1-1 Correspondence}
\definition[one-to-one and onto] A function $f: A \rightarrow B$ is \under{one-to-one (1-1)} if $a_1 \neq a_2$ in $A$ implies that $f(a_1) \neq f(a_2)$ in $B$. The function $f$ is \under{onto} if, given any $b \in B$, it is possible to find an element $a \in A$ for which $f(a) = b$.
\remark
If a function $f: A \rightarrow B$ is both 1-1 and onto, then there is a 1-1 correspondence between two sets.

\definition[the same cardinality] The set $A$ \under{has the same cardinality as $B$} if there exists $f: A \rightarrow B$ that is $1-1$ and onto. In this case, we write $A \sim B$.

\subsubsection{Countable Sets} 
A set $A$ is \under{countable} if $\mb{N} \sim A$. An infinite set that is not countable is called an \under{uncountable} set.

\theorem
The set $\mc{Q}$ is countable.

\begin{proof}
	Set $A_1 = \{0\}$ and for each $n \geq 2$, let $A_n$ be the set given by
	$$A_n = \{ \pm \frac{p}{q}: \text{where }p,q \in \mb{N} \text{ are in lowest terms with} p + q = n \}$$ 
	e.g. $A_2 = \{ \frac{1}{1}, \frac{-1}{1}\}, A_3 = \{\frac{1}{2}, \frac{-1}{2}, \frac{2}{1}, \frac{-2}{1}\}$
	
\begin{figure}[H]
	\centering
	\includegraphics[scale=0.6]{p1.png}
\end{figure}
The above correspondence is onto because every rational number appears in the correspondence exactly once. The above correspondence is 1-1 because $A_N$ were constructed to be disjoint so that no rational number appears twice.
\end{proof}

\theorem
The set $\real$ is uncountable.

\begin{proof}
	Assume for contradiction that there does exist a bijection function $f: \mb{N} \rightarrow \real$. \\
	Let $x_1 = f(1), x_2 = f(2)$ and so on. Then since $f$ is onto, can write
	\begin{equation}
			\real = \{x_1, x_2, x_3, x_4, \hdots\}
	\end{equation}
	and be confident that every real number appears somewhere on the list. \\
	We will now use the Nested Interval Property to produce a real number that is not there.
	Let $I_1$ be a closed interval that does not contain $x_1$. given an interval $I_n$, construct $I_{n+1}$ to satisfy $I_{n+1} \subseteq I_n$ and $x_{n+1} \notin I_{n+1}$. \\
	If $x_{n_0}$ is some real number from the list in $(1)$, then we have $x_{n_0} \notin I_{n_0}$, and it follows that
	$$ x_{n_0} \notin \cap_{n=1}^\infty I_n$$
	Since we are assuming that the list in $(1)$ contains every real number, then
	$$\cap_{n=1}^\infty I_n = \emptyset$$
	However, the NIP asserts that $\cap_{n=1}^\infty I_n \neq \emptyset$, which is a contradiction.
\end{proof}

\theorem If $A \subseteq B$ and $B$ is countable, then $A$ is either countable or finite.

\theorem 
(i) If $A_1, A_2, \hdots, A_m$ are countable sets, then the union $A_1 \cup A_2 \cup \hdots \cup A_m$ is countable. \\
(ii) If $A_n$ is a countable set for each $n \in \mb{N}$, then $\cup_{n=1}^\infty$ is countable.

\subsection{Cantor's Theorem}
\theorem The open interval $(0,1) = \{x \in \real: 0 < x <1\}$ is uncountable.

\notation
Given a set $A$, the power set $P(A)$ refers to the collection of all subsets of $A$.

\theorem[Cantor's Theorem]
Given any set $A$, there does not exist a function $f: A \rightarrow P(A)$ that is onto.


\theorem[Schr\"oder-Bernstein Theorem]
Assume there are 1-1 maps from $A$ to $B$ and from $B$ to $A$. Then there is a bijection from $A$ to $B$.







\end{document}

