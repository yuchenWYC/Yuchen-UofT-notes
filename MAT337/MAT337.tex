\documentclass[11pt]{article}
% Libraries.

\usepackage{amsmath}
\usepackage{amssymb}
\usepackage{esint}
\usepackage[margin=3cm]{geometry}
%\usepackage{pgfplots}
\usepackage{graphicx}
\usepackage{enumitem}
\usepackage{hyperref}
\usepackage{fancyhdr}
\usepackage{perpage}
\usepackage[dvipsnames, pdftex]{xcolor}
\usepackage{float}
\usepackage{xargs}
\usepackage{/Users/raina/Desktop/uoft-notes/raina}
\usepackage[
	colorinlistoftodos,
	prependcaption,
	textsize=tiny
]{todonotes}


% Property settings.
\MakePerPage{footnote}
\pagestyle{headings}

% Attr.
\title{STA414\\ Lecture Notes}
\author{Yuchen Wang}
\date{\today}

\begin{document}
    \maketitle
    \tableofcontents
    \newpage

\section{Real Numbers}
\subsection{Discussion: The Irrationality of $\sqrt{2}$}
If we make natural numbers $\mb{N}$ closed under subtraction, we obtain 
$$\mb{Z} = \{ \hdots, -1, 0, 1, \hdots \}$$
If we take the closure of $\mb{Z}$ under division by non-zero numbers, we obtain
$$\mb{Q} = \{\frac{m}{n}: m \in \mb{Z}, n \in \mb{N}, (m,n) = 1 \}$$
\remark
$(m, n) = 1$ means that if $d \in \mb{N}$ divides both $m$ and $n$, then $d = 1$.

\theorem
There is no $r \in \mb{Q}$ s.t. $r^2 = 2$.

\begin{proof}
	Assume for contradiction that there are $m \in \mb{Z}. n \in \mb{N}$ s.t. $\frac{m}{n} = \sqrt{2}$ and $(m,n) = 1$. \\
	Then $m^2 = 2n^2$ so that $m^2$ is an even complete square. \\
	Suppose $m = p_1 \hdots p_r$ where $p_i$s are prime numbers. Then $2n^2 = m^2 = p_1^2 \hdots p_r^2 \implies p_i^2 = 2^2$. \\
	Then $4 | m^2$ and $2|n^2$, so $n$ has to be even. Therefore both $m$ and $n$ are even. \\
	Then $2 | m$ and $2|n$, which leads to a contradiction that if $d \in \mb{N}$ divides both $m$ and $n$, then $d = 1$.
\end{proof}

\subsection{Preliminaries}
\definition[set]
A \under{set} is any collection of objects.
\definition[function]
Given two sets $A$ and $B$, a \under{function} from $A$ to $B$ is a rule or mapping that takes each element $x \in A$ and associates with it a single element of $B$. In this case, we write $(f: A \rightarrow B)$. It is the set of pairs $(A, B) \in A \times B$ s.t.
\begin{enumerate}
	\item If $(x,y_1) \in f$ and $(x,y_2) \in f$, then $y_1 = y_2$.
	\item For all $x \in A$, there is some $y \in B$ s.t. $f(x) = y$.
\end{enumerate}
The set $A$ is said to be the \under{domain} of $f$. The \under{range} of $f$ is not necessarily equal to $B$ but refers to the subset of $B$ given by $\{ y \in B: y = f(x)$ for some $x \in A \}$.

\example[absolute value function]
For every $x$, 
$$|x| = \begin{cases}
	x \quad & x \geq 0 \\
	-x \quad & x < 0
\end{cases}$$

\theorem[triangle inequality]
$$|x+y| \leq |x| + |y|$$
\begin{proof}
	\begin{align*}
		(x + y)^2 &= x^2 + y^2 + 2xy \\
		&\leq |x|^2 + |y|^2 + 2|x||y| \\
		&= (|x| + |y|)^2 \\
		\implies |x + y| &= \sqrt{(x+y)^2}\\
		&\leq \sqrt{(|x| + |y|)^2} \\
		&= ||x| + |y||\\
		&= |x| + |y|
	\end{align*}
\end{proof}

\definition[maximum and minimum]
Assume set $X \subseteq \real$. Then the maximum (minimum) of $X$ is an element $a \in X$ s.t. for all $x \in X, x \leq a (x \geq a)$.

\definition[least upper bound / supremum]
The \under{least upper bound} of X (denoted by $\sup(X)$) is a real number $a \in \real$ s.t.

\begin{enumerate}
	\item For all $x \in X, x \leq a$ (this means that $a$ is an upper bound for $X$)
	\item If $b$ is an upper bound for $X$, then $a \leq b$
\end{enumerate}

\example
\begin{align*}
	\max([0, 1]) &= 1 \\
	\min([0, 1]) &= 0 \\
	\sup((0,1)) &= 1 \\
	\sup(\real), \sup(\mb{N}) \, DNE
\end{align*}
\subsection{The axiom of completeness}

\definition[initial segment]
$X \subseteq \mb{Q}$ is said to be an \under{initial segment} if
\begin{enumerate}
	\item $X \neq \emptyset$
	\item For all $x, y \in \mb{Q}$, if $x < y$ and $y \in X$, then $x \in X$.
	\item $X \neq \mb{Q}$
\end{enumerate}

\definition[real numbers]
$\real = \{\sup(X): X$ is an initial segment of $\mb{Q}\}$ \\
Properties of $\real$:
\begin{enumerate}
	\item $\real$ is an \red{ordered field}
	\item \q
\end{enumerate}

\lemma[supremum] Suppose $A \subseteq \real$ and $s \in \real$ is an upper bound for $A$. If $\forall \epsilon > 0, \exists a \in A, a + \epsilon > s$, then $s = \sup(A)$
\begin{proof}
($\impliedby$) Assume for contradiction that $t \in \real$ is an upper bound for $A$ and $t < s$. \\
Let $\epsilon = \frac{s - t}{2}$. Obviously $\epsilon > 0$. \\
But then $\forall a \in A, a + \epsilon \leq t + \epsilon < s$, which is a contradiction. \\
($\implies$) Assume for contradiction that $\epsilon_0 > 0$ and $\forall a \in A, a + \epsilon \leq S$)\\
Then $\forall a \in A, a \leq S - \epsilon_0$. \\
So $s - \epsilon_0$ is an upper bound for $A$, which is a contradiction that $a + \epsilon > s$. 	
\end{proof}

\theorem[\blue{the axiom of completeness}] If $X \subset \real$ is bounded above, then $X$ has a least upper bound.
\begin{proof}
	For $x \in X$, let $Ax$ be the initial segment of $\mb{Q}$ corresponding to $x$. \\
	Since $X$ is bounded above, pick $b \in \real$ s.t. $\forall x \in X, x < b$. Then $b \notin \underset{x \in X}{\cup} Ax$. Note that $\underset{x \in X}{\cup} Ax$ is an initial segment of $\mb{Q}$. Then $\sup(\underset{x \in X}{\cup} Ax)$ is $\sup(X)$.
\end{proof}


\subsection{Consequences of Completeness}
\definition[nested sequence of sets]
Assume $\langle A_n: n \in \mb{N}\rangle$ is a sequence of sets. \\
$\langle A_n: n \in \mb{N}\rangle$ is said to be \under{nested} if
$$A_{n+1} \subseteq A_n$$

\theorem[Nested Interval Property]
Assume $\langle I_n: n \in \mb{N}\rangle$ is a nested sequence of closed intervals of $\real$. Then $$\underset{n}{\cap} I_n \neq \emptyset$$
\begin{proof}
Let $[a_n, b_n] = I_n$ where $a_n, b_n \in \real$. \\
Since $\langle I_n | n \in \mb{N} \rangle$ is nested,
$$a_n \leq a_{n+1} \leq b_{n+1} \leq b_n \quad (\dagger)$$ for all $n \in \mb{N}$\\
Let $A = \{ a_n: n \in \mb{N} \}$.\\
Note that $b_1$ is an upper bound for $A$. So $A$ has a supremum in $\real$. \\
We claim that $\sup(A) \in \underset{n}{\cap} I_n$.\\
By $(\dagger)$, for all $n \in \mb{N}, \sup(A) \leq b_n$\\
Obviously, for all $n \in \mb{N}, \sup(A) \geq a_n$\\
So $\forall n \in \mb{N}, a_n \leq \sup(A) \leq b_n$. \\
Therefore $\forall n \in \mb{N}, \sup(A) \in [a_n, b_n]$. 
\end{proof}

\example
$$\underset{n \in \mb{N}}{\cap} (0, \frac{1}{n}) = \emptyset$$
$$\underset{n \in \mb{N}}{\cap} [0, \frac{1}{n}] = \{0\}$$

\theorem[Archimedian Property]
\begin{enumerate}
	\item For every $y \in \real$, there is $n \in \mb{N}$ s.t. $y \leq n$.
	\item For every $y > 0$, there is $n \in \mb{N} s.t. \frac{1}{n} < y$.
\end{enumerate}
\begin{proof}
	(1) Assume for contradiction that $\mb{N}$ is bounded in $\real$. \\
	Let $\alpha = \sup(\mb{N})$. Then there is a natural number $n \in \mb{N}$ s.t. $n > \alpha - 1$. \\
	But then $n + 1 > (\alpha - 1) + 1 = \alpha$, which is a natural number greater than $\alpha$, contradiction. \\
	(2) Exercise.
\end{proof}

\theorem[density of $\mb{Q}$ in $\real$]
For every two real numbers $a$ and $b$ with $a < b$, there exists a rational number $r$ satisfying $a < r < b$.
\begin{proof}
Let $n \in \mb{N}$ s.t. $\frac{1}{n} < b - a, 1 < nb - na$. \\
Let $m \in \mb{Z}$ s.t. $na < m < nb$. \\
Then $a < \frac{m}{n} < b$. \\
Pick $r = \frac{m}{n}$ and we are done.
\end{proof}

\subsection{Cardinality}
``The size of a set"
\subsubsection{1-1 Correspondence}
\definition[one-to-one and onto] A function $f: A \rightarrow B$ is \under{one-to-one (1-1)} if $a_1 \neq a_2$ in $A$ implies that $f(a_1) \neq f(a_2)$ in $B$. The function $f$ is \under{onto} if, given any $b \in B$, it is possible to find an element $a \in A$ for which $f(a) = b$.

\proposition
If $f: A \rightarrow B$ and $g: B \rightarrow C$ is 1-1, then $g \circ f: A \rightarrow C$ is 1-1.


\remark
If a function $f: A \rightarrow B$ is both 1-1 and onto, then there is a 1-1 correspondence between two sets.

\definition[the same cardinality] The set $A$ \under{has the same cardinality as $B$} if there exists $f: A \rightarrow B$ that is $1-1$ and onto. In this case, we write $A \sim B$.

\proposition
If $A \sim B, B \sim C$, then $A \sim C$

\proposition
If $Card(A) \leq Card(B) \leq Card(C)$, then $Card(A) \leq Card(C)$

\subsubsection{Countable Sets} 
A set $A$ is \under{countable} if $\mb{N} \sim A$. An infinite set that is not countable is called an \under{uncountable} set.

\theorem
The set $\mc{Q}$ is countable.

\begin{proof}
	Set $A_1 = \{0\}$ and for each $n \geq 2$, let $A_n$ be the set given by
	$$A_n = \{ \pm \frac{p}{q}: \text{where }p,q \in \mb{N} \text{ are in lowest terms with} p + q = n \}$$ 
	e.g. $A_2 = \{ \frac{1}{1}, \frac{-1}{1}\}, A_3 = \{\frac{1}{2}, \frac{-1}{2}, \frac{2}{1}, \frac{-2}{1}\}$
	
\begin{figure}[H]
	\centering
	\includegraphics[scale=0.6]{p1.png}
\end{figure}
The above correspondence is onto because every rational number appears in the correspondence exactly once. The above correspondence is 1-1 because $A_N$ were constructed to be disjoint so that no rational number appears twice.
\end{proof}

\theorem
The set $\real$ is uncountable.

\begin{proof}
	Assume for contradiction that there does exist a bijection function $f: \mb{N} \rightarrow \real$. \\
	Let $x_1 = f(1), x_2 = f(2)$ and so on. Then since $f$ is onto, can write
	\begin{equation}
			\real = \{x_1, x_2, x_3, x_4, \hdots\}
	\end{equation}
	and be confident that every real number appears somewhere on the list. \\
	We will now use the Nested Interval Property to produce a real number that is not there.
	Let $I_1$ be a closed interval that does not contain $x_1$. given an interval $I_n$, construct $I_{n+1}$ to satisfy $I_{n+1} \subseteq I_n$ and $x_{n+1} \notin I_{n+1}$. \\
	If $x_{n_0}$ is some real number from the list in $(1)$, then we have $x_{n_0} \notin I_{n_0}$, and it follows that
	$$ x_{n_0} \notin \cap_{n=1}^\infty I_n$$
	Since we are assuming that the list in $(1)$ contains every real number, then
	$$\cap_{n=1}^\infty I_n = \emptyset$$
	However, the NIP asserts that $\cap_{n=1}^\infty I_n \neq \emptyset$, which is a contradiction.
\end{proof}

\theorem If $A \subseteq B$ and $B$ is countable, then $A$ is either countable or finite.

\theorem 
(i) If $A_1, A_2, \hdots, A_m$ are countable sets, then the union $A_1 \cup A_2 \cup \hdots \cup A_m$ is countable. \\
(ii) If $A_n$ is a countable set for each $n \in \mb{N}$, then $\cup_{n=1}^\infty A_n$ is countable.

\theorem The open interval $(0,1) = \{x \in \real: 0 < x <1\}$ is uncountable.

\subsection{Cantor's Theorem}
\notation
Given a set $A$, the power set $P(A)$ refers to the collection of all subsets of $A$.

\theorem[Cantor's Theorem]
Given any set $A$, there does not exist a function $f: A \rightarrow P(A)$ that is onto.
\begin{proof}
	Assume, for contradiction, that $f: A \rightarrow P(A)$ is onto. For each element $a \in A$, $f(a)$ is a particular subset of $A$. The assumption that $f$ is onto means that every subset of $A$ appears as $f(a)$ for some $a \in A$. To arrive at a contradiction, we will produce a subset $B \subseteq A$ that is not equal to $f(a)$ for any $a \in A$.\\
	Construct $B$ using the following rule. For each element $a \in A$, consider the subset $f(a)$. This subset of $A$ may contain the element $a$ or it may not. This depends on the function $f$. If $f(a)$ does not contain $a$, then we include $a$ in our set $B$: Let 
	$$B = \{a \in A: a \notin f(a)\}$$
	Since we have assumed that our function $f: A \rightarrow P(A)$ is onto, it must be that $B = f(a')$ for some $a' \in A$.\\
	\tb{Case 1} $a' \in B$ \\
	Then $a' \notin f(a') = B$, a contradiction. \\
	\tb{Case 2} $a' \notin B$ \\
	Then $a' \in f(a') = B$, a contradiction.
\end{proof}


\theorem[Schr\"oder-Bernstein Theorem]
If there are 1-1 functions $f: A \rightarrow B$ and $h: B \rightarrow A$, then there is a bijection $g: A \rightarrow B$.

\begin{proof}
	\tb{Claim:} the statement of the theorem is equivalent to the following:\\
	If $B \subseteq A$ and $f: A \rightarrow B$ is $1-1$, then there is a bijection $g: A \rightarrow B$. \quad (*) \\\\
	\tb{proof of claim:}
	theorem $\implies$ (*): \\
	Take $h: X \rightarrow Y$ with $h(x) = x$, then $X \subseteq Y$. \\
	(*) $\implies$ theorem: \\
	Let $f: A \rightarrow B$ and $h: B \rightarrow A$ be 1-1 functions, as in the theorem. We need to show that there is bijection $g: A \rightarrow B$. \\
	Notice that $A \subseteq h(B)$ and $h \circ f: A \rightarrow h(B) $ is a 1-1 function. So by (*), there is a bijection $g_0: A \rightarrow h(B)$. \\
	But $h: B \rightarrow h(B)$ is also a bijection. So $g = h^{-1} \circ g_0: A \rightarrow B$ is a bijection (using the fact that bijections are closed under compositions). \\\\
	Now it suffices to prove (*). \\
	Assume set $X \subseteq Y$ and $f: Y \rightarrow X$. Let $W = \bigcup_{n=0}^\infty f^n(Y \setminus X)$.\\
	Define $g: Y \rightarrow X$ by:
	\begin{itemize}
		\item If $y \in W$, then $g(y) = f(y)$
		\item If $y \in Z:= Y \setminus W$, then $g(y) = y$
	\end{itemize}
	We need to show that $g: Y \rightarrow X$ is a well-defined bijection. \\
	Since $f$ is 1-1, for all $m < n$, $f^m(Y \setminus X) \cap f^n(Y \setminus X) = \emptyset$\\
	Note that 
	\begin{align*}
		Y \setminus W &= Y \setminus \bigcup_{n=0}^\infty f^n(Y \setminus X) \\
		&= [Y \setminus (Y \setminus X)] \setminus \bigcup_{n=1}^\infty f^n(Y \setminus X) \\
		&= X \setminus \bigcup_{n=1}^\infty f^n(Y \setminus X)
	\end{align*}
	Therefore for all $y \in Y, g(y) \in X$. \\
	(Show $g$ is 1-1) Now assume $y_1, y_2 \in Y$ and $g(y_1) = g(y_2)$. We show that $y_1 = y_2$. \\
	\tb{Case 1} $y_1, y_2 \in W$ \\
	Then $g(y_1) = g(y_2) \implies f(y_1) = f(y_2) \implies y_1 = y_2$. \\
	\tb{Case 2} $y_1 \in W$ but $y_2 \in Y\setminus W$ \\
	Then $g(y_1) = g(y_2) \implies f(y_1) = y_2$ \\
	Note that if $y_1 \in W$, then for some $n \geq 0, y_1 \in f^n(Y \setminus X)$ \\
	Then $y_2 \in f^{n+1}(Y \setminus X) \subseteq W$ \\
	So $y_2 \in W$, which leads to a contradiction. \\
	\tb{Case 3} $y_1, y_2$ are both in $Z:= Y \setminus W$\\
	Then $g(y_1) = g(y_2) \implies y_1 = y_2$. \\
	Therefore by case 1,2,3, $g$ is 1-1.\\
	(Show $g$ is onto) Let $x \in X$. We need to find $y \in Y$ s.t. $g(y) = X$. \\
	If $x \in Z$, take $y = x$. \\
	If $x \in \bigcup_{n=1}^\infty f^n(Y \setminus X)$, then fix $n \in \mb{N}$ s.t. $x \in f^n(Y \setminus X)$. \\
	But $f^n(Y \setminus X) = f(f^{n-1}(Y \setminus X))$ \\
	Pick $y \in f^{n-1}(Y \setminus X)$ s.t. $f(y) = x$. \\
	Then $y \in W$ and $g(y) = x$. Therefore $g$ is onto. 
\end{proof}



\section{Metric Spaces and the Baire Category Theorem}
\definition[metric and metric space] Given a set $X$, a function $d: X \times X \rightarrow \real$ is a \under{metric} on $X$ if for all $x, y \in X$:
\begin{enumerate}
	\item $d(x,y) \geq 0$ with $d(x,y) = 0$ if and only if $x = y$;
	\item $d(x,y) = d(y,x)$;
	\item for all $z \in X, d(x, y) \leq d(x,z) + d(z,y)$
\end{enumerate}
A \under{metric space} is a set $X$ together with a metric $d$.

\example
The set $\real$ considered with $d: \real^2 \rightarrow [0, \infty), (x,y) \mapsto |x - y|$ is a metric space.

\example
In general, $\real^n$ considered with the Euclidean distance is a metric space.
$$d(\vx, \vy) = \sqrt{\sum_{i=1}^n (x_i - y_i)^2}$$

\example
Let $x$ be a set. The \under{discrete metric} $d$ on $X$ is defined by
$$d(x,y) = \begin{cases}
	0 \quad x = y \\
	1 \quad x \neq y
\end{cases}$$

\paragraph{Fact} If $(X, d)$ is a metric space, $d'(x,y) = \max\{1, d(x,y)\}$ for all $x, y \in X$, then $(X, d')$ is also a metric space.

\example
Let $X = \{f: A \rightarrow \real\}$ \\
$$d(f,g) = \sup\{|f(x) - g(x)|: x \in A\}$$ if the supremum exists.

\subsection{Basic Definitions}
\definition
Let $(X, d)$ be a metric space. A sequence $(X_n) \subseteq X$ \under{converges} to an element $x \in X$ if $\forall \epsilon > 0, \exists N \in \mb{N}, n \geq N \implies d(x_n, x) < \epsilon$. \\\\
\tb{Key property: } If $\underset{n \rightarrow \infty}{\lim} x_n = x, \underset{n \rightarrow \infty}{\lim} x_n = y$, then $x = y$.
\begin{proof}
	WTS $d(x,y) = 0$ \\
	Let $\epsilon > 0$. We will show that $d(x,y) < \epsilon$. \\
	Since $\underset{n \rightarrow \infty}{\lim} x_n = x$, then $\exists N_1, \forall n \geq N_1, d(x_n, x) < \frac{\epsilon}{2}$ \\
	Since $\underset{n \rightarrow \infty}{\lim} x_n = y$, then $\exists N_2, \forall n \geq N_2, d(x_n, y) < \frac{\epsilon}{2}$ \\
	Take $n \geq \max(N_1, N_2)$, then $d(x,y) \leq d(x_n, x) + d(x_n, y) < \frac{\epsilon}{2} + \frac{\epsilon}{2} = \epsilon$.
\end{proof}

\proposition Suppose $(X, d)$ is a metric space, $(X, \tau)$ is a topological space, and $F \subseteq X$. If $\underset{n \rightarrow \infty}{\lim} x_n = x$, $(x_n) \subseteq F$ and $F$ is closed, then $x \in F$.
\begin{proof}
	Suppose $x \notin F$, i.e., $x \in X \setminus F$. \\
	Since $F$ is closed, then $X \setminus F$ is open, so there is $\epsilon > 0$ s.t. $B_\epsilon(x) \subseteq X \setminus F$. \\
	Let $N$ be such that $\forall n \geq N, d(x_n, x) < \epsilon$.\\
	 Then $x_n \in B_\epsilon(x)$, which implies that $(x_n) \subseteq X \setminus F$, a contradiction.
\end{proof}

\proposition Suppose $(X, d)$ is a metric space and $F \subseteq X$. If $F$ is not closed, then there exists $(x_n) \subseteq F$ and $x \notin F$ s.t. $\underset{n \rightarrow \infty}{\lim} x_n = x$.

\begin{proof}
	If $F$ is not closed, then $X \setminus F$ is not open, so there is $x \in X \setminus F$ s.t. $B_\epsilon(x) \not \subseteq X \setminus F$ for all $\epsilon > 0$. \\
	Take $x_n \in B_{1/n}(x) \setminus (X \setminus F)= B_{1/n}(x) \cap F$ for each $n \in \mb{N}$, then $(x_n) \subseteq F$ and $\underset{n \rightarrow \infty}{\lim} x_n = x$.
\end{proof}

\definition[Cauchy sequence]
 A sequence $(x_n)$ in a metric space $(x_n)$ in a metric space $(X, d)$ is a \under{Cauchy sequence} if $\forall \epsilon >0, \exists N \in \mb{N}, m, n \geq N \implies d(x_m, x_n) < \epsilon$.
 
\proposition
A convergent sequence is Cauchy. 
\begin{proof}
	Let $(x_n)$ be a convergent sequence, so that $\underset{n \rightarrow \infty}{\lim} x_n = x$. To check $(x_n)$ is Cauchy, let $\epsilon > 0$. We need to find $N$ s.t. $\forall m, n \geq N, d(x_n, x_m) < \epsilon$.\\
	Apply $\underset{n \rightarrow \infty}{\lim} x_n = x$ to $\frac{\epsilon}{2}$, we get $N$ s.t. $\forall n \geq N, d(x, x_n) < \frac{\epsilon}{2}$. \\
	Notice that $N$ works for Cauchy: \\
	Take $m, n \geq N$, then $$d(x_n, x_m) \leq d(x_n, x) + d(x, x_m) < \frac{\epsilon}{2} + \frac{\epsilon}{2} = \epsilon$$
\remark
When $X = \real$ with the usual metric, A Cauchy sequence is convergent (the converse is true). \\
In general not true. For example, $X = \real \setminus \{0\}, d(x,y) = |x - y|, (x_n) = \frac{1}{n}$.

\end{proof}

\definition[completeness of metric spaces]
A metric space $(X, d)$ is \under{complete} if every Cauchy sequence in $X$ converges to an element of $X$.

\example
$\real, d(x,y) = |x - y|$
\example
$(X, d), d$ discrete metric.
\example 
$C[0,1], d(f,g) = \underset{x \in [0, 1]}{\sup} |f(x) - g(x)| = ||f - g||_{\infty}$
\example
$(\mb{N}^{\mb{N}}, d), d((x_n), (y_n)) = \frac{1}{\min\{n: x_n \neq y_n\}}$
\\
where $\mb{N}^{\mb{N}} = \{x: \mb{N} \rightarrow \mb{N}\}$.

\definition
Let $(X, d_1)$ and $(Y, d_2)$ be metric spaces. A function $f: X \rightarrow Y$ is \under{continuous at $x \in X$} if $\forall \epsilon > 0, \exists \delta > 0, d_1(x,y) < \delta \implies d_2(f(x), f(y)) < \epsilon$.

\subsection{Topology on Metric Spaces}
\definition[$\epsilon$-neighbourhood]
Given $\epsilon > 0$ and an element $x$ in the metric space $(X, d)$, the \under{$\epsilon$-neighbourhood} of $x$ is the set $V_\epsilon(x) = \{y \in X: d(x, y) < \epsilon\}$

\definition[compactness]
A subset $K$ of a metric space $(X, d)$ is \under{compact} if every sequence in $K$ has a convergent subsequence that converges to a limit in $K$.

\definition[closure and interior]
Given a subset $E$ of a metric space $(X, d)$, the \under{closure} $\bar{E}$ is the union of $E$ together with its limit points. The \under{interior} of $E$ is denoted by $E^\circ$ and is defined as
$$E^\circ = \{x \in E: \exists V_\epsilon(x) \subseteq E\}$$

\remark
$(X, \tau), \tau \subseteq P(X), E \subseteq X$
\begin{enumerate}
	\item $\bar{E}$ = minimal closed superset of $E$ = $\bigcap\{H: H \text{ closed }, H \supseteq E\}$
	\item $E^\circ$ = maximal open subset of $E$ = $\bigcup\{U: U \text{ open }, U \subseteq E\}$
\end{enumerate}

\example $(X, d)$ is a metric space, $\tau_d$ is the topology determined by $d$: $U \in \tau_d$ iff $\forall x \in U, \exists \epsilon > 0, B_\epsilon(x) \subseteq U$ \\
$ F \subseteq X$

\begin{align}
	\bar{F} &= \{ x \in X: \forall \epsilon > 0, B_\epsilon(x) \cap F \neq \emptyset\} \\
	&= \bigcap \{ H: H \text{ closed } H \supseteq F\} \\
	&= \{ \underset{n \rightarrow \infty}{\lim} x_n:  (x_n) \subseteq F, \underset{n \rightarrow \infty}{\lim} x_n \text{ exists } \}
\end{align}

%TODO: I really, really cannot understand professor's proof...



%\begin{proof}
%\tb{Claim 1: } $$\bar{F} = \{ x \in X: \forall \epsilon > 0, B_\epsilon(x) \cap F \neq \emptyset\} := H$$
%%Note that $H$ is closed and $\supseteq F$ \\
%%Then $U = X \setminus H$ is open. Take $y \in X \setminus H$. \\
%%We need to find $\epsilon > 0$ s.t. $B_\epsilon(y) \subseteq X \setminus H$\\
%%Note that for a given $\epsilon > 0$, $$B_\epsilon(y) \cap F \neq \emptyset \equiv B_\epsilon(y) \cap \bar{F} \neq \emptyset$$
%%Note that $X \setminus B_\epsilon(y) \supseteq F$ is a closed superset of $F$.\\
%Suppose for contradiction that $H$ is not the minimal superset of $F$, so there is a closed set $H \subsetneq H$ s.t. $H' \supseteq F$. \\ 
%Notice that $V = X \setminus H'$ is open. \\
%Then take $y \in H \setminus H' \subseteq V$. So there is $\epsilon > 0, B_\epsilon(y) \subseteq V$. But then $B_\epsilon(y) \cap H' = \emptyset, B_\epsilon(y) \cap F = \emptyset, y \notin H \implies $ contradiction.
%
%
%\end{proof}


\begin{align}
	F^\circ &= \{ x \in X: \exists \epsilon > 0, B_\epsilon(x) \cap F \neq \emptyset\} \\
	&= \bigcup \{B_\epsilon(x): \epsilon > 0, x \in F, B_\epsilon(x) \subseteq F \} \\
\end{align}






\definition[density]
A set $A \subseteq X$ is \under{dense} in the metric space $(X, d)$ if $\bar{A} = X$. A subset $E$ of a metric space $(X, d)$ is \under{nowhere-dense} in $X$ if $\bar{E}^\circ$ is empty.

\subsection{Baire's Theorem}
\definition[nowhere-dense]
A set $E$ is \under{nowhere-dense} if $\bar{E}$ contains no nonempty open intervals.
\theorem[Baire's Theorem]
The set of real numbers $\real$ cannot be written as the countable union of nowhere-dense sets.


\subsection{The Baire Category Theorem}
\theorem Let $(X, d)$ be a complete metric space, and let $\{O_n\}$ be a countable collection of dense, open subsets of $X$. Then, $\bigcap_{n=1}^\infty\{O_n\}$ is not empty.
\begin{proof}
	
\end{proof}
\theorem[Baire Category Theorem]
A complete metric space is not the union of a countable collection of nowhere-dense sets.
\begin{proof}
	
\end{proof}

\theorem
The set
$$D = \{f \in C[0,1]: f'(x) \text{ exists for some } x \in [0, 1]\}$$ is a set of first category in $C[0,1]$.

\subsection{Topology of $(X, d)$}
\definition[open ball]
An \under{open ball} with radius $r$ and center $x$ is
$$B_r(x) = \{y \in X: d(x, y) < r\}$$

\definition[open set]
A set $U \subseteq X$ is \under{open} iff
$$\forall x \in U, \exists \epsilon > 0 \text{ s.t. } B_\epsilon(x) \subseteq U$$

\example $B_\epsilon(x)$ is open.
\begin{proof}
	Fix $x \in X$ and $\epsilon > 0$. We want to show: $\forall y \in B_\epsilon(x), \exists \delta > 0$ s.t. $B_\delta(y) \subseteq B_\epsilon(x)$. \\
	Take $y \in B_\epsilon(x)$, then $d(x,y) < \epsilon$. Take $\delta = \epsilon - d(x,y) > 0$. Take any $z \in B_\delta(y)$, we have
	$$d(x,z) \leq d(x,y) + d(y,z) < d(x,y) + \epsilon - d(x,y) = \epsilon$$
	Thus $z \in B_\epsilon(x)$ so $B_\delta(y) \subseteq B_\epsilon(x)$.
\end{proof}
\definition[topological space] A \under{topological space} is a pair $(X, \tau)$, where $X$ is a set and $\tau$ a subset of the power set of $X$ which we call open such that
\begin{enumerate}
	\item $\emptyset, X \in \tau$
	\item $U_1, \hdots, U_n \in \tau \implies \bigcap_{i=1}^n U_i \in \tau$
	\item $U_1, \hdots, U_n \in \tau \implies \bigcup_{i=1}^n U_i \in \tau$
\end{enumerate}

\example
$(X, \{\emptyset, X \})$

\example
$(X, P(X))$ is a \under{discrete topological space}, where $P(X)$ is the power set of $X$.
\example
Given $(X, d)$ a metric space, define $\tau_d: $ a set $U \in \tau_d \iff \forall x \in U, \exists \epsilon > 0, B_\epsilon(x) \subseteq U$. Then $\tau_d$ is a topology.
\begin{proof}
	(1) First, $\emptyset, X \in \tau_d$ since $\forall x \in \emptyset, B_1(x) \subseteq \emptyset$ and $\forall x \in X, B_1(x) \subseteq X$.\\
	Then suppose $U_1, \hdots, U_n \in \tau_d$.\\
	(2) we want to show:
	$$U = \bigcap_{i=1}^n U_i \in \tau_d \iff \forall x \in U, \exists \epsilon > 0 \text{ s.t. } B_\epsilon(x) \subseteq U$$
	Since $x \in U$, then $\forall i = 1, \hdots, n, x \in U_i: \exists \epsilon_i > 0$ s.t. $B_{\epsilon_i}(x) \subseteq U_i$. \\
	Take $\epsilon = \underset{1 \leq i \leq n}{\min} \epsilon_i$, thus $B_\epsilon(x) \subseteq U_i \, \forall i$. Hence $B_\epsilon(x) \subseteq U_i \subseteq U$. \\
	(3) We also want to show:
	$$\bigcup_{i=1}^n U_i \in \tau_d \iff \forall x \in U, \exists \epsilon > 0 \text{ s.t. } B_\epsilon(x) \subseteq U$$
	Let $x \in U$, then there is some $U_i$ s.t. $x \in U_i$. Since $U_i \in \tau_d$, then $\exists \epsilon >0$ s.t. $B_\epsilon(x) \subseteq U_i \subseteq U$.\\
	Therefore, $\tau_d$ is a topology.
\end{proof}

\definition
A \blue{subset} $F$ of a topological space $(X, \tau)$ is \under{closed} if $X \setminus F$ is open. \\
\property Given a topological space $(X, \tau)$ and a subset $F$ of it, we have: 
\begin{enumerate}
	\item $\emptyset, X$ are closed
	\item If $F_1, \hdots, F_n$ are closed, then $\bigcup_{i=1}^n F_i$ is closed
	\item If $F_1, \hdots, F_n$ are closed, then $\bigcap_{i=1}^n F_i$ is closed
\end{enumerate}

\definition[topological closure and interior]
Given a topological space $(X, \tau)$, where $\tau \subseteq P(X)$, and a set $F \subseteq X$, the \under{topological closure} of $F$ is the minimal closed superset of $F$, i.e.,
$$\bar{F} = \bigcap \{H: H \text{ is closed}, H \supseteq F\}$$
The \under{interior} of $F$ is the maximal open subset of $F$, i.e.,
$$F^\circ = \bigcap \{U: U \text{ is open}, U \subseteq F\}$$

\example
Given $(X, d)$ a metric space, define $\tau_d: $ a set $U \in \tau_d \iff \forall x \in U, \exists \epsilon > 0, B_\epsilon(x) \subseteq U$. Suppose $F \subseteq X$, then
$$\bar{F} = \{x \in X: \forall \epsilon > 0, B_\epsilon(x) \cap F \neq \emptyset\} = \{\limit x_n: (x_n) \subseteq F, \limit x_n \text{ exists}\}$$






\section{Sequences and Series}
\subsection{The Limit of a Sequence}
\definition[sequence]
A \under{sequence} is a function whose domain is $\mb{N}$






















\end{document}

